\documentclass{beamer}
\usepackage[english,francais]{babel}
\usepackage[utf8]{inputenc}
\usepackage{multicol}
\usepackage{bm}

\usepackage{graphicx}
\graphicspath{{./Forex_Credit_Equity/}}

\newcommand{\FIG}[3]{\includegraphics<#1>[width=#2]{#3}}

\newcommand{\FIGSCALE}[3]{\includegraphics<#1>[resolution=72dpi]{#3}}

\usetheme{Warsaw}
\title[Produits dérivés actions,change et credit]{Produits dérivés de change}
\author{Richard Guillemot}
\institute{DIFIQ}
\date{27 Mars 2016}

\begin{document}

\begin{frame}
\titlepage
\end{frame}

\begin{frame}
\frametitle{Taux de change \textbf{"Spot"} (as of 24 March 2016)}
\Huge
\begin{center}
\textbf{\textcolor<3>{red}{EUR}/\textcolor<4>{blue}{USD}}=1.1162
\end{center}
\huge
\begin{center}
\onslide<2->{1 \textcolor<3>{red}{euro} vaut 1.1162 \textcolor<4>{blue}{dollar}.}
\end{center}
\large
\vspace{0.5cm}
\begin{itemize}
\item[]<3-> \textbf{\textcolor<3>{red}{EUR (euro)}} est la \textcolor<3>{red}{devise étrangère ou devise 1}.
\item[]<4-> \textbf{\textcolor<4>{blue}{USD (dollar)}} est la \textcolor<4>{blue}{devise domestique ou devise 2}.
\end{itemize}
\end{frame}

\begin{frame}{Taux de change - Jargon}
\Huge
\begin{overprint}
\onslide<1>\begin{center}EUR/USD=1.1162\fontsize{60}{70}\selectfont\textcolor{white}{0}\huge\end{center}
\onslide<2>\begin{center}EUR/USD=1.1\fontsize{60}{70}\selectfont1\huge62\end{center}
\onslide<3>\begin{center}EUR/USD=1.116\fontsize{60}{70}\selectfont2\huge\end{center}
\vspace{0.5cm}
\end{overprint}
\large
\begin{overprint}
\onslide<1>On considère 5 chiffres significatifs dans un taux de change.
\onslide<2>Le 3\up{ème} chiffre en partant de la gauche est appelé \textbf{"Big Figure"}.
\onslide<3>Le 5\up{ème} chiffre en partant de la gauche est appelé \textbf{"pips"}.
\end{overprint}
\end{frame}

\begin{frame}
\frametitle{Livraison ou Settlement}
\begin{center}
\begin{figure}[h]
\FIG{1}{12cm}{figures/settlement-1.png}
\FIG{2}{12cm}{figures/settlement-2.png}
\FIG{3}{12cm}{figures/settlement-3.png}
\FIG{4}{12cm}{figures/settlement-4.png}
\end{figure}
\end{center}
\end{frame}

\begin{frame}
\frametitle{Taux de change \& Taux d'intérêts}
As of 24 Mars 2016:
\begin{center}
\begin{tabular}{|l|l|l|l|l|}
\hline
\textbf{FX} & 	&  & \textbf{IR} & \textbf{BS} \\
\hline
\hline
EURUSD & 1.1162 & EUR & -0.27\% & -0.29\%  \\
GBPUSD & 1.4142 & USD & 0.81\% & \\
USDCHF & 0.9767 & GBP & 0.61\% & -0.14\% \\
USDJPY & 112.64 & CHF & -0.84\% & -0.26\% \\
USDCNY & 6.5125 & JPY & -0.08\% & -0.56\% \\
	&	& CNY & 	& 3.78\% \\
\hline		
\end{tabular}
\end{center}
\end{frame}

\begin{frame}
\frametitle{Historique EURUSD}
\begin{center}
\begin{figure}[h]
\FIG{1}{10cm}{figures/EURUSDHisto.png}
\end{figure}
\end{center}
\end{frame}

\begin{frame}
\frametitle{Les différents types de taux d'intérêts}
\begin{figure}[h]
\centering \FIG{1-}{7cm}{figures/capi_actu.png}
\end{figure}
\begin{center}
\begin{tabular}{|c|c|c|}
\hline
&\textbf{C}apitalisation&\textbf{A}ctualisation \\
\hline
  Taux Linéaire & \visible<2->{$1+\delta R^L$} &  \visible<3->{$\frac{1}{1+\delta R^L}$} \\
  Taux Actuariel & \visible<4->{$(1+\frac{\delta}{N}R^A)^N$} &  \visible<5->{$\frac{1}{(1+\frac{\delta}{N}R^A)^N}$} \\
  Taux Continu & \visible<6->{$e^{\delta R^C}$} & \visible<7->{$e^{-\delta R^C}$} \\
\hline
\end{tabular}
\end{center}
\visible<8->{
\[
	R^C < R^A < R^L
\]}
\end{frame}


\begin{frame}
\frametitle{Taux de change \textbf{"Forward"}}
\large
\begin{overprint}
\onslide<1>Comment garantir un taux de change à une date future \textbf{T} ?\\ Et à quel taux \textbf{X}.
\onslide<2>\textbf{Prêt} en $t$ de \textcolor{red}{$\frac{1}{1+\delta R^{EUR}}$ euros}.\\Remboursé en $T$ avec les intérêts, c'est à dire \textcolor{red}{1 euro}.
\onslide<3>\textbf{Change} \textcolor{red}{$\frac{1}{1+\delta R^{EUR}}$ euros} contre \textcolor{blue}{$\frac{S}{1+\delta R^{EUR}}$dollars}.
\onslide<4>\textbf{Emprunt} en $t$ de \textcolor{blue}{$\frac{S}{1+\delta R^{EUR}}$} dollars\\ Remboursé en $T$ avec les intérêts, c'est à dire \textcolor{blue}{$S\frac{1+\delta R^{USD}}{1+\delta R^{EUR}}$ dollars}.
\onslide<5>\[X=S\frac{1+\delta R^{USD}}{1+\delta R^{EUR}}\]
\end{overprint}
\begin{figure}
\begin{overprint}
\FIG{1}{11cm}{figures/fxfwd-1.png}
\FIG{2}{11cm}{figures/fxfwd-2.png}
\FIG{3}{11cm}{figures/fxfwd-3.png}
\FIG{4}{11cm}{figures/fxfwd-4.png}
\FIG{5}{11cm}{figures/fxfwd-5.png}
\end{overprint}
\end{figure}
\end{frame}

\begin{frame}
\small
\frametitle{Taux de change \textbf{"Forward"} - Récapitulatif}
As of 24 March 2016:
\begin{center}
\begin{tabular}{|l|l|l|l|}
\hline
\textbf{Notation} & \textbf{Description} & \textbf{Formule} & \textbf{Valeur} \\
\hline
\hline
$\delta$ & \visible<2->{Maturité du forward} & \visible<2->{$T-(t+2D)$} & \visible<2->{1 an = 365 jours} \\
$R^{EUR}$ & \visible<3->{Taux euro.} &  & \visible<3->{-0.27\%} \\
$R^{USD}$ & \visible<4->{Taux dollar.} &  & \visible<4->{0.81\%} \\
$S$ & \visible<5->{Taux de change spot.} &  & \visible<5->{1.1162} \\
$X$ & \visible<6->{Forward de change.} &  \visible<6->{$S\frac{1+\delta R^{USD}}{1+\delta R^{EUR}}$} & \visible<6->{??} \\
\hline
\end{tabular}
\end{center}
\[
\uncover<7->{X=} \uncover<8->{1.1162\times \frac{1+ 0.81\%}{1-0.27\%}\uncover<9->{=1.1283}} 
\]
\onslide<10->{Soit \textbf{121} points de base d'écart positif par rapport au taux spot.}
\end{frame}

\begin{frame}
\frametitle{Quizz}
Si on vend 100 millons euro dans 1 an au taux spot au lieu d'utiliser le taux foward précedemment calculé:\\
\vspace{0.5cm}
a) On perd 1077 kEUR \uncover<2>{\textcolor{green}{\textbf{VRAI}}}\\
b) On gagne 107,7 kEUR \uncover<2>{\textcolor{red}{\textbf{FAUX}}}\\
c) On perd 10.77 millions d'euros. \uncover<2>{\textcolor{red}{\textbf{FAUX}}}\\
d) On gagne 107,7 kEUR. \uncover<2>{\textcolor{red}{\textbf{FAUX}}}\\
\vspace{0.5cm}
\uncover<2>{On emprunte à 0.81\% en dollars et on prête à -0.27\% en euros !!!}
\end{frame}

\begin{frame}{Le taux monétaire, dépôt ou money market}

Voici l'échéancier de l'EURIBOR 6M:
\begin{figure}[h]
\vspace{2mm}
\FIG{1}{7cm}{figures/schema_euribor.jpg} 
\vspace{1mm}
\end{figure}
Le taux monétaire est défini comme:
\[
R=L(t,T_\text{start}, T_\text{end})=\frac{1}{\delta}\left(\frac{B(t,T_\text{start})}{B(t,T_\text{end})}-1\right)
\]
La période est caculée avec la convention Act 360:
\[
\delta = \frac{T_\text{end} - T_\text{start}} {360}
\]
\end{frame}

\begin{frame}{Réplication d'un emprunt futur}
\begin{overprint}
\begin{center}
\begin{figure}[h]
\FIG{1}{3in}{figures/fwd_replic1.png}
\FIG{2}{3in}{figures/fwd_replic2.png}
\FIG{3-}{3in}{figures/fwd_replic3.png}
\end{figure}
\end{center}
\onslide<1> Soit un emprunt à taux fixe qui démarre dans le futur. Nous allons le répliquer par 2 emprunts qui démarrent aujourd'hui.
\onslide<2> On prête aujourd'hui $B(t,T_{start})$ qui sera remboursé avec les intérêts en $T_{start}$ par un flux de 1.
\onslide<3> On emprunte aujourd'hui $(1+\delta K)B(t,T_{end})$ qui nous sera remboursé avec les intérêts en $T_{end}$ par un flux de $1+\delta K$.
\onslide<4> Il n'y a maintenant plus de flux futurs nous allons donc calculer le taux fixe $K^*$ qui égalise les flux aujourd'hui :
\[
K^*=R(t,T_{start}, T_{end})=\frac{1}{\delta}\left(\frac{B(t,T_{start})}{B(t,T_{end})}-1\right)
\]
\end{overprint}
\end{frame}

\begin{frame}{Obligation à taux variable}
\begin{overprint}
\begin{center}
\begin{figure}[h]
\FIG{1-}{11cm}{figures/bond_variable.png}
\end{figure}
\end{center}
\onslide<1>On calcule la valeur de l'obligation à taux variable:\\
\[
P = \sum_{i=1}^{n}\delta_i\times R(T_{i-1},T_{i})\times  B(t,T_i)+B(t,T_{n})
\]
\onslide<2>On estime la valeur actuelle du taux variable en utilisant le taux forward.\\
\[
P = \sum_{i=1}^{n}\delta_i\times \frac{1}{\delta_i}\left(\frac{B(t,T_{i-1})}{B(t,T_i)}-1\right)   \times  B(t,T_i)+B(t,T_{n})
\]
\onslide<3>\[
P = \sum_{i=1}^{n} \left(B(t,T_{i-1})-B(t,T_i)\right)+B(t,T_{n})
\]
\onslide<4-5>\[
P = 1-\only<4>{B(t,T_n)+B(t,T_n)}\only<5>{\cancel{B(t,T_n)}+\cancel{B(t,T_n)}}
\]
\onslide<6>\[
P = 1
\]
\textbf{La valeur d'une obligation à taux variable (sans marge) est insensible au niveau des taux (les jours de paiement de ses coupons).}
\end{overprint}
\end{frame}


\begin{frame}
\frametitle{Le swap de devises ou Cross-Currency Swap}
\large
\begin{overprint}
\onslide<1>On considère l'échéancier d'un swap standard.
\onslide<2>On échange en $t$+2D ouvrés \textcolor{blue}{$N^{USD}$} avec sa contrevaleur \textcolor{red}{$N^{EUR}$}.\\ On fera l'échange inverse à la maturité du swap $T$.
\onslide<3>On reçoit une jambe variable euro en contrepartie d'une jambe variable dollar.
\onslide<4>En pratique il faut retirer la \textbf{\textcolor{red}{marge de basis m}} à la jambe EUR pour mettre le swap au pair (valeur nulle).
\onslide<5>Un swap de devises d'un seule période est un foward de change de nominal \textcolor{red}{$N^{EUR}(1+\delta (L^{EUR}-\textbf{m}))$}.
\end{overprint}
\begin{figure}
\begin{overprint}
\FIG{1}{11cm}{figures/xccyswap-1.png}
\FIG{2}{11cm}{figures/xccyswap-2.png}
\FIG{3}{11cm}{figures/xccyswap-3.png}
\FIG{4}{11cm}{figures/xccyswap-4.png}
\FIG{5}{11cm}{figures/xccyswap-singleflow.png}
\end{overprint}
\end{figure}
\end{frame}

\begin{frame}
\frametitle{Taux de change \textbf{Forward} et marge de basis.}
\Huge
\[
	X=S\frac{1+\delta R^{USD}}{1+\delta (R^{EUR}-\textbf{m})}
\]
\Large
As of 24 March 2016:
\huge
\begin{center}
$m=29$ bps
\end{center}
\end{frame}

\begin{frame}
\frametitle{Delta de change et position de change}
\begin{itemize}
\item[-] Le \textbf{delta de change} est la sensibilité ou la dérivée au taux de change de la valeur d'un portefeuille en devise domestique.\\
\vspace{0.5cm}
\[
\Delta_{FX}=\frac{\partial \prod^d}{\partial S}
\]
\item[-] La \textbf{position de change} correspond au nominaux $N^i$  équivalents au portefeuille dans chacune des devises. Elle indique la taille des opérations de change "Spot" nécessaires pour neutraliser le risque.

\end{itemize}
\end{frame}

\begin{frame}
\frametitle{Delta de change et position de change}
\small
Illustration avec les 2 devises euro et dollar:\\ 
\vspace{0.5cm}
\begin{tabular}{|l|l l|}
\hline
Taux de change&$S$&$= EUR/USD$\\
Valeur du portefeuille en dollar&$\Pi^{USD}$&$=N^{EUR} \times S + N^{USD}$\\
Delta de change&$\Delta_{EURUSD}$&$=N^{EUR}$\\
Position de change&&$(N^{EUR},N^{USD})$\\
\hline
\end{tabular}
\end{frame}

\begin{frame}
\frametitle{Exercice}
On reprend les données du premier exemple la marge de basis m est égale à 29 points de base:\\
\begin{itemize}
\item[-] \textbf{Opération 1}: Une banque francaise doit recevoir de son client \textcolor{blue}{113 millions de dollars} contre \textcolor{red}{100 millions d'euros} dans 1 an.
\item[-] \textbf{Opération 2}: Sa filliale américaine doit recevoir de son client \textcolor{red}{89 millions d'euros} contre \textcolor{blue}{100 millions de dollars} dans 1 an.
\end{itemize}
Pour chacune des 2 opérations et pour le portefeuille total de la banque:\\
\begin{enumerate}
\item Quel est le Profit \& Loss (PNL) pour la banque ?
\item Quels sont les Delta FX et la position de change ?
\item Quelles sont la sensibilités à un mouvement de 1 point de base des taux euro, dollar et de la marge de basis ?
\item Quelles opérations doit réaliser la banque pour neutraliser son risque de change ?
\end{enumerate}
\end{frame}

\begin{frame}{Exercice - Solution}

\begin{tabular}{|c|r|r|r|l|}
\hline
&\textbf{Cas 1}&\textbf{Cas 2}&\textbf{TOTAL}& \\
\hline
\hline
\textbf{PNL EUR} &-142&728&491&kEUR\\
\textbf{PNL USD} &-158&706&546&kUSD\\
\hline
\hline
\textbf{Delta FX} &-100.56&89.50&-11.06&Mios EUR\\
\textbf{Sensi taux EUR} &10.05&-8.85&1.11&kEUR/bp\\
\textbf{Sensi taux USD} &-11.21&9.92&-1.29&kUSD/bp\\
\textbf{Sensi basis} &-10.05&8.85&-1.11&kEUR/bp\\
\hline
\hline
\textbf{NEUR} &-100.56&89.50&-11.07&Mios EUR/bp\\
\textbf{NUSD} &112.09&-99.20&12.90&Mios USD/bp\\
\hline
\end{tabular}
\\
\vspace{0.5cm}
Pour se couvrir, \uncover<2>{il faut vendre 12.90 millions de dollars.}
\end{frame}

\begin{frame}{Option de change}
Une \textbf{option de change} est un contrat asymétrique par lequel à une date future T:\\
\vspace{0.5cm}
\begin{overprint}
\onslide<1>\begin{itemize}
\item La contrepartie \textbf{vendeuse s'engage} à recevoir un montant \textcolor{red}{$N^1$ en devise 1} contre \textcolor{blue}{$N^2$ en devise 2}.
\item La contrepartie \textbf{acheteuse peut à son gré} recevoir un nominal \textcolor{blue}{$N^2$ en devise 2} contre un nominal \textcolor{red}{$N^1$ en devise 1}.
\end{itemize}
\onslide<2>\begin{itemize}
\item La contrepartie \textbf{vendeuse s'engage} à recevoir un montant \textcolor{red}{$N^{EUR}$ en euro} contre \textcolor{blue}{$N^{USD}$ en dollar}.
\item La contrepartie \textbf{acheteuse peut à son gré} recevoir un nominal \textcolor{blue}{$N^{USD}$ en dollar} contre un nominal \textcolor{red}{$N^{EUR}$ en euro}.
\end{itemize}
\end{overprint}
\end{frame}

\end{document}
