\documentclass{beamer}
\usepackage[english,francais]{babel}
\usepackage[utf8]{inputenc}
\usepackage{multicol}
\usepackage{bm}

\usepackage{graphicx}
\graphicspath{{./Forex_Credit_Equity/}}

\newcommand{\FIG}[3]{\includegraphics<#1>[width=#2]{#3}}

\newcommand{\FIGSCALE}[3]{\includegraphics<#1>[resolution=72dpi]{#3}}

\usetheme{Warsaw}
\title[Produits dérivés actions,change et credit]{Produits dérivés de change}
\author{Richard Guillemot}
\institute{DIFIQ}
\date{10 Avril 2015}

\begin{document}

\begin{frame}
\titlepage
\end{frame}

\begin{frame}
\frametitle{Taux de change \textbf{"Spot"} (as of 2 Avril 2015)}
\Huge
\begin{center}
\textbf{\textcolor<3>{red}{EUR}/\textcolor<4>{blue}{USD}}=1.08785
\end{center}
\huge
\begin{center}
\onslide<2->{1 \textcolor<3>{red}{euro} vaut 1.08785 \textcolor<4>{blue}{dollar}.}
\end{center}
\large
\vspace{0.5cm}
\begin{itemize}
\item[]<3-> \textbf{\textcolor<3>{red}{EUR (euro)}} est la \textcolor<3>{red}{devise étrangère ou devise 1}.
\item[]<4-> \textbf{\textcolor<4>{blue}{USD (dollar)}} est la \textcolor<4>{blue}{devise domestique ou devise 2}.
\end{itemize}
\end{frame}

\begin{frame}
\frametitle{Livraison ou Settlement}
\begin{center}
\begin{figure}[h]
\FIG{1}{12cm}{figures/settlement-1.png}
\FIG{2}{12cm}{figures/settlement-2.png}
\FIG{3}{12cm}{figures/settlement-3.png}
\FIG{4}{12cm}{figures/settlement-4.png}
\end{figure}
\end{center}
\end{frame}

\begin{frame}
\frametitle{Taux de change \& Taux d'intérêts}
As of 2 Avril 2015:
\begin{center}
\begin{tabular}{|l|l|l|l|l|}
\hline
\textbf{FX} & 	&  & \textbf{IR} & \textbf{BS} \\
\hline
\hline
EURUSD & 1.0879 & EUR & 0.01\% & -0.27\%  \\
GBPUSD & 1.4829 & USD & 0.45\% & \\
USDCHF & 0.9581 & GBP & 0.61\% & -0.05\% \\
USDJPY & 119.76 & CHF & -0.88\% & -0.40\% \\
USDCNY & 6.1396 & JPY & 0.11\% & -0.39\% \\
	&	& CNY & 	& 3.83\% \\
\hline		
\end{tabular}
\end{center}
\end{frame}

\begin{frame}
\frametitle{Historique EURUSD}
\begin{center}
\begin{figure}[h]
\FIG{1}{10cm}{figures/EURUSDHisto.png}
\end{figure}
\end{center}
\end{frame}

\begin{frame}
\frametitle{Les différents types de taux d'intérêts}
\begin{figure}[h]
\centering \FIG{1-}{7cm}{figures/capi_actu.png}
\end{figure}
\begin{center}
\begin{tabular}{|c|c|c|}
\hline
&\textbf{C}apitalisation&\textbf{A}ctualisation \\
\hline
  Taux Linéaire & \visible<2->{$1+\delta R^L$} &  \visible<3->{$\frac{1}{1+\delta R^L}$} \\
  Taux Actuariel & \visible<4->{$(1+\frac{\delta}{N}R^A)^N$} &  \visible<5->{$\frac{1}{(1+\frac{\delta}{N}R^A)^N}$} \\
  Taux Continu & \visible<6->{$e^{\delta R^C}$} & \visible<7->{$e^{-\delta R^C}$} \\
\hline
\end{tabular}
\end{center}
\visible<8->{
\[
	R^C < R^A < R^L
\]}
\end{frame}


\begin{frame}
\frametitle{Taux de change \textbf{"Forward"}}
\large
\begin{overprint}
\onslide<1>Comment garantir un taux de change à une date future \textbf{T} ?\\ Et à quel taux \textbf{X}.
\onslide<2>\textbf{Prêt} en $t$ de \textcolor{red}{$\frac{1}{1+\delta R^{EUR}}$ euros}.\\Remboursé en $T$ avec les intérêts, c'est à dire \textcolor{red}{1 euro}.
\onslide<3>\textbf{Change} \textcolor{red}{$\frac{1}{1+\delta R^{EUR}}$ euros} contre \textcolor{blue}{$\frac{S}{1+\delta R^{EUR}}$dollars}.
\onslide<4>\textbf{Emprunt} en $t$ de \textcolor{blue}{$\frac{S}{1+\delta R^{EUR}}$} dollars\\ Remboursé en $T$ avec les intérêts, c'est à dire \textcolor{blue}{$S\frac{1+\delta R^{USD}}{1+\delta R^{EUR}}$ dollars}.
\onslide<5>\[X=S\frac{1+\delta R^{USD}}{1+\delta R^{EUR}}\]
\end{overprint}
\begin{figure}
\begin{overprint}
\FIG{1}{11cm}{figures/fxfwd-1.png}
\FIG{2}{11cm}{figures/fxfwd-2.png}
\FIG{3}{11cm}{figures/fxfwd-3.png}
\FIG{4}{11cm}{figures/fxfwd-4.png}
\FIG{5}{11cm}{figures/fxfwd-5.png}
\end{overprint}
\end{figure}
\end{frame}

\begin{frame}
\small
\frametitle{Taux de change \textbf{"Forward"} - Récapitulatif}
As of 2 Avril 2015:
\begin{center}
\begin{tabular}{|l|l|l|l|}
\hline
\textbf{Notation} & \textbf{Description} & \textbf{Formule} & \textbf{Valeur} \\
\hline
\hline
$\delta$ & \visible<2->{Maturité du forward} & \visible<2->{$T-(t+2D)$} & \visible<2->{1 an = 365 jours} \\
$R^{EUR}$ & \visible<3->{Taux zéro coupon euro.} &  & \visible<3->{0.01\%} \\
$R^{USD}$ & \visible<4->{Taux zéro coupon dollar.} &  & \visible<4->{0.45\%} \\
$S$ & \visible<5->{Taux de change spot.} &  & \visible<5->{1.08785} \\
$X$ & \visible<6->{Forward de change.} &  \visible<6->{$S\frac{1+\delta R^{USD}}{1+\delta R^{EUR}}$} & \visible<6->{??} \\
\hline
\end{tabular}
\end{center}
\[
\uncover<7->{X=} \uncover<8->{1.0878\times \frac{1+ 0.45\%}{1+0.01\%}\uncover<9->{=1.09264}} 
\]
\onslide<10->{Soit \textbf{47.86} points de base d'écart positif par rapport au taux spot.}
\end{frame}

\begin{frame}
\frametitle{Quizz}
Si on vend 100 millons euro dans 1 an au taux spot au lieu d'utiliser le taux foward précedemment calculé:\\
\vspace{0.5cm}
a) On perd 442 kEUR \uncover<2>{\textcolor{green}{\textbf{VRAI}}}\\
b) On gagne 44,2 kEUR \uncover<2>{\textcolor{red}{\textbf{FAUX}}}\\
c) On perd 4.42 millions d'euros. \uncover<2>{\textcolor{red}{\textbf{FAUX}}}\\
d) On gagne 442 kEUR. \uncover<2>{\textcolor{red}{\textbf{FAUX}}}\\
\vspace{0.5cm}
\uncover<2>{On emprunte à 0.45\% en dollars et on prête à 0.01\% en euros !!!}
\end{frame}

\begin{frame}
\frametitle{Le swap de devises ou Cross-Currency Swap}
\large
\begin{overprint}
\onslide<1>On considère l'échéancier d'un swap standard.
\onslide<2>On échange en $t$+2D ouvrés \textcolor{blue}{$N^{USD}$} avec sa contrevaleur \textcolor{red}{$N^{EUR}$}.\\ On fera l'échange inverse à la maturité du swap $T$.
\onslide<3>On reçoit une jambe variable euro en contrepartie d'une jambe variable dollar.
\onslide<4>En pratique il faut retirer la \textbf{\textcolor{red}{marge de basis m}} à la jambe EUR pour mettre le swap au pair (valeur nulle).
\onslide<5>Un swap de devises d'un seule période est un foward de change de nominal \textcolor{red}{$N^{EUR}(1+\delta (L^{EUR}-\textbf{m}))$}.
\end{overprint}
\begin{figure}
\begin{overprint}
\FIG{1}{11cm}{figures/xccyswap-1.png}
\FIG{2}{11cm}{figures/xccyswap-2.png}
\FIG{3}{11cm}{figures/xccyswap-3.png}
\FIG{4}{11cm}{figures/xccyswap-4.png}
\FIG{5}{11cm}{figures/xccyswap-singleflow.png}
\end{overprint}
\end{figure}
\end{frame}

\begin{frame}
\frametitle{Taux de change \textbf{Forward} et marge de basis.}
\Huge
\[
	X=S\frac{1+\delta R^{USD}}{1+\delta (R^{EUR}-\textbf{m})}
\]
\Large
As of 2 Avril 2015:
\huge
\begin{center}
$m=27$ bps
\end{center}
\end{frame}

\begin{frame}
\frametitle{Delta de change et position de change}
\begin{itemize}
\item[-] Le \textbf{delta de change} est la sensibilité ou la dérivée au taux de change de la valeur d'un portefeuille en devise domestique.\\
\vspace{0.5cm}
\[
\Delta_{FX}=\frac{\partial \prod^d}{\partial S}
\]
\item[-] La \textbf{position de change} correspond au nominaux $N^i$  équivalents au portefeuille dans chacune des devises. Elle indique la taille des opérations de change "Spot" nécessaires pour neutraliser le risque.

\end{itemize}
\end{frame}

\begin{frame}
\frametitle{Delta de change et position de change}
\small
Illustration avec les 2 devises euro et dollar:\\ 
\vspace{0.5cm}
\begin{tabular}{|l|l l|}
\hline
Taux de change&$S$&$= EUR/USD$\\
Valeur du portefeuille en dollar&$\Pi^{USD}$&$=N^{EUR} \times S + N^{USD}$\\
Delta de change&$\Delta_{EURUSD}$&$=N^{EUR}$\\
Position de change&&$(N^{EUR},N^{USD})$\\
\hline
\end{tabular}
\end{frame}

\begin{frame}
\frametitle{Exercice}
On reprend les données du premier exemple la marge de basis m est égale à 27 points de base:\\
\begin{itemize}
\item[-] \textbf{Opération 1}: Une banque francaise doit recevoir de son client \textcolor{blue}{109 millions de dollars} contre \textcolor{red}{100 millions d'euros} dans 1 an.
\item[-] \textbf{Opération 2}: Sa filliale américaine doit recevoir de son client \textcolor{red}{92 millions d'euros} contre \textcolor{blue}{100 millions de dollars} dans 1 an.
\end{itemize}
Pour chacune des 2 opérations et pour le portefeuille total de la banque:\\
\begin{enumerate}
\item Quel est le Profit \& Loss (PNL) pour la banque ?
\item Quels sont les Delta FX et la position de change ?
\item Quelles sont la sensibilités à un mouvement de 1 point de base des taux euro, dollar et de la marge de basis ?
\item Quelles opérations doit réaliser la banque pour neutraliser son risque de change ?
\end{enumerate}
\end{frame}

\begin{frame}{Exercice - Solution}

\begin{tabular}{|c|r|r|r|l|}
\hline
&\textbf{Cas 1}&\textbf{Cas 2}&\textbf{TOTAL}& \\
\hline
\hline
\textbf{PNL EUR} &-513&728&215&kEUR\\
\textbf{PNL USD} &-558&792&234&kUSD\\
\hline
\hline
\textbf{Delta FX} &-100.26&92.24&-8.02&Mios EUR\\
\textbf{Sensi taux EUR} &10.03&-9.22&0.80&kEUR/bp\\
\textbf{Sensi taux USD} &-10.85&9.95&-0.90&kUSD/bp\\
\textbf{Sensi basis} &-10.03&9.22&-0.80&kEUR/bp\\
\hline
\hline
\textbf{NEUR} &-100.261&92.240&-8.021&Mios EUR/bp\\
\textbf{NUSD} &108.512&-99.552&8.960&Mios USD/bp\\
\hline
\end{tabular}
\\
\vspace{0.5cm}
Pour se couvrir, \uncover<2>{il faut vendre 8.960 millions de dollars contre 8.021 millions d'euros.}
\end{frame}

\begin{frame}{Option de change}
Une \textbf{option de change} est un contrat asymétrique par lequel à une date future T:\\
\vspace{0.5cm}
\begin{overprint}
\onslide<1>\begin{itemize}
\item La contrepartie \textbf{vendeuse s'engage} à recevoir un montant \textcolor{red}{$N^1$ en devise 1} contre \textcolor{blue}{$N^2$ en devise 2}.
\item La contrepartie \textbf{acheteuse peut à son gré} recevoir un nominal \textcolor{blue}{$N^2$ en devise 2} contre un nominal \textcolor{red}{$N^1$ en devise 1}.
\end{itemize}
\onslide<2>\begin{itemize}
\item La contrepartie \textbf{vendeuse s'engage} à recevoir un montant \textcolor{red}{$N^{EUR}$ en euro} contre \textcolor{blue}{$N^{USD}$ en dollar}.
\item La contrepartie \textbf{acheteuse peut à son gré} recevoir un nominal \textcolor{blue}{$N^{USD}$ en dollar} contre un nominal \textcolor{red}{$N^{EUR}$ en euro}.
\end{itemize}
\end{overprint}
\end{frame}

\begin{frame}{Option de change - Payoff}
Quel est le payoff d'une option de change ?
\vspace{0.5cm}
\small
\begin{tabular}{|c|c|c|}
\hline
&$\mathbf{S^{EUR/USD}}$&$\mathbf{S^{USD/EUR}}$\\
\hline
\textcolor{red}{\textbf{EUR}}&\visible<3->{\textcolor{red}{$\frac{(N^{EUR} \times S^{EUR/USD}-N^{USD})_+}{S^{EUR/USD}}$}}&\visible<4->{\textcolor{red}{$(N^{EUR}-N^{USD} \times S^{USD/EUR})_+$}} \\
\textcolor{blue}{\textbf{USD}}& \visible<2->{\textcolor{blue}{$(N^{EUR} \times S^{EUR/USD}-N^{USD})_+$}} & \visible<5->{\textcolor{blue}{$\frac{(N^{EUR}-N^{USD} \times S^{USD/EUR})_+}{S^{USD/EUR}}$}}\\
\hline
\end{tabular}
\begin{center}
\textcolor{red}{100 Mios d'euros} call contre \textcolor{blue}{109 Mios de dollars} put.\\
\end{center}
\begin{figure}
\begin{center}
%\begin{overprint}
\FIG{1}{9cm}{figures/fxopt-payoff-0.png}
\FIG{2}{9cm}{figures/fxopt-payoff-1.png}
\FIG{3}{9cm}{figures/fxopt-payoff-2.png}
\FIG{4}{9cm}{figures/fxopt-payoff-3.png}
\FIG{5}{9cm}{figures/fxopt-payoff-4.png}
%\end{overprint}
\end{center}
\end{figure}
\end{frame}

\begin{frame}{Option de change - Black \& Scholes }
En contrepartie le vendeur reçoit de la part de l'acheteur une prime (\textbf{p})  que l'on peut calculer à l'aide de la formule de Black \& Scholes:\\
\vspace{0.5cm}
\begin{overprint}
\onslide<1>\begin{center}
$e^{-r^1 \times T} \times N^1 \times S \times \mathcal{N}(d_1)-e^{-r^2 \times T} \times N^2 \times \mathcal{N}(d_2)$
\end{center}
\onslide<2>\begin{center}
$e^{-r^{EUR} \times T} \times N^{EUR} \times S^{EUR/USD} \times \mathcal{N}(d_1)-e^{-r^{USD} \times T} \times N^{USD} \times \mathcal{N}(d_2)$
\end{center}
\end{overprint}
avec:\\
\begin{overprint}
\onslide<1>\[
\begin{split}
&\mathcal{N} : \text{fonction de répartition de la loi normale centrée réduite}\\
&d_1=\frac{\ln\left( \frac{N^1}{N^2} S \right)+(r^1-r^2) \times T+\frac{1}{2}\sigma^2 T}{\sigma\sqrt{T}}\\
&d_2=d_1-\sigma\sqrt{T}
\end{split}
\]
\onslide<2>\[
\begin{split}
&\mathcal{N} : \text{fonction de répartition de la loi normale centrée réduite}\\
&d_1=\frac{\ln\left( \frac{N^{EUR}}{N^{USD}} S^{EUR/USD} \right)+(r^{EUR}-r^{USD}) \times T+\frac{1}{2}\sigma^2 T}{\sigma\sqrt{T}}\\
&d_2=d_1-\sigma\sqrt{T}
\end{split}
\]
\end{overprint}
\end{frame}

\begin{frame}{Option de change - Symétrie}
\begin{overprint}
\onslide<1>On peut exprimer la prime (\textbf{p}) de plusieurs manières:\\
\onslide<2>Comme un call sur EUR/USD:\\
\onslide<3>Comme un put sur USD/EUR:\\
\end{overprint}
\vspace{0.5cm}
\begin{overprint}
\onslide<1>\begin{center}
$e^{-r_{EUR} \times T} \times N^{EUR} \times S^{EUR/USD} \times \mathcal{N}(d_1)-e^{-r_{USD} \times T} \times N^{USD} \times \mathcal{N}(d_2)$
\end{center}
\onslide<2>\begin{center}
$e^{-r_{USD} \times T} \times N^{EUR} \times \big[ F^{EUR/USD} \times \mathcal{N}(d_1)-K \times \mathcal{N}(d_2) \big]$
\end{center}
\onslide<3>\begin{center}
$e^{-r_{EUR} \times T} \times N^{USD} \times \big[ \frac{1}{K} \times \mathcal{N}(-d_2)-F^{USD/EUR} \times \mathcal{N}(-d_1) \big]$
\end{center}
\end{overprint}
avec:\\
\begin{overprint}
\onslide<1>\begin{align*}
d_1&=\frac{\ln\left( \frac{N^{EUR}}{N^{USD}} S^{EUR/USD} \right)+(r_{EUR}-r_{USD}) \times T+\frac{1}{2}\sigma^2 T}{\sigma\sqrt{T}}\\
d_2&=d_1-\sigma\sqrt{T}
\end{align*}
\onslide<2>\begin{align*}
d_1&=\frac{\ln\left( \frac{F^{EUR/USD}}{K} \right) +\frac{1}{2}\sigma^2 T}{\sigma\sqrt{T}}\\
d_2&=d_1-\sigma\sqrt{T}\\
K&=\frac{N^{USD}}{N^{EUR}}\\
F&=S^{EUR/USD}e^{(r^{USD}-r^{EUR}) \times T}
\end{align*}
\onslide<3>\begin{align*}
d_1&=\frac{\ln\left( F^{USD/EUR} \times K \right) +\frac{1}{2}\sigma^2 T}{\sigma\sqrt{T}}\\
d_2&=d_1-\sigma\sqrt{T}\\
K&=\frac{N^{USD}}{N^{EUR}}\\
F&=S^{USD/EUR}e^{(r^{EUR}-r^{USD}) \times T}
\end{align*}
\end{overprint}
\end{frame}

\begin{frame}{Quantile de la loi normale - Sens de la volatilité}
Au bout d'un an un actif financer de volatilité $\sigma$ a plus d'\textbf{une} chance sur \textbf{deux} de s'être écartée de $\pm \sigma$ de sa valeur initiale.
\begin{center}
\begin{figure}[h]
\FIG{1}{11cm}{figures/loinormale_quantile.png}
\end{figure}
\end{center}
\end{frame}


\begin{frame}{Option de change - Jargon}
\Huge
\begin{overprint}
\onslide<1>\begin{center}EUR/USD=1.0878\fontsize{60}{70}\selectfont\textcolor{white}{0}\huge\end{center}
\onslide<2>\begin{center}EUR/USD=1.0\fontsize{60}{70}\selectfont8\huge78\end{center}
\onslide<3>\begin{center}EUR/USD=1.087\fontsize{60}{70}\selectfont8\huge\end{center}
\vspace{0.5cm}
\end{overprint}
\large
\begin{overprint}
\onslide<1>On considère 5 chiffres significatifs dans un taux de change.
\onslide<2>Le 3\up{ème} chiffre en partant de la gauche est appelé \textbf{"Big Figure"}.
\onslide<3>Le 5\up{ème} chiffre en partant de la gauche est appelé \textbf{"pips"}.
\end{overprint}
\end{frame}


\begin{frame}{Option de change - Cotation de la prime - Exercice}
On considère les mêmes données de marché que précédemment avec une volatilité $\sigma=12\%$ et on cote la prime d'une option de change de maturité 1 an qui reçoit \textcolor{red}{100 millions d'euros} contre \textcolor{blue}{109 millions de dollars}.\\
\vspace{0.5cm}
Les 6 modes de cotations:\\
\vspace{0.5cm}

\begin{tabular}{|l|c|l|}
\hline
Prix en dollars&\visible<2->{$p$}&\visible<2->{5.488 Mios USD}\\
Prix en euros&\visible<3->{$\frac{p}{S}$}&\visible<3->{5.045 Mios EUR}\\
Prix en \% de nominal dollar&\visible<4->{$\frac{p}{N \times K}$}& \visible<4->{5.0353\%}\\
Prix en \% de nominal euro&\visible<5->{$\frac{p}{N \times S}$}& \visible<5->{5.0453\%}\\
Prix en dollars pips per EUR&\visible<6->{$\frac{p\times 1e^4}{N}$}& \visible<6->{548.85 USD pips}\\
Prix en euros pips per USD&\visible<7->{$\frac{p\times 1e^4}{S \times N \times K}$}& \visible<7->{462.87 EUR pips}\\
\hline
\end{tabular}
\end{frame}

\begin{frame}{Option de change - Delta de change}
Le Delta de change $\delta$ est le pourcentage du nominal en devise 1 qu'il faut vendre pour couvrir la position de change.
\[
\delta=\frac{\partial p}{\partial S}=e^{-r^{EUR} \times T} \times \mathcal{N}(d_1)
\]
On peut exprimer de façon équivalente le delta de change en pourcentage du nominal $\delta^{reverse}$ en devise 2:
\[
\delta^{reverse}=-\frac{\delta \times S}{K}
\]
\\
\visible<2>{\textcolor{red}{\textbf{Attention ces formules supposent que la prime est payée en dollars!!!}}}
\end{frame}

\begin{frame}{Option de change - Delta de change}
Dans le cas où la prime est payée en euros il faut ajuster le delta du paiement de la prime.\\
\vspace{0.5cm}
\begin{center}
\begin{tabular}{|c|c|c|c|}
\hline
delta ccy&premium ccy&Formule&Delta\\
\hline
\% EUR & EUR & \visible<2->{$\delta-p^{EUR}$} &\visible<4->{49.19\%}\\
\% EUR & USD & $\delta$ &\visible<3->{54.23\%}\\
\% USD & EUR & \visible<2->{$-\frac{(\delta-p^{EUR})S}{K}$} &\visible<4->{-49.09\%}\\
\% USD & USD & $-\frac{\delta}{K}$&\visible<3->{-54.12\%}\\
\hline
\end{tabular}
\end{center}
\visible<3->{La prime est égale à 5.0452\% du nominal EUR.}
\end{frame}

\begin{frame}{Option de change - Delta de change}
Comment évolue le delta de change en fonction du strike ?
\begin{figure}
\begin{center}
%\begin{overprint}
\FIG{1}{11cm}{figures/fxopt-delta-0.png}
\FIG{2}{11cm}{figures/fxopt-delta-1.png}
\FIG{3}{11cm}{figures/fxopt-delta-2.png}
\FIG{4}{11cm}{figures/fxopt-delta-3.png}
\FIG{5}{11cm}{figures/fxopt-delta-4.png}
%\end{overprint}
\end{center}
\end{figure}
\end{frame}

\begin{frame}
\small
\frametitle{\textbf{USDJPY} - Exercice}
Refaire tous les calculs précédent avec ces nouvelles données.\\
\vspace{0.5cm}
As of 2 Avril 2015:
\begin{center}
\begin{tabular}{|l|l|l|l|}
\hline
\textbf{Notation} & \textbf{Description} & \textbf{Formule} & \textbf{Valeur} \\
\hline
\hline
$\delta$ & Maturité du forward & $T-(t+2D)$ & 1 an = 365 jours \\
$R^{EUR}$ & Taux zéro coupon dollar. &  & 0.11\% \\
$R^{USD}$ & Taux zéro coupon yen. &  & 0.45\% \\
$S$ & Taux de change spot. &  & 119.76 \\
$m$ & Marge de basis. &  & 0.39\% \\
\hline
\end{tabular}
\end{center}
\end{frame}

\begin{frame}
\frametitle{Besoin d'un client américain.}
Un client américain doit payer son fournisseur français dans 1 an \textbf{\textcolor{red}{100 millions d'euros}}.\\ 
\uncover<2->{Pour des raisons "stratégiques" il ne souhaite pas couvrir cette position de change à terme.\\}
\vspace{0.5cm}
\uncover<3->{Cependant il souhaite tout de même se protéger contre des mouvements trop importants du taux de change.\\
Ainsi:\\
\begin{itemize}
\item Il veut payer au \textbf{maximum} \textbf{\textcolor{blue}{119 millions de dollars}}.
\item A l'inverse il veut payer au \textbf{minimum} \textbf{\textcolor{blue}{99 millions de dollars}}.
\end{itemize}}
\vspace{0.5cm}
\uncover<4->{\textbf{Comment statisfaire le besoin de notre client ?}}

\vspace{0.5cm}
\end{frame}



\end{document}
