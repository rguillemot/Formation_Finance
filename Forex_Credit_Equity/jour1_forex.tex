\documentclass{beamer}
\usepackage[english,francais]{babel}
\usepackage[utf8]{inputenc}
\usepackage{multicol}

\usepackage{graphicx}
\graphicspath{{./Forex_Credit_Equity/}}

\newcommand{\FIG}[3]{\includegraphics<#1>[width=#2]{#3}}

\usetheme{Warsaw}
\title[Produits dérivés actions,change et credit]{Produits dérivés de change}
\author{Richard Guillemot}
\institute{DIFIQ}
\date{11 Avril 2014}

\begin{document}

\begin{frame}
\titlepage
\end{frame}

\begin{frame}
\frametitle{Taux de change \textbf{"Spot"}}
\Huge
\begin{center}
\textbf{\textcolor<3>{red}{EUR}/\textcolor<4>{blue}{USD}}=1.3885
\end{center}
\huge
\begin{center}
\onslide<2->{1 \textcolor<3>{red}{euro} vaut 1.3885 \textcolor<4>{blue}{dollar}.}
\end{center}
\large
\vspace{0.5cm}
\begin{itemize}
\item[]<3-> \textbf{\textcolor<3>{red}{EUR (euro)}} est la \textcolor<3>{red}{devise étrangère ou devise 1}.
\item[]<4-> \textbf{\textcolor<4>{blue}{USD (dollar)}} est la \textcolor<4>{blue}{devise domestique ou devise 2}.
\end{itemize}
\end{frame}

\begin{frame}
\frametitle{Livraison ou Settlement}
\begin{center}
\begin{figure}[h]
\FIG{1}{12cm}{figures/settlement-1.png}
\FIG{2}{12cm}{figures/settlement-2.png}
\FIG{3}{12cm}{figures/settlement-3.png}
\FIG{4}{12cm}{figures/settlement-4.png}
\end{figure}
\end{center}
\end{frame}

\begin{frame}
\frametitle{Taux de change \textbf{"Forward"}}
\large
\begin{overprint}
\onslide<1>Comment garantir un taux de change à une date future \textbf{T} ?\\ Et à quel taux \textbf{X}.
\onslide<2>\textbf{Prêt} en $t$ de \textcolor{red}{$\frac{1}{1+\delta R^{EUR}}$ euros}.\\Remboursé en $T$ avec les intérêts, c'est à dire \textcolor{red}{1 euros}.
\onslide<3>\textbf{Change} \textcolor{red}{$\frac{1}{1+\delta R^{EUR}}$ euros} contre \textcolor{blue}{$\frac{S}{1+\delta r^{EUR}}$dollars}.
\onslide<4>\textbf{Emprunt} en $t$ de \textcolor{blue}{$\frac{S}{1+\delta R^{EUR}}$} dollars\\ Remboursé en $T$ avec les intérêts, c'est à dire \textcolor{blue}{$S\frac{1+\delta R^{USD}}{1+\delta R^{EUR}}$ dollars}.
\onslide<5>\[X=S\frac{1+\delta R^{USD}}{1+\delta R^{EUR}}\]
\end{overprint}
\begin{figure}
\begin{overprint}
\FIG{1}{11cm}{figures/fxfwd-1.png}
\FIG{2}{11cm}{figures/fxfwd-2.png}
\FIG{3}{11cm}{figures/fxfwd-3.png}
\FIG{4}{11cm}{figures/fxfwd-4.png}
\FIG{5}{11cm}{figures/fxfwd-5.png}
\end{overprint}
\end{figure}
\end{frame}

\begin{frame}
\small
\frametitle{Taux de change \textbf{"Forward"} - Récapitulatif}
\begin{center}
\begin{tabular}{|l|l|l|l|}
\hline
\textbf{Notation} & \textbf{Description} & \textbf{Formule} & \textbf{Valeur} \\
\hline
\hline
\onslide<2->
$\delta$ & Maturité du forward & $T-(t+2D)$ & 1 an = 365 jours \\
\onslide<3->
$R^{EUR}$ & Taux zéro coupon euro. &  & 0.5\% \\
\onslide<4->
$R^{USD}$ & Taux zéro coupon euro. &  & 0.3\% \\
\onslide<5->
$S$ & Taux de change spot. &  & 1.3889 \\
\onslide<6->
$X$ & Forward de change. &  $S\frac{1+\delta R^{USD}}{1+\delta R^{EUR}}$ & ?? \\
\hline
\end{tabular}
\end{center}
\[
\uncover<7->{X=} \uncover<8->{1.3889\times \frac{1+\frac{365}{360} \times 0.3\%}{1+\frac{365}{360} \times 0.5\%}\uncover<9->{=1.3861}} 
\]
\onslide<10->{Soit \textbf{27.6} points de base d'écart négatif par rapport au taux spot.}
\end{frame}

\begin{frame}
\frametitle{Quizz}
Si on vend 100 Mios euro dans 1 an d'euros au taux spot au lieu d'utiliser le taux foward précedemment calculé:\\
\vspace{0.5cm}
a) On gagne 276 kEUR \uncover<2>{\textcolor{green}{\textbf{VRAI}}}\\
b) On perd 27 kEUR \uncover<2>{\textcolor{red}{\textbf{FAUX}}}\\
c) On gagne 2.76 millions d'euros. \uncover<2>{\textcolor{red}{\textbf{FAUX}}}\\
d) On perd 276 kEUR. \uncover<2>{\textcolor{red}{\textbf{FAUX}}}\\
\vspace{0.5cm}
\uncover<2>{On emprunte à 0.3\% en dollars et on prête à 0.5\% en euros !!!}
\end{frame}

\begin{frame}
\frametitle{Le swap de devises ou Cross-Currency Swap}
\large
\begin{overprint}
\onslide<1>On considère l'échéancier d'un swap standard.
\onslide<2>On échange en $t$+2D ouvrés \textcolor{blue}{$N^{USD}$} avec sa contrevaleur \textcolor{red}{$N^{EUR}$}.\\ On fera l'échange inverse à la maturité du swap $T$.
\onslide<3>On reçoit une jambe variable euro en contrepartie d'une jambe variable dollar.
\onslide<4>En pratique il faut retirer la \textbf{\textcolor{red}{marge de basis m}} à la jambe EUR pour mettre le swap au pair (valeur nulle).
\onslide<5>Un swap de devises d'un seule période est un foward de change de nominal \textcolor{red}{$N^{EUR}(1+\delta (L^{EUR}-\textbf{m}))$}.
\end{overprint}
\begin{figure}
\begin{overprint}
\FIG{1}{11cm}{figures/xccyswap-1.png}
\FIG{2}{11cm}{figures/xccyswap-2.png}
\FIG{3}{11cm}{figures/xccyswap-3.png}
\FIG{4}{11cm}{figures/xccyswap-4.png}
\FIG{5}{11cm}{figures/xccyswap-singleflow.png}
\end{overprint}
\end{figure}
\end{frame}

\begin{frame}
\frametitle{Taux de change \textbf{Forward} et marge de basis.}
\Huge
\[
	X=S\frac{1+\delta R^{USD}}{1+\delta (R^{EUR}-\textbf{m})}
\]
\end{frame}

\begin{frame}
\frametitle{Delta de change et position de change}
\begin{itemize}
\item[-] Le \textbf{delta de change} est la sensibilité ou la dérivé au taux de change de la valeur d'un portefeuille en devise domestique.\\
\vspace{0.5cm}
\[
\Delta_{FX}=\frac{\partial \prod^d}{\partial S}
\]
\item[-] La \textbf{position de change} correspond au nominaux équivalents $N^i$ au portefeuille dans chacune des devises. Elle indique la taille des opérations de change "Spot" nécessaires pour neutraliser le risque.

\end{itemize}
\end{frame}

\begin{frame}
\frametitle{Delta de change et position de change}
\small
Illustration avec les 2 devises euro et dollar:\\ 
\vspace{0.5cm}
\begin{tabular}{|l|l l|}
\hline
Taux de change&$S$&$= EUR/USD$\\
Valeur du portefeuille en dollar&$\Pi^{USD}$&$=N^{EUR} \times S + N^{USD}$\\
Delta de change&$\Delta_{EURUSD}$&$=N^{EUR}$\\
Position de change&&$(N^{EUR},N^{USD})$\\
\hline
\end{tabular}
\end{frame}

\begin{frame}
\frametitle{Problème}
On reprend les données du premier exemple la marge de basis m égale à 5 points de base:\\
\begin{itemize}
\item[-] \textbf{Opération 1}: Une banque francaise doit recevoir de son client \textcolor{blue}{138.70 millions de dollars} contre \textcolor{red}{100 millions d'euros} dans 1 an.
\item[-] \textbf{Opération 2}: Sa filliale américaine doit recevoir de son client \textcolor{red}{72.11 millions d'euros} contre \textcolor{blue}{100 millions de dollars} dans 1 an.
\end{itemize}
Pour chacune des 2 opérations et le portefeuille total de la banque:\\
\begin{enumerate}
\item Quel est le Profit \& Loss (PNL) pour la banque ?
\item Quels sont de Delta FX et la position de change ?
\item Quelle est la sensibilité à un mouvement de 1 point de base des taux euros, dollar et de la marge de basis ?
\item Quelles opérations doit réaliser la banque pour neutraliser son risque de change ?
\end{enumerate}
\end{frame}

\begin{frame}{Problème - Solution}
\begin{center}
\begin{tabular}{|c|r|r|r|l|}
\hline
&\textbf{Cas 1}&\textbf{Cas 2}&\textbf{TOTAL}& \\
\hline
\hline
\textbf{PNL EUR} &12&3&15&kEUR\\
\textbf{PNL USD} &17&4&21&kUSD\\
\hline
\hline
\textbf{Delta FX} &-99.55&71.79&-27.77&Mios EUR\\
\textbf{Sensi taux} &9.91&-7.15&2.76&kEUR/bp\\
\textbf{Sensi taux} &-13.79&9.94&-3.85&kUSD/bp\\
\textbf{Sensi basis} &-9.91&7.15&-2.76&kEUR/bp\\
\hline
\hline
\textbf{NEUR} &-99.552&71.787&-27.765&Mios EUR/bp\\
\textbf{NUSD} &138.285&-99.701&38.584&Mios USD/bp\\
\hline
\end{tabular}
\end{center}
Il faut vendre 38.584 millions de dollar contre 27.780 millions d'euros.
\end{frame}

\begin{frame}{Option de change}
Une \textbf{option de change} est un contrat asymétrique par lequel à une date future T:\\
\vspace{0.5cm}
\begin{overprint}
\onslide<1>\begin{itemize}
\item La contrepartie \textbf{vendeuse s'engage} à recevoir un montant \textcolor{red}{$N^1$ en devise 1} contre \textcolor{blue}{$N^2$ en devise 2}.
\item La contrepartie \textbf{acheteuse peut à son gré} recevoir un nominal \textcolor{blue}{$N^2$ en devise 2} contre un nominal \textcolor{red}{$N^1$ en devise 1}.
\end{itemize}
\onslide<2>\begin{itemize}
\item La contrepartie \textbf{vendeuse s'engage} à recevoir un montant \textcolor{red}{$N^{EUR}$ en euro} contre \textcolor{blue}{$N^{USD}$ en dollar}.
\item La contrepartie \textbf{acheteuse peut à son gré} recevoir un nominal \textcolor{blue}{$N^{USD}$ en dollar} contre un nominal \textcolor{red}{$N^{EUR}$ en euros}.
\end{itemize}
\end{overprint}
\end{frame}

\begin{frame}{Option de change - Payoff}
Quel est le payoff d'une option de change ?
\vspace{0.5cm}
\small
\begin{overprint}
\onslide<1>\begin{tabular}{|c|c|c|}
\hline
&$\mathbf{S^{EUR/USD}}$&$\mathbf{S^{USD/EUR}}$\\
\hline
\textcolor{red}{\textbf{EUR}}&\textcolor{white}{$\frac{(N^{EUR} \times S^{EUR/USD}-N^{USD})_+}{S^{EUR/USD}}$}&\textcolor{white}{$(N^{EUR}-N^{USD} \times S^{USD/EUR})_+$} \\
\textcolor{blue}{\textbf{USD}}& \textcolor{white}{$(N^{EUR} \times S^{EUR/USD}-N^{USD})_+$} & \textcolor{white}{$\frac{(N^{EUR}-N^{USD} \times S^{USD/EUR})_+}{S^{USD/EUR}}$} \\
\hline
\end{tabular}
\onslide<2>\begin{tabular}{|c|c|c|}
\hline
&$\mathbf{S^{EUR/USD}}$&$\mathbf{S^{USD/EUR}}$\\
\hline
\textcolor{red}{\textbf{EUR}}&\textcolor{white}{$\frac{(N^{EUR} \times S^{EUR/USD}-N^{USD})_+}{S^{EUR/USD}}$}&\textcolor{white}{$(N^{EUR}-N^{USD} \times S^{USD/EUR})_+$} \\
\textcolor{blue}{\textbf{USD}}& \textcolor{blue}{$(N^{EUR} \times S^{EUR/USD}-N^{USD})_+$} & \textcolor{white}{$\frac{(N^{EUR}-N^{USD} \times S^{USD/EUR})_+}{S^{USD/EUR}}$} \\
\hline
\end{tabular}
\onslide<3>\begin{tabular}{|c|c|c|}
\hline
&$\mathbf{S^{EUR/USD}}$&$\mathbf{S^{USD/EUR}}$\\
\hline
\textcolor{red}{\textbf{EUR}}&\textcolor{red}{$\frac{(N^{EUR} \times S^{EUR/USD}-N^{USD})_+}{S^{EUR/USD}}$}&\textcolor{white}{$(N^{EUR}-N^{USD} \times S^{USD/EUR})_+$} \\
\textcolor{blue}{\textbf{USD}}& \textcolor{blue}{$(N^{EUR} \times S^{EUR/USD}-N^{USD})_+$} & \textcolor{white}{$\frac{(N^{EUR}-N^{USD} \times S^{USD/EUR})_+}{S^{USD/EUR}}$} \\
\hline
\end{tabular}
\onslide<4>\begin{tabular}{|c|c|c|}
\hline
&$\mathbf{S^{EUR/USD}}$&$\mathbf{S^{USD/EUR}}$\\
\hline
\textcolor{red}{\textbf{EUR}}&\textcolor{red}{$\frac{(N^{EUR} \times S^{EUR/USD}-N^{USD})_+}{S^{EUR/USD}}$}&\textcolor{red}{$(N^{EUR}-N^{USD} \times S^{USD/EUR})_+$} \\
\textcolor{blue}{\textbf{USD}}& \textcolor{blue}{$(N^{EUR} \times S^{EUR/USD}-N^{USD})_+$} & \textcolor{white}{$\frac{(N^{EUR}-N^{USD} \times S^{USD/EUR})_+}{S^{USD/EUR}}$} \\
\hline
\end{tabular}
\onslide<5>\begin{tabular}{|c|c|c|}
\hline
&$\mathbf{S^{EUR/USD}}$&$\mathbf{S^{USD/EUR}}$\\
\hline
\textcolor{red}{\textbf{EUR}}&\textcolor{red}{$\frac{(N^{EUR} \times S^{EUR/USD}-N^{USD})_+}{S^{EUR/USD}}$}&\textcolor{red}{$(N^{EUR}-N^{USD} \times S^{USD/EUR})_+$} \\
\textcolor{blue}{\textbf{USD}}& \textcolor{blue}{$(N^{EUR} \times S^{EUR/USD}-N^{USD})_+$} & \textcolor{blue}{$\frac{(N^{EUR}-N^{USD} \times S^{USD/EUR})_+}{S^{USD/EUR}}$} \\
\hline
\end{tabular}
\end{overprint}
\begin{center}
\textcolor{red}{100 Mios d'euros} call contre \textcolor{blue}{139 Mios de dollars} put.\\
\end{center}
\begin{figure}
\begin{center}
%\begin{overprint}
\FIG{1}{9cm}{figures/fxopt-payoff-0.png}
\FIG{2}{9cm}{figures/fxopt-payoff-1.png}
\FIG{3}{9cm}{figures/fxopt-payoff-2.png}
\FIG{4}{9cm}{figures/fxopt-payoff-3.png}
\FIG{5}{9cm}{figures/fxopt-payoff-4.png}
%\end{overprint}
\end{center}
\end{figure}
\end{frame}

\begin{frame}{Option de change - Black \& Scholes }
En contrepartie le vendeur reçoit un prime de la part de l'acheteur que l'on peut calculer à l'aide de la formule de Black \& Scholes:\\
\vspace{0.5cm}
\begin{overprint}
\onslide<1>\begin{center}
$e^{-r^1 \times T} \times N^1 \times S \times \mathcal{N}(d_1)-e^{-r^2 \times T} \times N^2 \times \mathcal{N}(d_2)$
\end{center}
\onslide<2>\begin{center}
$e^{-r^{EUR} \times T} \times N^{EUR} \times S^{EUR/USD} \times \mathcal{N}(d_1)-e^{-r^{USD} \times T} \times N^{USD} \times \mathcal{N}(d_2)$
\end{center}
\end{overprint}
avec:\\
\begin{overprint}
\onslide<1>\[
\begin{split}
&\mathcal{N} : \text{fonction de répartition de la loi normale centrée réduite}\\
&d_1=\frac{\ln\left( \frac{N^1}{N^2} S \right)+(r^1-r^2) \times T+\frac{1}{2}\sigma^2 T}{\sigma\sqrt{T}}\\
&d_2=d_1-\sigma\sqrt{T}
\end{split}
\]
\onslide<2>\[
\begin{split}
&\mathcal{N} : \text{fonction de répartition de la loi normale centrée réduite}\\
&d_1=\frac{\ln\left( \frac{N^{EUR}}{N^{USD}} S^{EUR/USD} \right)+(r^{EUR}-r^{USD}) \times T+\frac{1}{2}\sigma^2 T}{\sigma\sqrt{T}}\\
&d_2=d_1-\sigma\sqrt{T}
\end{split}
\]
\end{overprint}
\end{frame}

\begin{frame}{Option de change - Symétrie}
\begin{overprint}
\onslide<1>On peut exprimer la prime de l'option de plusieurs manières:\\
\onslide<2>Comme un call sur EUR/USD:\\
\onslide<3>Comme un put sur USD/EUR:\\
\end{overprint}
\vspace{0.5cm}
\begin{overprint}
\onslide<1>\begin{center}
$e^{-r_{EUR} \times T} \times N^{EUR} \times S^{EUR/USD} \times \mathcal{N}(d_1)-e^{-r_{USD} \times T} \times N^{USD} \times \mathcal{N}(d_2)$
\end{center}
\onslide<2>\begin{center}
$e^{-r_{USD} \times T} \times N^{EUR} \times \big[ F^{EUR/USD} \times \mathcal{N}(d_1)-K \times \mathcal{N}(d_2) \big]$
\end{center}
\onslide<3>\begin{center}
$e^{-r_{EUR} \times T} \times N^{USD} \times \big[ \frac{1}{K} \times \mathcal{N}(-d_2)-F^{USD/EUR} \times \mathcal{N}(-d_1) \big]$
\end{center}
\end{overprint}
avec:\\
\begin{overprint}
\onslide<1>\begin{align*}
d_1&=\frac{\ln\left( \frac{N^{EUR}}{N^{USD}} S^{EUR/USD} \right)+(r_{EUR}-r_{USD}) \times T+\frac{1}{2}\sigma^2 T}{\sigma\sqrt{T}}\\
d_2&=d_1-\sigma\sqrt{T}
\end{align*}
\onslide<2>\begin{align*}
d_1&=\frac{\ln\left( \frac{F^{EUR/USD}}{K} \right) +\frac{1}{2}\sigma^2 T}{\sigma\sqrt{T}}\\
d_2&=d_1-\sigma\sqrt{T}\\
K&=\frac{N^{USD}}{N^{EUR}}\\
F&=S^{EUR/USD}e^{(r^{EUR}-r^{USD}) \times T}
\end{align*}
\onslide<3>\begin{align*}
d_1&=\frac{\ln\left( F^{USD/EUR} \times K \right) +\frac{1}{2}\sigma^2 T}{\sigma\sqrt{T}}\\
d_2&=d_1-\sigma\sqrt{T}\\
K&=\frac{N^{USD}}{N^{EUR}}\\
F&=S^{USD/EUR}e^{(r^{USD}-r^{EUR}) \times T}
\end{align*}
\end{overprint}
\end{frame}

\begin{frame}{Option de change - Jargon}
\Huge
\begin{overprint}
\onslide<1>\begin{center}EUR/USD=1.3885\fontsize{60}{70}\selectfont\textcolor{white}{0}\huge\end{center}
\onslide<2>\begin{center}EUR/USD=1.3\fontsize{60}{70}\selectfont8\huge85\end{center}
\onslide<3>\begin{center}EUR/USD=1.388\fontsize{60}{70}\selectfont5\huge\end{center}
\vspace{0.5cm}
\end{overprint}
\large
\begin{overprint}
\onslide<1>On considère 5 chiffres significatifs dans un taux de change.
\onslide<2>Le 3\up{ème} chiffre en partant de la gauche est appelé \textbf{"Big Figure"}.
\onslide<3>Le 5\up{ème} chiffre en partant de la gauche est appelé \textbf{"pips"}.
\end{overprint}
\end{frame}

\begin{frame}{Option de change - Quotation de la prime}
On considère les même données de marché que précédemment et on quote la prime d'une option change de maturité 1 an qui reçoit \textcolor{red}{100 millions d'euros} contre \textcolor{blue}{139 millions de dollars}.\\
\begin{tabular}{|c|c|c|}
\hline
Cash EUR&&\\
Cash USD&&\\
\% de nominal EUR&&\\
\% de mominal USD&&\\
USD pips par EUR
% Nominal &&\\
\hline
\end{tabular}
\end{frame}

\end{document}
