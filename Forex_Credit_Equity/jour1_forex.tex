\documentclass{beamer}
\usepackage[english,francais]{babel}
\usepackage[utf8]{inputenc}
\usepackage{multicol}
\usepackage{bm}

\usepackage{graphicx}
\graphicspath{{./Forex_Credit_Equity/}}

\newcommand{\FIG}[3]{\includegraphics<#1>[width=#2]{#3}}
\newcommand{\FIGSCALE}[3]{\includegraphics<#1>[resolution=72dpi]{#3}}

\usetheme{Warsaw}
\title[Produits dérivés actions,change et credit]{Produits dérivés de change}
\author{Richard Guillemot}
\institute{DIFIQ}
\date{11 Avril 2014}

\begin{document}

\begin{frame}
\titlepage
\end{frame}

\begin{frame}
\frametitle{Taux de change \textbf{"Spot"}}
\Huge
\begin{center}
\textbf{\textcolor<3>{red}{EUR}/\textcolor<4>{blue}{USD}}=1.3889
\end{center}
\huge
\begin{center}
\onslide<2->{1 \textcolor<3>{red}{euro} vaut 1.3889 \textcolor<4>{blue}{dollar}.}
\end{center}
\large
\vspace{0.5cm}
\begin{itemize}
\item[]<3-> \textbf{\textcolor<3>{red}{EUR (euro)}} est la \textcolor<3>{red}{devise étrangère ou devise 1}.
\item[]<4-> \textbf{\textcolor<4>{blue}{USD (dollar)}} est la \textcolor<4>{blue}{devise domestique ou devise 2}.
\end{itemize}
\end{frame}

\begin{frame}
\frametitle{Livraison ou Settlement}
\begin{center}
\begin{figure}[h]
\FIG{1}{12cm}{figures/settlement-1.png}
\FIG{2}{12cm}{figures/settlement-2.png}
\FIG{3}{12cm}{figures/settlement-3.png}
\FIG{4}{12cm}{figures/settlement-4.png}
\end{figure}
\end{center}
\end{frame}

\begin{frame}
\frametitle{Taux de change \textbf{"Forward"}}
\large
\begin{overprint}
\onslide<1>Comment garantir un taux de change à une date future \textbf{T} ?\\ Et à quel taux \textbf{X}.
\onslide<2>\textbf{Prêt} en $t$ de \textcolor{red}{$\frac{1}{1+\delta R^{EUR}}$ euros}.\\Remboursé en $T$ avec les intérêts, c'est à dire \textcolor{red}{1 euros}.
\onslide<3>\textbf{Change} \textcolor{red}{$\frac{1}{1+\delta R^{EUR}}$ euros} contre \textcolor{blue}{$\frac{S}{1+\delta r^{EUR}}$dollars}.
\onslide<4>\textbf{Emprunt} en $t$ de \textcolor{blue}{$\frac{S}{1+\delta R^{EUR}}$} dollars\\ Remboursé en $T$ avec les intérêts, c'est à dire \textcolor{blue}{$S\frac{1+\delta R^{USD}}{1+\delta R^{EUR}}$ dollars}.
\onslide<5>\[X=S\frac{1+\delta R^{USD}}{1+\delta R^{EUR}}\]
\end{overprint}
\begin{figure}
\begin{overprint}
\FIG{1}{11cm}{figures/fxfwd-1.png}
\FIG{2}{11cm}{figures/fxfwd-2.png}
\FIG{3}{11cm}{figures/fxfwd-3.png}
\FIG{4}{11cm}{figures/fxfwd-4.png}
\FIG{5}{11cm}{figures/fxfwd-5.png}
\end{overprint}
\end{figure}
\end{frame}

\begin{frame}
\small
\frametitle{Taux de change \textbf{"Forward"} - Récapitulatif}
\begin{center}
\begin{tabular}{|l|l|l|l|}
\hline
\textbf{Notation} & \textbf{Description} & \textbf{Formule} & \textbf{Valeur} \\
\hline
\hline
$\delta$ & \visible<2->{Maturité du forward} & \visible<2->{$T-(t+2D)$} & \visible<2->{1 an = 365 jours} \\
$R^{EUR}$ & \visible<3->{Taux zéro coupon euro.} &  & \visible<3->{0.5\%} \\
$R^{USD}$ & \visible<4->{Taux zéro coupon dollar.} &  & \visible<4->{0.3\%} \\
$S$ & \visible<5->{Taux de change spot.} &  & \visible<5->{1.3889} \\
$X$ & \visible<6->{Forward de change.} &  \visible<6->{$S\frac{1+\delta R^{USD}}{1+\delta R^{EUR}}$} & \visible<6->{??} \\
\hline
\end{tabular}
\end{center}
\[
\uncover<7->{X=} \uncover<8->{1.3889\times \frac{1+\frac{365}{360} \times 0.3\%}{1+\frac{365}{360} \times 0.5\%}\uncover<9->{=1.3861}} 
\]
\onslide<10->{Soit \textbf{27.6} points de base d'écart négatif par rapport au taux spot.}
\end{frame}

\begin{frame}
\frametitle{Quizz}
Si on vend 100 Mios euro dans 1 an d'euros au taux spot au lieu d'utiliser le taux foward précedemment calculé:\\
\vspace{0.5cm}
a) On gagne 276 kEUR \uncover<2>{\textcolor{green}{\textbf{VRAI}}}\\
b) On perd 27 kEUR \uncover<2>{\textcolor{red}{\textbf{FAUX}}}\\
c) On gagne 2.76 millions d'euros. \uncover<2>{\textcolor{red}{\textbf{FAUX}}}\\
d) On perd 276 kEUR. \uncover<2>{\textcolor{red}{\textbf{FAUX}}}\\
\vspace{0.5cm}
\uncover<2>{On emprunte à 0.3\% en dollars et on prête à 0.5\% en euros !!!}
\end{frame}

\begin{frame}
\frametitle{Le swap de devises ou Cross-Currency Swap}
\large
\begin{overprint}
\onslide<1>On considère l'échéancier d'un swap standard.
\onslide<2>On échange en $t$+2D ouvrés \textcolor{blue}{$N^{USD}$} avec sa contrevaleur \textcolor{red}{$N^{EUR}$}.\\ On fera l'échange inverse à la maturité du swap $T$.
\onslide<3>On reçoit une jambe variable euro en contrepartie d'une jambe variable dollar.
\onslide<4>En pratique il faut retirer la \textbf{\textcolor{red}{marge de basis m}} à la jambe EUR pour mettre le swap au pair (valeur nulle).
\onslide<5>Un swap de devises d'un seule période est un foward de change de nominal \textcolor{red}{$N^{EUR}(1+\delta (L^{EUR}-\textbf{m}))$}.
\end{overprint}
\begin{figure}
\begin{overprint}
\FIG{1}{11cm}{figures/xccyswap-1.png}
\FIG{2}{11cm}{figures/xccyswap-2.png}
\FIG{3}{11cm}{figures/xccyswap-3.png}
\FIG{4}{11cm}{figures/xccyswap-4.png}
\FIG{5}{11cm}{figures/xccyswap-singleflow.png}
\end{overprint}
\end{figure}
\end{frame}

\begin{frame}
\frametitle{Taux de change \textbf{Forward} et marge de basis.}
\Huge
\[
	X=S\frac{1+\delta R^{USD}}{1+\delta (R^{EUR}-\textbf{m})}
\]
\end{frame}

\begin{frame}
\frametitle{Delta de change et position de change}
\begin{itemize}
\item[-] Le \textbf{delta de change} est la sensibilité ou la dérivé au taux de change de la valeur d'un portefeuille en devise domestique.\\
\vspace{0.5cm}
\[
\Delta_{FX}=\frac{\partial \prod^d}{\partial S}
\]
\item[-] La \textbf{position de change} correspond au nominaux équivalents $N^i$ au portefeuille dans chacune des devises. Elle indique la taille des opérations de change "Spot" nécessaires pour neutraliser le risque.

\end{itemize}
\end{frame}

\begin{frame}
\frametitle{Delta de change et position de change}
\small
Illustration avec les 2 devises euro et dollar:\\ 
\vspace{0.5cm}
\begin{tabular}{|l|l l|}
\hline
Taux de change&$S$&$= EUR/USD$\\
Valeur du portefeuille en dollar&$\Pi^{USD}$&$=N^{EUR} \times S + N^{USD}$\\
Delta de change&$\Delta_{EURUSD}$&$=N^{EUR}$\\
Position de change&&$(N^{EUR},N^{USD})$\\
\hline
\end{tabular}
\end{frame}

\begin{frame}
\frametitle{Exercice}
On reprend les données du premier exemple la marge de basis m égale à 5 points de base:\\
\begin{itemize}
\item[-] \textbf{Opération 1}: Une banque francaise doit recevoir de son client \textcolor{blue}{138.70 millions de dollars} contre \textcolor{red}{100 millions d'euros} dans 1 an.
\item[-] \textbf{Opération 2}: Sa filliale américaine doit recevoir de son client \textcolor{red}{72.11 millions d'euros} contre \textcolor{blue}{100 millions de dollars} dans 1 an.
\end{itemize}
Pour chacune des 2 opérations et le portefeuille total de la banque:\\
\begin{enumerate}
\item Quel est le Profit \& Loss (PNL) pour la banque ?
\item Quels sont de Delta FX et la position de change ?
\item Quelle est la sensibilité à un mouvement de 1 point de base des taux euros, dollar et de la marge de basis ?
\item Quelles opérations doit réaliser la banque pour neutraliser son risque de change ?
\end{enumerate}
\end{frame}

\begin{frame}{Exercice - Solution}
\begin{center}
\begin{tabular}{|c|r|r|r|l|}
\hline
&\textbf{Cas 1}&\textbf{Cas 2}&\textbf{TOTAL}& \\
\hline
\hline
\textbf{PNL EUR} &12&3&15&kEUR\\
\textbf{PNL USD} &17&4&21&kUSD\\
\hline
\hline
\textbf{Delta FX} &-99.55&71.79&-27.77&Mios EUR\\
\textbf{Sensi taux} &9.91&-7.15&2.76&kEUR/bp\\
\textbf{Sensi taux} &-13.79&9.94&-3.85&kUSD/bp\\
\textbf{Sensi basis} &-9.91&7.15&-2.76&kEUR/bp\\
\hline
\hline
\textbf{NEUR} &-99.552&71.787&-27.765&Mios EUR/bp\\
\textbf{NUSD} &138.285&-99.701&38.584&Mios USD/bp\\
\hline
\end{tabular}
\end{center}
Il faut vendre 38.584 millions de dollar contre 27.780 millions d'euros.
\end{frame}

\begin{frame}{Option de change}
Une \textbf{option de change} est un contrat asymétrique par lequel à une date future T:\\
\vspace{0.5cm}
\begin{overprint}
\onslide<1>\begin{itemize}
\item La contrepartie \textbf{vendeuse s'engage} à recevoir un montant \textcolor{red}{$N^1$ en devise 1} contre \textcolor{blue}{$N^2$ en devise 2}.
\item La contrepartie \textbf{acheteuse peut à son gré} recevoir un nominal \textcolor{blue}{$N^2$ en devise 2} contre un nominal \textcolor{red}{$N^1$ en devise 1}.
\end{itemize}
\onslide<2>\begin{itemize}
\item La contrepartie \textbf{vendeuse s'engage} à recevoir un montant \textcolor{red}{$N^{EUR}$ en euro} contre \textcolor{blue}{$N^{USD}$ en dollar}.
\item La contrepartie \textbf{acheteuse peut à son gré} recevoir un nominal \textcolor{blue}{$N^{USD}$ en dollar} contre un nominal \textcolor{red}{$N^{EUR}$ en euros}.
\end{itemize}
\end{overprint}
\end{frame}

\begin{frame}{Option de change - Payoff}
Quel est le payoff d'une option de change ?
\vspace{0.5cm}
\small
\begin{tabular}{|c|c|c|}
\hline
&$\mathbf{S^{EUR/USD}}$&$\mathbf{S^{USD/EUR}}$\\
\hline
\textcolor{red}{\textbf{EUR}}&\visible<3->{\textcolor{red}{$\frac{(N^{EUR} \times S^{EUR/USD}-N^{USD})_+}{S^{EUR/USD}}$}}&\visible<4->{\textcolor{red}{$(N^{EUR}-N^{USD} \times S^{USD/EUR})_+$}} \\
\textcolor{blue}{\textbf{USD}}& \visible<2->{\textcolor{blue}{$(N^{EUR} \times S^{EUR/USD}-N^{USD})_+$}} & \visible<5->{\textcolor{blue}{$\frac{(N^{EUR}-N^{USD} \times S^{USD/EUR})_+}{S^{USD/EUR}}$}}\\
\hline
\end{tabular}
\begin{center}
\textcolor{red}{100 Mios d'euros} call contre \textcolor{blue}{139 Mios de dollars} put.\\
\end{center}
\begin{figure}
\begin{center}
%\begin{overprint}
\FIG{1}{9cm}{figures/fxopt-payoff-0.png}
\FIG{2}{9cm}{figures/fxopt-payoff-1.png}
\FIG{3}{9cm}{figures/fxopt-payoff-2.png}
\FIG{4}{9cm}{figures/fxopt-payoff-3.png}
\FIG{5}{9cm}{figures/fxopt-payoff-4.png}
%\end{overprint}
\end{center}
\end{figure}
\end{frame}

\begin{frame}{Option de change - Black \& Scholes }
En contrepartie le vendeur reçoit une prime (\textbf{p}) de la part de l'acheteur que l'on peut calculer à l'aide de la formule de Black \& Scholes:\\
\vspace{0.5cm}
\begin{overprint}
\onslide<1>\begin{center}
$e^{-r^1 \times T} \times N^1 \times S \times \mathcal{N}(d_1)-e^{-r^2 \times T} \times N^2 \times \mathcal{N}(d_2)$
\end{center}
\onslide<2>\begin{center}
$e^{-r^{EUR} \times T} \times N^{EUR} \times S^{EUR/USD} \times \mathcal{N}(d_1)-e^{-r^{USD} \times T} \times N^{USD} \times \mathcal{N}(d_2)$
\end{center}
\end{overprint}
avec:\\
\begin{overprint}
\onslide<1>\[
\begin{split}
&\mathcal{N} : \text{fonction de répartition de la loi normale centrée réduite}\\
&d_1=\frac{\ln\left( \frac{N^1}{N^2} S \right)+(r^1-r^2) \times T+\frac{1}{2}\sigma^2 T}{\sigma\sqrt{T}}\\
&d_2=d_1-\sigma\sqrt{T}
\end{split}
\]
\onslide<2>\[
\begin{split}
&\mathcal{N} : \text{fonction de répartition de la loi normale centrée réduite}\\
&d_1=\frac{\ln\left( \frac{N^{EUR}}{N^{USD}} S^{EUR/USD} \right)+(r^{EUR}-r^{USD}) \times T+\frac{1}{2}\sigma^2 T}{\sigma\sqrt{T}}\\
&d_2=d_1-\sigma\sqrt{T}
\end{split}
\]
\end{overprint}
\end{frame}

\begin{frame}{Option de change - Symétrie}
\begin{overprint}
\onslide<1>On peut exprimer la prime (\textbf{p}) de l'option de plusieurs manières:\\
\onslide<2>Comme un call sur EUR/USD:\\
\onslide<3>Comme un put sur USD/EUR:\\
\end{overprint}
\vspace{0.5cm}
\begin{overprint}
\onslide<1>\begin{center}
$e^{-r_{EUR} \times T} \times N^{EUR} \times S^{EUR/USD} \times \mathcal{N}(d_1)-e^{-r_{USD} \times T} \times N^{USD} \times \mathcal{N}(d_2)$
\end{center}
\onslide<2>\begin{center}
$e^{-r_{USD} \times T} \times N^{EUR} \times \big[ F^{EUR/USD} \times \mathcal{N}(d_1)-K \times \mathcal{N}(d_2) \big]$
\end{center}
\onslide<3>\begin{center}
$e^{-r_{EUR} \times T} \times N^{USD} \times \big[ \frac{1}{K} \times \mathcal{N}(-d_2)-F^{USD/EUR} \times \mathcal{N}(-d_1) \big]$
\end{center}
\end{overprint}
avec:\\
\begin{overprint}
\onslide<1>\begin{align*}
d_1&=\frac{\ln\left( \frac{N^{EUR}}{N^{USD}} S^{EUR/USD} \right)+(r_{EUR}-r_{USD}) \times T+\frac{1}{2}\sigma^2 T}{\sigma\sqrt{T}}\\
d_2&=d_1-\sigma\sqrt{T}
\end{align*}
\onslide<2>\begin{align*}
d_1&=\frac{\ln\left( \frac{F^{EUR/USD}}{K} \right) +\frac{1}{2}\sigma^2 T}{\sigma\sqrt{T}}\\
d_2&=d_1-\sigma\sqrt{T}\\
K&=\frac{N^{USD}}{N^{EUR}}\\
F&=S^{EUR/USD}e^{(r^{EUR}-r^{USD}) \times T}
\end{align*}
\onslide<3>\begin{align*}
d_1&=\frac{\ln\left( F^{USD/EUR} \times K \right) +\frac{1}{2}\sigma^2 T}{\sigma\sqrt{T}}\\
d_2&=d_1-\sigma\sqrt{T}\\
K&=\frac{N^{USD}}{N^{EUR}}\\
F&=S^{USD/EUR}e^{(r^{USD}-r^{EUR}) \times T}
\end{align*}
\end{overprint}
\end{frame}

\begin{frame}{Option de change - Jargon}
\Huge
\begin{overprint}
\onslide<1>\begin{center}EUR/USD=1.3889\fontsize{60}{70}\selectfont\textcolor{white}{0}\huge\end{center}
\onslide<2>\begin{center}EUR/USD=1.3\fontsize{60}{70}\selectfont8\huge89\end{center}
\onslide<3>\begin{center}EUR/USD=1.388\fontsize{60}{70}\selectfont9\huge\end{center}
\vspace{0.5cm}
\end{overprint}
\large
\begin{overprint}
\onslide<1>On considère 5 chiffres significatifs dans un taux de change.
\onslide<2>Le 3\up{ème} chiffre en partant de la gauche est appelé \textbf{"Big Figure"}.
\onslide<3>Le 5\up{ème} chiffre en partant de la gauche est appelé \textbf{"pips"}.
\end{overprint}
\end{frame}


\begin{frame}{Option de change - Cotation de la prime - Exercice}
On considère les même données de marché que précédemment avec une volatilité $\sigma=12\%$ et on quote la prime d'une option change de maturité 1 an qui reçoit \textcolor{red}{100 millions d'euros} contre \textcolor{blue}{139 millions de dollars}.\\
\vspace{0.5cm}
Les 6 modes de cotations:\\
\vspace{0.5cm}
\begin{tabular}{|l|c|l|}
\hline
Prix en dollars&\visible<2->{$p$}&\visible<2->{6.501 Mios USD}\\
Prix en euros&\visible<3->{$\frac{p}{S}$}&\visible<3->{4.681 Mios EUR}\\
Prix en \% de nominal dollar&\visible<4->{$\frac{p}{N \times K}$}& \visible<4->{4.6771\%}\\
Prix en \% de nominal euro&\visible<5->{$\frac{p}{N \times S}$}& \visible<5->{4.6808\%}\\
Prix en dollars pips per EUR&\visible<6->{$\frac{p}{1e^4}$}& \visible<6->{6.5012 kUSD pips}\\
Prix en euros pips per USD&\visible<7->{$\frac{p}{S \times K \times 1e^4}$}& \visible<7->{3.3675 kEUR pips}\\
\hline
\end{tabular}
\end{frame}

\begin{frame}{Option de change - Delta de change}
Le Delta de change $\delta$ est le pourcentage du nominal en devise 1 qu'il faut vendre pour couvrir la position de change.
\[
\delta=\frac{\partial p}{\partial S}=e^{-r^{EUR} \times T} \times \mathcal{N}(d_1)
\]
On peut exprimer de façon équivalente le delta de change en pourcentage du nominal $\delta^{reverse}$ en devise 2:
\[
\delta^{reverse}=-\frac{\delta \times S}{K}
\]
\\
\visible<2>{\textcolor{red}{\textbf{Attention ces formules supposent que la prime est payée en dollars!!!}}}
\end{frame}

\begin{frame}{Option de change - Delta de change}
Dans le cas où la prime est payée en euros il faut ajuster le delta du paiement de la prime.\\
\vspace{0.5cm}
\begin{center}
\begin{tabular}{|c|c|c|c|}
\hline
delta ccy&premium ccy&Formule&Delta\\
\hline
\% EUR & EUR & \visible<2->{$\delta-p^{EUR}$} &\visible<4->{46.91\%}\\
\% EUR & USD & $\delta$ &\visible<3->{51.59\%}\\
\% USD & EUR & \visible<2->{$-\frac{(\delta-p^{EUR})S}{K}$} &\visible<4->{-46.87\%}\\
\% USD & USD & $-\frac{\delta}{K}$&\visible<3->{-51.55\%}\\
\hline
\end{tabular}
\end{center}
\visible<3->{La prime est égale à 4.6808\% du nominal EUR.}
\end{frame}

\begin{frame}{Option de change - Delta de change}
Comment évolue le delta de change en fonction du strike ?
\begin{figure}
\begin{center}
%\begin{overprint}
\FIG{1}{11cm}{figures/fxopt-delta-0.png}
\FIG{2}{11cm}{figures/fxopt-delta-1.png}
\FIG{3}{11cm}{figures/fxopt-delta-2.png}
\FIG{4}{11cm}{figures/fxopt-delta-3.png}
\FIG{5}{11cm}{figures/fxopt-delta-4.png}
%\end{overprint}
\end{center}
\end{figure}
\end{frame}

\begin{frame}{Black Scholes: $\delta$ versus Vega}
\begin{align*}
\bm{\delta}=\frac{\partial p}{\partial S}=e^{-r^{1} \times T} \mathcal{N}(d_1)& \text{\textbf{Vega}}=\frac{\partial p}{\partial \sigma}=e^{-r^{f} \times T}S\sqrt{T}\mathcal{N'}(d_1)
\end{align*}
\[
\text{\textbf{Vega}}=e^{-r^{1} \times T}S\sqrt{T}\mathcal{N'}(\mathcal{N}^{-1}(\bm{\delta} e^{r^{1} \times T}))=f(\bm{\delta})
\]
\begin{figure}
\centering
\FIG{1}{9cm}{figures/fxopt-deltavsvega.png}
\end{figure}
\end{frame}

\begin{frame}{Zero delta straddle}
Un \textbf{zéro delta straddle} EURUSD est pour un même strike ($K^{ATM}$) et un même nominal:
\begin{itemize}
\item l'achat d'un call EUR
\item et l'achat d'un put EUR.
\end{itemize}
La delta du portefeuille doit être nul.
\begin{figure}
\begin{center}
\FIG{1}{11cm}{figures/zero-delta-straddle-0.png}
\FIG{2}{11cm}{figures/zero-delta-straddle-1.png}
\FIG{3}{11cm}{figures/zero-delta-straddle-2.png}
\FIG{4}{11cm}{figures/zero-delta-straddle-3.png}
\FIG{5}{11cm}{figures/zero-delta-straddle-4.png}
\FIG{6}{11cm}{figures/zero-delta-straddle-5.png}
\end{center}
\end{figure}
\end{frame}

\begin{frame}{25\% delta risk reversal}
Un \textbf{25\% delta risk reversal} EURUSD est pour un même nominal: 
\begin{itemize}
\item l'achat d'un call EUR de delta égal à 25\% de strike $K^{25Call}$
\item et la vente d'un put EUR de delta égal à -25\% $K^{25Put}$.
\end{itemize}
\begin{figure}
\begin{center}
\FIG{1}{10cm}{figures/risk-reversal-0.png}
\FIG{2}{10cm}{figures/risk-reversal-1.png}
\FIG{3}{10cm}{figures/risk-reversal-2.png}
\end{center}
\end{figure}
\end{frame}

\begin{frame}{25\% delta butterfly}
Un \textbf{25\% delta butterfly} est pour un même nominal:
\begin{itemize}
\item l'achat d'un call EUR de strike $K^{25Call}$
\item l'achat d'un call EUR de strike $K^{25Put}$
\item et la vente de 2 call EUR de strike $K^{ATM}$.
\end{itemize}
\FIG{1}{10cm}{figures/butterfly-0.png}
\FIG{2}{10cm}{figures/butterfly-1.png}
\FIG{3}{10cm}{figures/butterfly-2.png}
\end{frame}

\begin{frame}{Cotation du smile de change}
Les différentes options de changes ne sont pas cotés en prix mais en volatilité.\\
\vspace{0.5cm}

\begin{tabular}{|l|l|l|}
\hline
&\textbf{Cotation}\\
\hline
\textbf{0\% delta straddle}&$\sigma^{ATM}$\\
\textbf{25\% delta risk reversal}&$RR^{25}=\sigma^{25Call}-\sigma^{25Put}$\\
\textbf{25\% delta butterfly}&$BF^{25}=\sigma^{25Call}+\sigma^{25Put}-2 \times \sigma^{ATM}$\\
\hline
\end{tabular}
\vspace{0.5cm}
\\
Comment à partir des cotations de marché des différents produits reconstituer le smile de change?
\end{frame}

\begin{frame}{Cotation du smile de change}
\begin{itemize}
\item \textbf{Etape 1}: On calcule le 3 points de volatilité de change.
\begin{align*}
\sigma^{25Call}&=\sigma^{ATM}+BF^{25}+\frac{1}{2} RR^{25}\\
\sigma^{25Put}&=\sigma^{ATM}+BF^{25}-\frac{1}{2} RR^{25}\\
\end{align*}
\item \textbf{Etape 2}: On calcule les strikes à partir de la volatilité et du delta.
\end{itemize}
\end{frame}

\begin{frame}{Cotation du smile de change - Exercice}
Construire la smile de change 1 an à partir des données suivantes:
\begin{center}
\begin{tabular}{|l|l|l|l|}
\hline
\textbf{Maturité}&1 an&$\mathbf{\sigma^{ATM}}$&12\%\\
\textbf{EUR/USD}&1.3889&$\mathbf{RR^{25}}$&-2\%\\
$\mathbf{r^{USD}}$&0.3\%&$\mathbf{BF^{25}}$&1\%\\
$\mathbf{r^{EUR}}$&0.5\%&$\mathbf{RR^{10}}$&-4\%\\
\textbf{Basis EUR}&0.1\%&$\mathbf{BF{10}}$&4\%\\
\hline
\end{tabular}
\end{center}
\begin{figure}
\FIG{1}{7cm}{figures/fxopt-smile.png}
\end{figure}
\end{frame}

\begin{frame}{Cotation du smile de change - Exercice}

\begin{center}
\begin{tabular}{|c|c|c|c|} 
\hline 
$\mathbf{K^{10Put}}$&\visible<3->{1.1201}&$\mathbf{\sigma^{10Put}}$&\visible<2->{18.0\%}\\
$\mathbf{K^{25Put}}$&\visible<3->{1.2755}&$\mathbf{\sigma^{25Put}}$&\visible<2->{14.0\%}\\
$\mathbf{K^{ATM}}$&\visible<3->{1.3975}&$\mathbf{\sigma^{ATM}}$&\visible<2->{12.0\%}\\
$\mathbf{K^{25Call}}$&\visible<3->{1.5148}&$\mathbf{\sigma^{25Call}}$&\visible<2->{12.0\%}\\
$\mathbf{K^{10Call}}$&\visible<3->{1.6760}&$\mathbf{\sigma^{10Call}}$&\visible<2->{14.0\%}\\
\hline 
\end{tabular}
\end{center} 
\end{frame}

\begin{frame}
\frametitle<1>{Interpolation linéaire}
\frametitle<2->{Interpolation spline cubique}
\begin{align*}
y&=q(x)=(1-t)\times y_{1}+t \times y_{2}\visible<2->{+\underbrace{t \times (1-t) \times (\mathbf{a} \times (1-t)+\mathbf{b} \times t)}_{\text{Termes quadratiques et cubiques}}}\\
t&=\frac{x-x_1}{x_2-x_1}
\end{align*}
\begin{figure}
\begin{center}
\FIG{1,2}{9cm}{figures/interpolation-0.png}
\FIG{3}{9cm}{figures/interpolation-1.png}
\FIG{4}{9cm}{figures/interpolation-2.png}
\end{center}
\end{figure}
\end{frame}

\begin{frame}
\frametitle{Interpolation spline cubique}
On peut facilement calculer les dérivés premières et secondes de $q$ aux points $x_1$ et $x_2$:\\
\vspace{0.5cm}
\begin{tabular}{r l r l r l}
$q'(x)$&$=\frac{\partial q}{\partial x}$&$q'(x_1)$&$=\frac{y_2-y_1}{x_2-x_1}+\frac{a}{x_2-x_1}$&$q'(x_2)$&$=\frac{y_2-y_1}{x_2-x_1}-\frac{b}{x_2-x_1}$\\
$q''(x)$&$=\frac{\partial^2 q}{\partial x^2}$&$q''(x_1)$&$=2\frac{b-2a}{(x_2-x_1)^2}$&$q''(x_2)$&$=2\frac{a-2b}{(x_2-x_1)^2}$\\
\end{tabular}
\\
\vspace{0.5cm}
On peut facilement cacluler $a$ et $b$ en fonctions des dérivés premières:
\begin{align*}
a&=\underbrace{q'(x_1)}_{k_1}(x_2-x_1)-(y_2-y_1)\\
b&=\underbrace{q'(x_2)}_{k_2}(x_2-x_1)+(y_2-y_1)\\
\end{align*}
\end{frame}

\begin{frame}{Interpolation spline cubique}
\begin{figure}
\begin{center}
\FIG{1-}{9cm}{figures/interpolation-3.png}
\end{center}
\end{figure}
On considère $n$ tronçons de spline qui raccordent les $n+1$ points de $(x_0,y_0)$ à  $(x_n,y_n)$.
\end{frame}
\begin{frame}{Interpolation spline cubique}
\begin{center}
\begin{tabular}{c c c}
$k_{i-1}$&$k_i$&$k_{i+1}$\\
\hline
$\frac{y_{i}-y_{i-1}}{x_{i}-x_{i-1}}+\frac{a_i}{x_i-x_{i-1}}$&$\frac{y_{i+1}-y_{i}}{x_{i+1}+x_{i}}+\frac{a_{i+1}}{x_{i+1}-x_{i}}$&\\
&$\frac{y_{i}-y_{i-1}}{x_{i}-x_{i-1}}-\frac{b_i}{x_i-x_{i-1}}$&$\frac{y_{i+1}-y_{i}}{x_{i+1}+x_{i}}-\frac{b_{i+1}}{x_{i+1}-x_{i}}$\\
\hline
\end{tabular}
\end{center}
\begin{overprint}
\onslide<1>\[
q''(x_i)=2\frac{b_i-2a_i}{(x_{i}-x_{i-1})^2}=2\frac{a_{i+1}-2b_{i+1}}{(x_{i+1}-x_{i})^2}
\]
\onslide<2>\[
\underbrace{\frac{1}{x_i-x_{i-1}}k_{i-1}}_{a_{i,i-1}}+\underbrace{2\big[ \frac{1}{x_i-x_{i-1}} + \frac{1}{x_{i+1}-x_{i}}  \big]}_{a_{i,i}} k_{i} +\underbrace{\frac{1}{x_{i+1}-x_{i}}}_{a_{i,i+1}}k_{i+1}
\]
\[
=\underbrace{3\big[ \frac{y_{i}-y_{i-1}}{(x_i-x_{i-1})^2} + \frac{y_{i+1}-y_{i}}{(x_{i+1}-x_{i})^2} \big]}_{b_i}
\]
\end{overprint}
\end{frame}

\begin{frame}{Interpolation spline cubique}
Pour les points extrêmes on suppose que la dérivée seconde est nulle:\\
\begin{overprint}
\onslide<1>\begin{align*}
q''(x_0)&=\frac{b_1 -2 a_1}{(x_1-x_0)^2}=0\\
q''(x_n)&=\frac{a_n -2 b_n}{(x_n-x_{n-1})^2}=0\\
\end{align*}
\onslide<2>\begin{align*}
\underbrace{2\frac{1}{x_{1}-x_{0}}  }_{a_{0,0}} k_{0}&+\underbrace{\frac{1}{x_{1}-x_{0}}}_{a_{0,1}}k_{1}&&=\underbrace{3\frac{y_{1}-y_{0}}{(x_{1}-x_{0})^2}}_{b_0}\\
&\underbrace{\frac{1}{x_{n}-x_{n-1}}}_{a_{n,n-1}}k_{n-1}&+\underbrace{2\frac{1}{x_{1}-x_{0}}}_{a_{n,n}} k_{n-1} &=\underbrace{3\frac{y_{n}-y_{n-1}}{(x_{n}-x_{n-1})^2}}_{b_n}
\end{align*}
\end{overprint}
\end{frame}
\begin{frame}{Interpolation spline cubique}
Il nous faut maintenant résoudre le système linéaire précedemment défini où $K$ est l'inconnu:\\
\begin{overprint}
\onslide<1>\Huge
\[
A \times K = B
\]
\onslide<2>\footnotesize
\[
\underbrace{\left( 
\begin{array}{ccccccc}
a_{0,0} & a_{0,1} & \hdots & \hdots & \hdots & \hdots & 0 \\
a_{1,0} & \ddots & \ddots & \ddots  &  &  & \vdots  \\
\vdots & & a_{i,i-1} & a_{i,i} & a_{i,i+1} & & \vdots \\
\vdots & & & \ddots & \ddots & \ddots & a_{n-1,n}\\
0 & \hdots & \hdots & \hdots & \hdots & a_{n,n-1} & a_{n,n} 
\end{array}
\right)}_{\text{\huge A}}
\underbrace{\left(
\begin{array}{c}
k_0\\
\vdots\\
k_i\\
\vdots\\
k_n\\
\end{array}
\right)}_{\text{\huge K}}
=
\underbrace{\left(
\begin{array}{c}
b_0\\
\vdots\\
b_i\\
\vdots\\
b_n\\
\end{array}
\right)}_{\text{\huge B}}
\]
\end{overprint}
\end{frame}

\begin{frame}{Interpolation spline cubique - Exercice}
Construire un spline cubique à partir de points de smile calculés précédemment.
\begin{center}
\begin{tabular}{|c|c|c|c|}
\hline
&\textbf{k}&\textbf{a}&\textbf{b}\\
\hline
\visible<2->{0}&\visible<3->{-27.43\%}&\visible<2->{-}&\visible<2->{-}\\
\visible<2->{1}&\visible<3->{-22.38\%}&\visible<3->{-0.26\%}&\visible<3->{-0.52\%}\\
\visible<2->{2}&\visible<3->{-8.37\%}&\visible<3->{-0.73\%}&\visible<3->{-0.98\%}\\
\visible<2->{3}&\visible<3->{7.09\%}&\visible<3->{-0.98\%}&\visible<3->{-0.83\%}\\
\visible<2->{4}&\visible<3->{15.07\%}&\visible<3->{-0.86\%}&\visible<3->{-0.43\%}\\
\hline
\end{tabular}
\end{center}
\end{frame}

\begin{frame}{Option Digitale de change}
\end{frame}

\begin{frame}{Sensibilités au change et aux paramètres de smile}
On peut calculer la sensibilité de chacun des 3 produits 
\begin{itemize}
\item \textbf{ZDS}: Zéro Delta Straddle
\item \textbf{RR25}: Risk Reversal 25 delta
\item \textbf{BF25}: Buterfly 25 delta
\end{itemize}
aux paramètres du smile:\\
\begin{overprint}
\onslide<1>\begin{center}
\begin{tabular}{|c|c|c|c|c|}
\hline
\textbf{Avec Smile}&\textbf{Delta FX}&\textbf{Sensi ATM}&\textbf{Sensi RR25}&\textbf{SensiBF25}\\
\hline
ZDS&5\%&0.56\%&0.00\%&0.00\%\\
RR25&38\%&0.03\%&0.39\%&0.01\%\\
BF25&-2\%&-0.16\%&0.00\%&0.35\%\\
RR10&10\%&-0.00\%&0.32\%&-0.09\%\\
BF10&-4\%&-0.39\%&-0.00\%&0.55\%\\
\hline
\end{tabular}
\end{center}
\onslide<2>\begin{center}
\begin{tabular}{|c|c|c|c|c|}
\hline
\textbf{Sans Smile}&\textbf{Delta FX}&\textbf{Sensi ATM}&\textbf{Sensi RR25}&\textbf{SensiBF25}\\
\hline
ZDS&0\%&0.55\%&0.00\%&0.00\%\\
RR25&50\%&0.00\%&0.44\%&-0.02\%\\
BF25&-0\%&-0.10\%&-0.01\%&0.39\%\\
RR10&20\%&0.00\%&0.48\%&-0.05\%\\
BF10&-0\%&-0.29\%&-0.00\%&0.73\%\\
\hline
\end{tabular}
\end{center}
\end{overprint}
\end{frame}



\end{document}
