\documentclass{beamer}
\usepackage[english,francais]{babel}
\usepackage[utf8]{inputenc}
\usepackage{multicol}

\usepackage{graphicx}
\graphicspath{{./Forex_Credit_Equity/}}

\newcommand{\FIG}[3]{\includegraphics<#1>[width=#2]{#3}}

\usetheme{Warsaw}
\title[Produits dérivés actions,change et credit]{Produits dérivés de change}
\author{Richard Guillemot}
\institute{DIFIQ}
\date{11 Avril 2014}

\begin{document}

\begin{frame}
\titlepage
\end{frame}

\begin{frame}
\frametitle{Taux de change \textbf{"Spot"}}
\Huge
\begin{center}
\textbf{\textcolor<3>{red}{EUR}/\textcolor<4>{blue}{USD}}=1.3885
\end{center}
\huge
\begin{center}
\onslide<2->{1 \textcolor<3>{red}{euro} vaut 1.3885 \textcolor<4>{blue}{dollar}.}
\end{center}
\large
\vspace{0.5cm}
\begin{itemize}
\item[]<3-> \textbf{\textcolor<3>{red}{EUR (euro)}} est la \textcolor<3>{red}{devise étrangère ou devise 1}.
\item[]<4-> \textbf{\textcolor<4>{blue}{USD (dollar)}} est la \textcolor<4>{blue}{devise domestique ou devise 2}.
\end{itemize}
\end{frame}

\begin{frame}
\frametitle{Livraison ou Settlement}
\begin{center}
\begin{figure}[h]
\FIG{1}{12cm}{figures/settlement-1.png}
\FIG{2}{12cm}{figures/settlement-2.png}
\FIG{3}{12cm}{figures/settlement-3.png}
\FIG{4}{12cm}{figures/settlement-4.png}
\end{figure}
\end{center}
\end{frame}

\begin{frame}
\frametitle{Taux de change \textbf{"Forward"}}
\large
\begin{overprint}
\onslide<1>Comment garantir un taux de change à une date future \textbf{T} ?\\ Et à quel taux \textbf{X}.
\onslide<2>\textbf{Prêt} de$t$ \textcolor{red}{$\frac{1}{1+\delta R^{EUR}}$ euros}.\\Remboursé en $T$ avec les intérêts, c'est à dire \textcolor{red}{1 euros}.
\onslide<3>\textbf{Change} \textcolor{red}{$\frac{1}{1+\delta R^{EUR}}$ euros} contre \textcolor{blue}{$\frac{S}{1+\delta r^{EUR}}$dollars}.
\onslide<4>\textbf{Emprunt} en $t$ de \textcolor{blue}{$\frac{S}{1+\delta R^{EUR}}$} dollars\\ Remboursé en $T$ avec les intérêts, c'est à dire \textcolor{blue}{$S\frac{1+\delta R^{USD}}{1+\delta R^{EUR}}$ dollars}.
\onslide<5>\[X=S\frac{1+\delta R^{USD}}{1+\delta R^{EUR}}\]
\end{overprint}
\begin{figure}
\begin{overprint}
\FIG{1}{11cm}{figures/fxfwd-1.png}
\FIG{2}{11cm}{figures/fxfwd-2.png}
\FIG{3}{11cm}{figures/fxfwd-3.png}
\FIG{4}{11cm}{figures/fxfwd-4.png}
\FIG{5}{11cm}{figures/fxfwd-5.png}
\end{overprint}
\end{figure}
\end{frame}

\begin{frame}
\small
\frametitle{Taux de change \textbf{"Forward"} - Récapitulatif}
\begin{center}
\begin{tabular}{|l|l|l|l|}
\hline
\textbf{Notation} & \textbf{Description} & \textbf{Formule} & \textbf{Valeur} \\
\hline
\hline
\onslide<2->
$\delta$ & Maturité du forward & $T-(t+2D)$ & 1 an = 365 jours  \\
\onslide<3->
$R^{EUR}$ & Taux zéro coupon euro. &  & 0.5\% \\
\onslide<4->
$R^{USD}$ & Taux zéro coupon euro. &  & 0.3\% \\
\onslide<5->
$S$ & Taux de change spot. &  & 1.3889 \\
\onslide<6->
$X$ & Forward de change. &  $S\frac{1+\delta R^{USD}}{1+\delta R^{EUR}}$ & ?? \\
\hline
\end{tabular}
\end{center}
\[
\uncover<7->{X=} \uncover<8->{1.3889\times \frac{1+\frac{365}{360} \times 0.3\%}{1+\frac{365}{360} \times 0.5\%}\uncover<9->{=1.3861}} 
\]
\onslide<10->{Soit \textbf{27.6} points de base d'écart négatif par rapport au taux spot.}
\end{frame}

\begin{frame}
\frametitle{Quizz}
Si on vend 100 Mios euro dans 1 an d'euros au taux spot au lieu d'utiliser le taux foward précedemment calculé:\\
\vspace{0.5cm}
a) On gagne 276 kEUR\\
b) On perd 27 kEUR\\
c) On gagne 2.76 millions d'euros.\\
d) On perd 276 kEUR.\\
\end{frame}


\end{document}
