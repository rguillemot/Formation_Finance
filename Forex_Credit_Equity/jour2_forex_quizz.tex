\documentclass{beamer}
\usepackage[english,francais]{babel}
\usepackage[utf8]{inputenc}
\usepackage{multicol}
\usepackage{bm}
\usepackage{textcomp}

\usepackage{graphicx}
\graphicspath{{./Forex_Credit_Equity/}}

\newcommand{\FIG}[3]{\includegraphics<#1>[width=#2]{#3}}

\newcommand{\FIGSCALE}[3]{\includegraphics<#1>[resolution=72dpi]{#3}}

\usetheme{Warsaw}
\title[Produits dérivés actions,change et credit]{Produits dérivés de change}
\author{Richard Guillemot}
\institute{DIFIQ}
\date{5 Mai 2015}

\begin{document}

\begin{frame}
\titlepage
\end{frame}

\begin{frame}
\small
\frametitle{Données de marché EURUSD au 2 Avril 2015}
\begin{center}
\begin{tabular}{|l|l|l|l|}
\hline
\textbf{Notation} & \textbf{Description} & \textbf{Formule} & \textbf{Valeur} \\
\hline
\hline
$\delta$ & Maturité du forward & $T-(t+2D)$ & 1 an = 365 jours \\
$R^{EUR}$ & Taux zéro coupon euro. &  & 0.01\% \\
$R^{USD}$ & Taux zéro coupon dollar. &  & 0.45\% \\
$S$ & Taux de change spot. &  & 1.08785 \\
$m$ & Marge de basis. &  & 27 bps \\
\hline
\end{tabular}
\end{center}
\end{frame}

\begin{frame}
\frametitle{Quizz 7 - Opération au comptant.}
Je vend au comptant 50 millions d'euros contre 54.4 millions de dollar: \\
\vspace{0.5cm}
a) je perds 690 euros. \visible<2->{\textcolor{red}{\textbf{FAUX}}}\\
b) je gagne 6 900 euros. \visible<2->{\textcolor{green}{\textbf{VRAI}}} \\
c) je perds 69 000 euros. \visible<2->{\textcolor{red}{\textbf{FAUX}}}\\
d) je perds 690 000 euros. \visible<2->{\textcolor{red}{\textbf{FAUX}}} \\
\vspace{0.5cm}
\end{frame}

\begin{frame}
\frametitle{Quizz 7 - Opération au comptant.}
\huge
\[
\begin{split}
&\text{-50 millions \$}+\frac{\text{54.4 millions \$}}{1.08785}\\
&=\text{6 894 \texteuro}\\
\end{split}
\]
\end{frame}


\begin{frame}
\frametitle{Quizz 8 - Opération au comptant.}
Par la suite le cour de l'euro dollar augmente de 1 figure: \\
\vspace{0.5cm}
a) je gagne 455 euros. \visible<2->{\textcolor{red}{\textbf{FAUX}}}\\
b) je perds 4 555 euros. \visible<2->{\textcolor{red}{\textbf{FAUX}}} \\
c) je gagne 45 550 euros. \visible<2->{\textcolor{red}{\textbf{FAUX}}}\\
d) je perds 455 500 euros. \visible<2->{\textcolor{green}{\textbf{VRAI}}} \\
\vspace{0.5cm}
\uncover<2->{Le nouveau cour de l'euro dollar est de 1.09875}
\end{frame}


\begin{frame}
\frametitle{Quizz 8 - Opération au comptant.}
\huge
\[
\begin{split}
&\frac{\text{54.4 millions \$}}{1.09785}-\frac{\text{54.4 millions \$}}{1.08785}\\
&=\text{-455 498 \texteuro}\\
\end{split}
\]
\end{frame}


\begin{frame}
\frametitle{Quizz 9 - Opération à terme.}
Je vend à terme les 54.4 millions de dollars à un client. Je souhaite réaliser une marge de 50 000 euros. Quelle montant d'euros doit il donc me payer au terme de la transaction:\\
\vspace{0.5cm}
a) 49.110 millions d'euros. \visible<2->{\textcolor{red}{\textbf{FAUX}}}\\
b) 49.160 millions d'euros. \visible<2->{\textcolor{red}{\textbf{FAUX}}}\\
c) 49.210 millions d'euros. \visible<2->{\textcolor{green}{\textbf{VRAI}}}\\
d) 49.260 millions d'euros.\visible<2->{\textcolor{red}{\textbf{FAUX}}}\\
\vspace{0.5cm}
\end{frame}


\begin{frame}
\frametitle{Quizz 9 - Opération à terme.}
\Large
\[
\frac{\textbf{X} \text{\texteuro}}{1+0.01\%-0.27\%}-\frac{54.4 \text{ millions } \$}{[1+0.45\%] \times 1.09785}=50 000 \text{\texteuro}
\]
\end{frame}

\begin{frame}
\frametitle{Quizz 10 - Position de change.}
Quelle est la position de change de l'opération de change à terme précédente:\\
\vspace{0.5cm}
a) Long de 54.4 millions \$ et long de 49.210 millions \texteuro. \visible<2->{\textcolor{red}{\textbf{FAUX}}}\\
b) Short de 54.4 millions \$  et short de 49.210 millions \texteuro.\visible<2->{\textcolor{red}{\textbf{FAUX}}}\\
c) Long de 54.156 millions \$ et short 49.339 millions \texteuro.\visible<2->{\textcolor{red}{\textbf{FAUX}}}\\
d) Short de 54.156 millions \$ et long de 49.339 millions \texteuro.\visible<2->{\textcolor{green}{\textbf{VRAI}}}\\
\vspace{0.5cm}
\end{frame}

\begin{frame}
\frametitle{Quizz 10 - Position de change.}
\Large
\[
\begin{split}
N^{USD}&=\frac{-54.4 \text{ millions \$}}{1+[0.45\%]}\\
N^{EUR}&=\frac{+49.210 \text{ millions \texteuro}}{1+[0.01\%-0.27\%]}\\
\end{split}
\]
\end{frame}

\begin{frame}
\small
\frametitle{Question 11: Calculer le taux de change à terme USDJPY.}
Données de marché USDJPY au 2 Avril 2015:\\
\begin{center}
\begin{tabular}{|l|l|l|l|}
\hline
\textbf{Notation} & \textbf{Description} & \textbf{Formule} & \textbf{Valeur} \\
\hline
\hline
$\delta$ & Maturité du forward & $T-(t+2D)$ & 1 an = 365 jours \\
$R^{USD}$ & Taux zéro coupon dollar. &  & 0.45\% \\
$R^{USD}$ & Taux zéro coupon yen. &  & 0.11\% \\
$S$ & Taux de change spot. &  & 119.76 \\
$m$ & Marge de basis. &  & 39 bps \\
\hline
\end{tabular}
\end{center}
\uncover<2->{Le taux de change à terme USDJPY est de 118.89}
\end{frame}

\begin{frame}
\frametitle{Question 11: Calculer le taux de change à terme USDJPY.}
\LARGE
\[
\begin{split}
F^{USDJPY}&=119.76 \times \frac{1+[0.11\%-0.39\%]}{1+0.45\%}\\
&=118.89\\
\end{split}
\]
\end{frame}

\begin{frame}
\frametitle{Quizz 12: Options de change}
On souhaite acheter à terme un montant en euro contre dollar à un taux prédéterminé.\\
Si l'opération nous est défavorable on peut y renoncer.\\
\vspace{0.5cm}
Quel produit doit on traiter ?\\
\vspace{0.5cm}
a) On achète un call euro put dollar.\visible<2->{\textcolor{green}{\textbf{VRAI}}}\\
b) On vend un call euro put dollar.\visible<2->{\textcolor{red}{\textbf{FAUX}}}\\
c) On achète un put euro call dollar.\visible<2->{\textcolor{red}{\textbf{FAUX}}}\\
d) On vend un put euro call dollar.\visible<2->{\textcolor{red}{\textbf{FAUX}}}\\
\vspace{0.5cm}
\end{frame}

\begin{frame}
\frametitle{Quizz 12: Options de change}
\begin{center}
\FIG{1}{9cm}{figures/fxopt-quizz-1.png}
\end{center}
\end{frame}

\begin{frame}
\frametitle{Quizz 13: Options de change}
On souhaite vendre à terme un montant en euro contre dollar à un taux prédéterminé.\\
Si l'opération nous est défavorable on peut y renoncer.\\
\vspace{0.5cm}
Quel produit doit on traiter ?\\
\vspace{0.5cm}
a) On achète un call euro put dollar.\visible<2->{\textcolor{red}{\textbf{FAUX}}}\\
b) On vend un call euro put dollar.\visible<2->{\textcolor{red}{\textbf{FAUX}}}\\
c) On achète un put euro call dollar.\visible<2->{\textcolor{green}{\textbf{VRAI}}}\\
d) On vend un put euro call dollar.\visible<2->{\textcolor{red}{\textbf{FAUX}}}\\
\vspace{0.5cm}
\end{frame}


\begin{frame}
\frametitle{Quizz 13: Options de change}
\begin{center}
\FIG{1}{9cm}{figures/fxopt-quizz-2.png}
\end{center}
\end{frame}


\begin{frame}
\frametitle{Quizz 14: Options de change}
On souhaite acheter à terme un montant en euro contre dollar à un taux prédéterminé.\\
Si l'opération est défavorable à notre contrepartie elle peut y renoncer.\\
\vspace{0.5cm}
Quel produit doit on traiter ?\\
\vspace{0.5cm}
a) On achète un call euro put dollar.\visible<2->{\textcolor{red}{\textbf{FAUX}}}\\
b) On vend un call euro put dollar.\visible<2->{\textcolor{red}{\textbf{FAUX}}}\\
c) On achète un put euro call dollar.\visible<2->{\textcolor{red}{\textbf{FAUX}}}\\
d) On vend un put euro call dollar.\visible<2->{\textcolor{green}{\textbf{VRAI}}}\\
\vspace{0.5cm}
\end{frame}

\begin{frame}
\frametitle{Quizz 14: Options de change}
\begin{center}
\FIG{1}{9cm}{figures/fxopt-quizz-3.png}
\end{center}
\end{frame}


\begin{frame}
\frametitle{Quizz 15: Options de change}
On souhaite vendre à terme un montant en euro contre dollar à un taux prédéterminé.\\
Si l'opération est défavorable à notre contrepartie elle peut y renoncer.\\
\vspace{0.5cm}
Quel produit doit on traiter ?\\
\vspace{0.5cm}
a) On achète un call euro put dollar.\visible<2->{\textcolor{red}{\textbf{FAUX}}}\\
b) On vend un call euro put dollar.\visible<2->{\textcolor{green}{\textbf{VRAI}}}\\
c) On achète un put euro call dollar.\visible<2->{\textcolor{red}{\textbf{FAUX}}}\\
d) On vend un put euro call dollar.\visible<2->{\textcolor{red}{\textbf{FAUX}}}\\
\vspace{0.5cm}
\end{frame}

\begin{frame}
\frametitle{Quizz 15: Options de change}
\begin{center}
\FIG{1}{9cm}{figures/fxopt-quizz-4.png}
\end{center}
\end{frame}


\end{document}
