\documentclass{beamer}
\usepackage[english,francais]{babel}
\usepackage[utf8]{inputenc}
\usepackage{multicol}
\usepackage{bm}
\usepackage{textcomp}

\usepackage{graphicx}
\graphicspath{{./Forex_Credit_Equity/}}

\newcommand{\FIG}[3]{\includegraphics<#1>[width=#2]{#3}}

\newcommand{\FIGSCALE}[3]{\includegraphics<#1>[resolution=72dpi]{#3}}

\usetheme{Warsaw}
\title[Produits dérivés de change et de credit]{Produits dérivés de change}
\author{Richard Guillemot}
\institute{DIFIQ}
\date{12 Avril 2016}

\begin{document}

\begin{frame}
\titlepage
\end{frame}

\begin{frame}
\frametitle{Quizz 1 et 2 - Taux de change}
\begin{center}
\begin{tabular}{|l|l|l|}
\hline
\textbf{FX} & 2 Avril 2015 & 24 Mars 2016  \\
\hline
\hline
EURUSD & 1.0879 & 1.1162  \\
GBPUSD &  1.4829 & 1.4142  \\
USDCHF & 0.9581 & 0.9767  \\
USDJPY & 119.76 & 112.64  \\
USDCNY & 6.1396 & 6.5125  \\
\hline		
\end{tabular}
\end{center}
\vspace{0.5cm}
Quelles sont les devises qui se sont appréciées contre le dollars ?\\
Quelles sont les devises qui se sont appréciées contre l'euro ?\\
\end{frame}

\begin{frame}
\small
\frametitle{Données de marché EURUSD  24 Mars 2016}
\begin{center}
\begin{tabular}{|l|l|l|l|}
\hline
\textbf{Notation} & \textbf{Description} & \textbf{Formule} & \textbf{Valeur} \\
\hline
\hline
$\delta$ & Maturité du forward & $T-(t+2D)$ & 1 an = 365 jours \\
$R^{EUR}$ & Taux zéro coupon euro. &  & -0.027\% \\
$R^{USD}$ & Taux zéro coupon dollar. &  & 0.81\% \\
$S$ & Taux de change spot. &  & 1.1162 \\
$m$ & Marge de basis. &  & 29 bps \\
\hline
\end{tabular}
\end{center}
\end{frame}

\begin{frame}
\frametitle{Quizz 4 - Carry Trade.}
Une opération classique de carry trade consiste à emprunter en euro et à prêter en dollar.\\ 
A partir de quel niveau de taux de change euro dollari, au terme de cette opérationi, cette opération est profitable ?\\
\[
\begin{split}
&\text{1 millions \texteuro} \times \Big[0.27\%+0.81\%+[\frac{1.1182}{\text{??}}-1]\Big]\\
&\geq\text{0 \texteuro}\\
\end{split}
\]
\visible<2->{\[
\text{EURUSD}\geq 1.1283
\]}
\end{frame}


\begin{frame}
\frametitle{Quizz 5 \& 6 - Forex basis ou marge de cross currency swap.}
A quel taux emprunte en euro une banque américaine qui se finance en dollar ?\\
\center \visible<2->{$-0.27\%-0.29\%=0.56\%$}\\
\vspace{0.5cm}
A quel taux emprunte en dollar une banque européenne qui se finance en euro ?\\  
\center \visible<2->{$0.81\%+0.29\%=1.10\%$}\\
\end{frame}



\begin{frame}
\frametitle{Quizz 7 - Opération au comptant.}
Je vend au comptant 50 millions d'euros contre 55.815 millions de dollar: \\
\vspace{0.5cm}
a) je perds 448 euros. \visible<2->{\textcolor{red}{\textbf{FAUX}}}\\
b) je gagne 4 480 euros. \visible<2->{\textcolor{green}{\textbf{VRAI}}} \\
c) je perds 44 800 euros. \visible<2->{\textcolor{red}{\textbf{FAUX}}}\\
d) je perds 448 000 euros. \visible<2->{\textcolor{red}{\textbf{FAUX}}} \\
\vspace{0.5cm}
\end{frame}

\begin{frame}
\frametitle{Quizz 7 - Opération au comptant.}
\huge
\[
\begin{split}
&\text{-50 millions \$}+\frac{\text{55.815 millions \$}}{1.1162}\\
&=\text{4 479 \texteuro}\\
\end{split}
\]
\end{frame}


\begin{frame}
\frametitle{Quizz 8 - Opération au comptant.}
Par la suite le cour de l'euro dollar augmente de 1 figure: \\
\vspace{0.5cm}
a) je gagne 439 euros. \visible<2->{\textcolor{red}{\textbf{FAUX}}}\\
b) je perds 4 395 euros. \visible<2->{\textcolor{red}{\textbf{FAUX}}} \\
c) je gagne 43 950 euros. \visible<2->{\textcolor{red}{\textbf{FAUX}}}\\
d) je perds 439 500 euros. \visible<2->{\textcolor{green}{\textbf{VRAI}}} \\
\vspace{0.5cm}
\uncover<2->{Le nouveau cour de l'euro dollar est de 1.1262}
\end{frame}


\begin{frame}
\frametitle{Quizz 8 - Opération au comptant.}
\huge
\[
\begin{split}
&\frac{\text{55.815 millions \$}}{1.1162}-\frac{\text{55.815 millions \$}}{1.1262}\\
&=\text{-439 531 \texteuro}\\
\end{split}
\]
\end{frame}


\begin{frame}
\frametitle{Quizz 9 - Opération à terme.}
Je vend à terme les 55.815 millions de dollars à un client. Je souhaite réaliser une marge de 50 000 euros. Quelle montant d'euros doit il donc me payer au terme de la transaction:\\
\vspace{0.5cm}
a) 49.275 millions d'euros. \visible<2->{\textcolor{red}{\textbf{FAUX}}}\\
b) 49.325 millions d'euros. \visible<2->{\textcolor{red}{\textbf{FAUX}}}\\
c) 49.375 millions d'euros. \visible<2->{\textcolor{green}{\textbf{VRAI}}}\\
d) 49.425 millions d'euros.\visible<2->{\textcolor{red}{\textbf{FAUX}}}\\
\vspace{0.5cm}
\end{frame}


\begin{frame}
\frametitle{Quizz 9 - Opération à terme.}
\Large
\[
\frac{\textbf{X} \text{\texteuro}}{1-0.27\%-0.29\%}-\frac{55.815 \text{ millions } \$}{[1+0.81\%] \times 1.1162}=50 000 \text{\texteuro}
\]
\end{frame}

\begin{frame}
\frametitle{Quizz 10 - Position de change.}
Quelle est la position de change de l'opération de change à terme précédente:\\
\vspace{0.5cm}
a) Long de 55.815 millions \$ et long de 49.375 millions \texteuro. \visible<2->{\textcolor{red}{\textbf{FAUX}}}\\
b) Short de 55.815 millions \$  et long de 49.375 millions \texteuro.\visible<2->{\textcolor{red}{\textbf{FAUX}}}\\
c) Long de 55.366 millions \$ et short 49.653 millions \texteuro.\visible<2->{\textcolor{red}{\textbf{FAUX}}}\\
d) Short de 55.366 millions \$ et long de 49.653 millions \texteuro.\visible<2->{\textcolor{green}{\textbf{VRAI}}}\\
\vspace{0.5cm}
\end{frame}

\begin{frame}
\frametitle{Quizz 10 - Position de change.}
\Large
\[
\begin{split}
N^{USD}&=\frac{-55.815 \text{ millions \$}}{1+[0.81\%]}\\
N^{EUR}&=\frac{+49.375 \text{ millions \texteuro}}{1+[-0.27\%-0.29\%]}\\
\end{split}
\]
\end{frame}

\begin{frame}
\frametitle{Quizz 10 bis}
Si on vend 100 millons euro dans 1 an au taux spot au lieu d'utiliser le taux foward précedemment calculé:\\
\vspace{0.5cm}
a) On perd 1.37 millios d'euros \uncover<2>{\textcolor{green}{\textbf{VRAI}}}\\
b) On gagne 137,7 kEUR \uncover<2>{\textcolor{red}{\textbf{FAUX}}}\\
c) On perd 13.7 millions d'euros. \uncover<2>{\textcolor{red}{\textbf{FAUX}}}\\
d) On gagne 137 kEUR. \uncover<2>{\textcolor{red}{\textbf{FAUX}}}\\
\vspace{0.5cm}
\uncover<2>{On emprunte à 0.81\% en dollars et on prête à -0.27\%  -0.29\% en euros !!!}
\end{frame}

\begin{frame}
\small
\frametitle{Question 11: Calculer le taux de change à terme USDJPY.}
Données de marché USDJPY au 24 Mars 2016:\\
\begin{center}
\begin{tabular}{|l|l|l|l|}
\hline
\textbf{Notation} & \textbf{Description} & \textbf{Formule} & \textbf{Valeur} \\
\hline
\hline
$\delta$ & Maturité du forward & $T-(t+2D)$ & 1 an = 365 jours \\
$R^{USD}$ & Taux zéro coupon dollar. &  & 0.81\% \\
$R^{USD}$ & Taux zéro coupon yen. &  & -0.08\% \\
$S$ & Taux de change spot. &  & 112.64 \\
$m$ & Marge de basis. &  & 56 bps \\
\hline
\end{tabular}
\end{center}
\uncover<2->{Le taux de change à terme USDJPY est de 111.02}
\end{frame}

\begin{frame}
\frametitle{Question 11: Calculer le taux de change à terme USDJPY.}
\LARGE
\[
\begin{split}
F^{USDJPY}&=112.64 \times \frac{1+[-0.08\%-0.56\%]}{1+0.81\%}\\
&=111.02\\
\end{split}
\]
\end{frame}

\begin{frame}<handout:0>
\frametitle{Quizz 12: Options de change}
On souhaite acheter à terme un montant en euro contre dollar à un taux prédéterminé.\\
Si l'opération nous est défavorable on peut y renoncer.\\
\vspace{0.5cm}
Quel produit doit on traiter ?\\
\vspace{0.5cm}
a) On achète un call euro put dollar.\visible<2->{\textcolor{green}{\textbf{VRAI}}}\\
b) On vend un call euro put dollar.\visible<2->{\textcolor{red}{\textbf{FAUX}}}\\
c) On achète un put euro call dollar.\visible<2->{\textcolor{red}{\textbf{FAUX}}}\\
d) On vend un put euro call dollar.\visible<2->{\textcolor{red}{\textbf{FAUX}}}\\
\vspace{0.5cm}
\end{frame}

\begin{frame}<handout:0>
\frametitle{Quizz 12: Options de change}
\begin{center}
\FIG{1}{9cm}{figures/fxopt-quizz-1.png}
\end{center}
\end{frame}

\begin{frame}<handout:0>
\frametitle{Quizz 13: Options de change}
On souhaite vendre à terme un montant en euro contre dollar à un taux prédéterminé.\\
Si l'opération nous est défavorable on peut y renoncer.\\
\vspace{0.5cm}
Quel produit doit on traiter ?\\
\vspace{0.5cm}
a) On achète un call euro put dollar.\visible<2->{\textcolor{red}{\textbf{FAUX}}}\\
b) On vend un call euro put dollar.\visible<2->{\textcolor{red}{\textbf{FAUX}}}\\
c) On achète un put euro call dollar.\visible<2->{\textcolor{green}{\textbf{VRAI}}}\\
d) On vend un put euro call dollar.\visible<2->{\textcolor{red}{\textbf{FAUX}}}\\
\vspace{0.5cm}
\end{frame}


\begin{frame}<handout:0>
\frametitle{Quizz 13: Options de change}
\begin{center}
\FIG{1}{9cm}{figures/fxopt-quizz-2.png}
\end{center}
\end{frame}


\begin{frame}<handout:0>
\frametitle{Quizz 14: Options de change}
On souhaite acheter à terme un montant en euro contre dollar à un taux prédéterminé.\\
Si l'opération est défavorable à notre contrepartie elle peut y renoncer.\\
\vspace{0.5cm}
Quel produit doit on traiter ?\\
\vspace{0.5cm}
a) On achète un call euro put dollar.\visible<2->{\textcolor{red}{\textbf{FAUX}}}\\
b) On vend un call euro put dollar.\visible<2->{\textcolor{red}{\textbf{FAUX}}}\\
c) On achète un put euro call dollar.\visible<2->{\textcolor{red}{\textbf{FAUX}}}\\
d) On vend un put euro call dollar.\visible<2->{\textcolor{green}{\textbf{VRAI}}}\\
\vspace{0.5cm}
\end{frame}

\begin{frame}<handout:0>
\frametitle{Quizz 14: Options de change}
\begin{center}
\FIG{1}{9cm}{figures/fxopt-quizz-3.png}
\end{center}
\end{frame}


\begin{frame}<handout:0>
\frametitle{Quizz 15: Options de change}
On souhaite vendre à terme un montant en euro contre dollar à un taux prédéterminé.\\
Si l'opération est défavorable à notre contrepartie elle peut y renoncer.\\
\vspace{0.5cm}
Quel produit doit on traiter ?\\
\vspace{0.5cm}
a) On achète un call euro put dollar.\visible<2->{\textcolor{red}{\textbf{FAUX}}}\\
b) On vend un call euro put dollar.\visible<2->{\textcolor{green}{\textbf{VRAI}}}\\
c) On achète un put euro call dollar.\visible<2->{\textcolor{red}{\textbf{FAUX}}}\\
d) On vend un put euro call dollar.\visible<2->{\textcolor{red}{\textbf{FAUX}}}\\
\vspace{0.5cm}
\end{frame}

\begin{frame}<handout:0>
\frametitle{Quizz 15: Options de change}
\begin{center}
\FIG{1}{9cm}{figures/fxopt-quizz-4.png}
\end{center}
\end{frame}


\end{document}
