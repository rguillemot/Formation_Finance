\documentclass{beamer}
\usepackage[english,francais]{babel}
\usepackage[utf8]{inputenc}
\usepackage{multicol}
\usepackage{bm}

\usepackage{graphicx}
\graphicspath{{./Forex_Credit_Equity/}}

\newcommand{\FIG}[3]{\includegraphics<#1>[width=#2]{#3}}

\newcommand{\FIGSCALE}[3]{\includegraphics<#1>[resolution=72dpi]{#3}}

\usetheme{Warsaw}
\title[Produits dérivés actions,change et credit]{Produits dérivés de change\\Séance Webex}
\author{Richard Guillemot}
\institute{DIFIQ}
\date{15 Avril 2014}

\begin{document}

\begin{frame}
\titlepage
\end{frame}

\begin{frame}
\frametitle{Besoin d'un client américain.}
Un client américain doit payer son fournisseur français dans 1 an \textbf{\textcolor{red}{100 millions d'euros}}.\\ 
\uncover<2->{Pour des raisons "stratégiques" il ne souhaite pas couvrir cette position de change à terme.\\}
\vspace{0.5cm}
\uncover<3->{Cependant il souhaite tout de même se protéger contre des mouvements trop importants du taux de change.\\
Ainsi:\\
\begin{itemize}
\item Il veut payer au \textbf{maximum} \textbf{\textcolor{blue}{149 millions de dollars}}.
\item A l'inverse il veut payer au \textbf{minimum} \textbf{\textcolor{blue}{129 millions de dollars}}.
\end{itemize}}
\vspace{0.5cm}
\uncover<4->{\textbf{Comment statisfaire le besoin de notre client ?}}

\vspace{0.5cm}
\end{frame}

\begin{frame}{Question}
Est ce que l'on rend vraiment un service à notre client ?\\
\vspace{0.5cm}
\begin{itemize}
\uncover<2->{\item \textbf{Oui:} Lorsque l'EUR/USD passe au dessus de 1.49.\\}
\uncover<3->{\item \textbf{Non:} Lorsque l'EUR/USD passe en dessous de 1.29.\\}
\end{itemize}
\end{frame}

\begin{frame}{Question}
Quel est le payoff du produit que la banque vend à son client ?\\
\vspace{0.5cm}
\begin{overprint}
\FIG{1}{11cm}{figures/probleme_rr_payoff.png}
\FIG{2}{11cm}{figures/probleme_rr_payoff_sol.png}
\end{overprint}
\end{frame}

\begin{frame}{Question}
Ce produit, \textbf{le Risk Reversal}, est équivalent à:\\
\vspace{0.5cm}
\begin{enumerate}
\item vendre 100 millions de call euro à 1.49 et acheter 100 millions de put euro à 1.29 \uncover<2>{\textcolor{red}{\textbf{FAUX}}}.
\item acheter 100 millions de call euro à 1.49 et acheter 100 millions de put euro à 1.29 \uncover<2>{\textcolor{red}{\textbf{FAUX}}}.
\item acheter 100 millions de call euro à 1.49 et vendre 100 millions de put euro à 1.29 \uncover<2>{\textcolor{green}{\textbf{VRAI}}}.
\item acheter 100 millions de call euro à 1.49 et vendre 100 millions de call euro à 1.29 \uncover<2>{\textcolor{red}{\textbf{FAUX}}}.
\end{enumerate}

\end{frame}

\begin{frame}{Valorisation et couverture}
A partir des données de marché suivantes:\\
\begin{center}
\begin{tabular}{|l|l|}
\hline
\textbf{Notation} & \textbf{Valeur} \\
\hline
\hline
$S$ & 1.3889 \\
$R^{EUR}$ & 0.5\% \\
$R^{USD}$ & 0.3\% \\
$m$ & 0.1 \% \\
$\sigma$ & 12\% \\
\hline
\end{tabular}
\end{center}
Calculer:
\begin{itemize}
\item La valeur du produit. Le client doit il vraiment nous payer ?\\
\item Le delta de change. \\
\item Le vega de change. \\ 
\end{itemize}
\end{frame}

\begin{frame}{Structuration}
Modifier les caractéristiques du produit de telle façon que:\\
\begin{itemize}
\item le PNL du produit soit nul.
\item le Delta FX du produit soit de 50 \%.
\end{itemize}
\vspace{0.5cm}
Dans les 2 cas calculer le PNL, le Delta FX et le Vega du produit modifié.
\end{frame}

\begin{frame}{Le Risk Reversal 25 Delta}
Le Risk Reversal 25 Delta est:\\
\begin{enumerate}
\item l'achat d'un call euro de delta 25\%.\\
\item la vente d'un put euro de delta -25 \%.
\end{enumerate}
\Large
\begin{align*}
K^{25 Delta Call}&=F\times e^{\mathcal{N}^{-1}(0.25 \times e^{r^{EUR}\times T})\times \sigma \sqrt{T} +\frac{1}{2}\sigma^2 \times T }\\
K^{25 Delta Put}&=F\times e^{\mathcal{N}^{-1}(-0.25 \times e^{r^{EUR}\times T})\times \sigma \sqrt{T} -\frac{1}{2}\sigma^2 \times T }
\end{align*}
\normalsize
Il est coté comme un différence de volatilité:\\
\[
	RR^{25 Delta}=\sigma(K^{25 Delta Call})-\sigma(K^{25 Delta Put})
\]
\end{frame}

\begin{frame}{Black Scholes: $\delta$ versus Vega}
\begin{align*}
\bm{\delta}=\frac{\partial p}{\partial S}=e^{-r^{EUR} \times T} \mathcal{N}(d_1)& \text{\textbf{Vega}}=\frac{\partial p}{\partial \sigma}=e^{-r^{EUR} \times T}S\sqrt{T}\mathcal{N'}(d_1)
\end{align*}
\[
\text{\textbf{Vega}}=e^{-r^{EUR} \times T}S\sqrt{T}\mathcal{N'}(\mathcal{N}^{-1}(\bm{\delta} e^{r^{EUR} \times T}))=f(\bm{\delta})
\]
\begin{figure}
\centering
\FIG{1}{9cm}{figures/fxopt-deltavsvega.png}
\end{figure}
\end{frame}

\begin{frame}{Question}
Pour un mouvement de 1 \% de la volatilité monnaie à la hausse, la valeur du Risk Reversal 25 Delta:\\
\begin{enumerate}
\item baisse de 50 kEUR.\uncover<2>{\textcolor{red}{\textbf{FAUX}}}
\item baisse de 440 kEUR.\uncover<2>{\textcolor{red}{\textbf{FAUX}}}
\item fluctue de quelques milliers d'euros.\uncover<2>{\textcolor{green}{\textbf{VRAI}}}
\item reste extactement la même.\uncover<2>{\textcolor{red}{\textbf{FAUX}}}
\end{enumerate}
\end{frame}

\begin{frame}{Question}
Pour un mouvement à la baisse de 1 \% de sa volatilité, la valeur du Risk Reversal 25 Delta:\\
\begin{enumerate}
\item baisse de 260 kEUR.\uncover<2>{\textcolor{red}{\textbf{FAUX}}}
\item baisse de 440 kEUR.\uncover<2>{\textcolor{green}{\textbf{VRAI}}}
\item fluctue de quelques milliers d'euros.\uncover<2>{\textcolor{red}{\textbf{FAUX}}}
\item reste extactement la même.\uncover<2>{\textcolor{red}{\textbf{FAUX}}}
\end{enumerate}
\end{frame}

\begin{frame}{Conclusion}
Ce que nous avons appris:
\begin{enumerate}
\item<2-> Le Risk Reversal est une stratégie permettant une couverture partielle d'un risque de change. Il est particulièrement adapté pour se protéger contre des mouvements importants du taux de change.
\item<3-> Il s'autofinance.
\item<4-> Il est peu sensible au niveau moyen du smile de change.
\item<5-> Il est très sensible à la pente du smile. 
\end{enumerate}
\end{frame}

\end{document}

