\documentclass{beamer}
\usepackage[english,francais]{babel}
\usepackage[utf8]{inputenc}
\usepackage{multicol}
\usepackage{bm}
\usepackage{tikz}
\usepackage{bbm}
\usepackage{cancel}
\usetikzlibrary{positioning}

\usepackage{graphicx}
\usepackage{multirow}
\graphicspath{{./Forex_Credit_Equity/}}

\newcommand{\FIG}[3]{\includegraphics<#1>[width=#2]{#3}}
\newcommand{\FIGSCALE}[3]{\includegraphics<#1>[resolution=72dpi]{#3}}

\usetheme{Warsaw}
\title[Produits dérivés actions,change et credit]{Produits dérivés de crédit}
\author{Richard Guillemot}
\institute{DIFIQ}
\date{22 Avril 2014}

\begin{document}

\begin{frame}
\titlepage
\end{frame}

\begin{frame}{Le Crédit}

\begin{exampleblock}{}
  {\large ``Un crédit est une mise à disposition d'argent sous forme de prêt, consentie par un \textcolor{red}{\textbf{créancier (créditeur, prêteur)}} à \textcolor{blue}{\textbf{un débiteur (emprunteur)}}.''}
\end{exampleblock}
\vspace{0.5cm}
\visible<2->{Le \textbf{Risque de Crédit} fait référence à l'incapacité du \textcolor{red}{débiteur} de remplir ses engagements (le remboursement du capital ou le paiement des intérêts) totalement ou en partie. On dit alors que ce dernier fait défaut.\\
Le \textbf{Risque de Crédit} est intégralement porté par \textcolor{blue}{le créancier}.} 
\end{frame}


\begin{frame}{Vocabulaire}
\begin{center}
\begin{tabular}{|l|l|}
\hline
\textbf{Français} & \textbf{Anglais} \\
\hline
\hline
Prêt, Crédit, Obligation&\visible<2->{Loan, Credit, Bond}\\ 
\hline
Créditeur, Prêteur&\visible<3->{Creditor, Lender}\\ 
\hline
Créancier, Emprunteur&\visible<4->{Debtor, Borrower}\\ 
\hline
(Qualité de) crédit&\visible<5->{Creditworthiness}\\
\hline
Défaut, Insolvabilité, Faillite&\visible<6->{Default, Insolvency, Bankruptcy}\\
\hline
Capital prêté, Intérêts & \visible<7->{Princpal, Interests}\\
\hline
Remboursement & \visible<8->{Repayment}\\
\hline
Prêt hypothécaire ou garanti & \visible<9->{Mortgage or secured loan}\\
\hline
Saisie&\visible<10->{Foreclosure}\\
\hline
\hline
\end{tabular}
\end{center}
\end{frame}

\begin{frame}{Struture du capital (ou passif) d'une entreprise.}
\begin{overprint}
\begin{tabular}{|c|lccc}
\cline{1-1}
\textbf{Capital}\\
\cline{1-1}
\cline{1-1}
Secured Debt& \multirow{4}{*}{$\left.\rule{0cm}{1cm}\right\}{\text{Dettes}}$}&\\
Senior Debt&\\
Subordinate Debt&\\
Mezzanine Debt&\\
\cline{1-1}
& \multirow{3}{*}{$\left.\rule{0cm}{0.75cm}\right\}{\text{Fonds propres}}$}&\\
Equity\\
\\
\cline{1-1}
\end{tabular}
\onslide<5>
\vspace{0.5cm}
Ratio à respecter pour une banque $\frac{\text{Fonds Propres}}{\text{Dettes}}\geq 8\%$
%\begin{tikzpicture}
%\draw (1,1) -- (2,2)
%\end{tikzpicture}
\end{overprint}
\begin{tikzpicture}[remember picture,overlay]
\draw<2> [->,black, ultra thick] (8,1.5) node[yshift=-0.2cm]{\textbf{Première perte}} -- (8,4.4) node[yshift=0.2cm] {\textbf{Dernière perte}};
\draw<3> [<-,black, ultra thick] (8,1.5) node[yshift=-0.2cm]{\textbf{Risque Maximum}} -- (8,4.4) node[yshift=0.2cm] {\textbf{Risque Minimum}};
\draw<4-> [<-,black, ultra thick] (8,1.5) node[yshift=-0.2cm]{\textbf{Intérêts élevés}} -- (8,4.4) node[yshift=0.2cm] {\textbf{Intérêts faibles}};
\end{tikzpicture}
\end{frame}

\begin{frame}{Risque de crédit}
Comment atténuer le risque de crédit ? \\
\visible<2->{En anglais : How to mitigate the credit risk ?}\\
\begin{enumerate}
\item \visible<3->{\textbf{Valorisation basée sur la qualité de crédit} : Introduction d'une marge (ou spread) de crédit.}
\item \visible<4->{\textbf{Conventions de crédit (Loan covenants)}: clauses restrictives pour le prêteur}.
\item \visible<5->{\textbf{Limtes de Risque}}
\item \visible<6->{\textbf{Diversification}}
\item \visible<7->{\textbf{Dépot de garantié}: Hypothèque, Collatéral, Haircut}
\item \visible<8->{\textbf{Securitization}}
\item \visible<9->{\textbf{Assurance ou \textcolor{red}{dérivés de crédits}}}
\end{enumerate}
\end{frame}

\begin{frame}{Dérivés de crédit}
\begin{exampleblock}{}
  {\large ``Un dérivé de crédit est instrument financier conçu pour transférer le risque de crédit associé à un \textcolor{red}{\textbf{emprunteur}} à une entité autre que le \textcolor{blue}{\textbf{prêteur}}.''}
\end{exampleblock}
\vspace{0.5cm}
\onslide<2->L'entité \textcolor{blue}{\textbf{vendeuse de protection}} et à laquelle est transférée le risque est dite \textcolor{blue}{\textbf{"Long Credit"}}.\\
\vspace{0.5cm}
\onslide<3->L'entité \textcolor{red}{\textbf{acheteuse de protection}} et qui transfert le risque est dite \textcolor{red}{\textbf{"Short Credit"}}.\\
\end{frame}

\begin{frame}{Unfunded or Funded Credit Derivatives}
Il existe 2 différents types de dérivés de crédit:
\vspace{0.5cm}
\begin{itemize}
\item \onslide<2->\textbf{Unfunded Credit Derivatives}: le vendeur de protection de détient pas d'actif. Exemple: Crédit Default Swap.
\item \onslide<3->\textbf{Funded Credit Derivatives}: le vendeur de protection détient un actif. Exemple: Credit Linked Note.
\end{itemize}
\vspace{0.5cm}
\onslide<4>Il est possible d'acheter un Crédit Default Swap sans détenir aucune créance associée!!!!
\end{frame}

\begin{frame}{Credit Default Swap}
\vspace{0.5cm}
\begin{center}
\begin{tabular}{|l|l|}
\hline
$N$&\visible<2->{Nominal}\\
$C$&\visible<3->{Coupon ou prime du CDS}\\ 
$Rec$&\visible<4->{Taux de recouvrement (Recovery rate)}\\
$\tau$&\visible<5>{Temps de défaut}\\
\hline
\end{tabular}
\end{center}
\begin{center}
\begin{figure}
\FIG{1-}{11cm}{figures/cds.png}
\end{figure}
\end{center}
\end{frame}

\begin{frame}{La courbe de taux "sans risque".}
$B(t,T)$ est la valeur en $t$ de l'obligation zéro coupon qui paie 1 unité de nominal en $T$:
\begin{overprint}
\onslide<1>\[
B(t,T)=\mathbb{E}\big[e^{\int_t^T r_s ds}|\mathcal{F}_t\big]
\]
\onslide<2->\[
B(t,T)=e^{-\int_t^T r_s ds}
\]
\end{overprint}
\vspace{0.5cm}
$r$ est appelé le taux court.\\
\onslide<2->On supposera par la suite que ce taux est déterministe.\\
\vspace{0.5cm}
\onslide<3>Dans le cas où $t=0$ on simplifiera la notation.\\
\[
B(T)=B(0,T)
\]
\end{frame}

\begin{frame}{La Probabilité de défaut}
$Q(t,T)$ est la probabilité vue de la date $t$ de ne pas faire défaut à une date ultérieur $T$:\\
\begin{overprint}
\onslide<1>\[
Q(t,T)=\mathbb{P}\big[\tau>T|\mathcal{F}_t\big]
\]
\onslide<2-5>\[
Q(t,T)=\mathbb{P}\big[\tau>T|\tau>t\big]
\]
\onslide<6->\[
Q(t,T)=\invisible<7->{\mathbbm{E}\big[}\color<7>{red}{{e^{-\int_{t}^{T}\lambda(s) ds}}}\invisible<7->{\big]}
\]
\end{overprint}
\onslide<3->Le modele à intensité suppose que la probabilité défaut suit formule suivante:\\ 
\begin{overprint}
\onslide<3>\[
\mathbb{P}\big[t < \tau \leq t+dt|\tau > t \big]=\lambda(t) dt
\]
\onslide<4>\[
\frac{\mathbb{P}\big[\tau \leq t+dt\big]-\mathbb{P}\big[\tau \leq t\big]}{1-\mathbb{P}\big[\tau \leq t \big]}=\lambda(t) dt
\]
\onslide<5-7>\[
dQ(0,t)=-\lambda(t) Q(0,t)dt
\]
\onslide<8>\[
dQ(t)=-\lambda(t) Q(t)dt
\]
\end{overprint}
\onslide<7->Par la suite on suppose que $\lambda(t)$ est déterministe.\\
\vspace{0.5cm}
\onslide<8>Dans le cas $t=0$ on simplfie la notation $Q(T)=Q(0,T)$
\end{frame}


\begin{frame}{Jambe de Protection}
\begin{overprint}
\onslide<1>\begin{align*}
PV_{\text{Protection}}&=N \times (1-Rec)\times\mathbb{E}[e^{-\int_{t}^{\tau}r_s ds}\mathbbm{1}_{\tau<T}]\\
\end{align*}
\onslide<2>\begin{align*}
PV_{\text{Protection}}&=N \times (1-Rec)\times\mathbb{E}[e^{-\int_{t}^{\tau}r_s ds}\mathbbm{1}_{\tau<T}]\\
&=-N \times (1-Rec) \times \int_{t}^{T} B(t,s) \frac{dQ(t,s)}{ds}ds\\
\end{align*}
\onslide<3>\begin{align*}
PV_{\text{Protection}}&=N \times (1-Rec)\times\mathbb{E}[e^{-\int_{t}^{\tau}r_s ds}\mathbbm{1}_{\tau<T}]\\
&=N \times (1-Rec) \times \lambda \times \int_{t}^{T} e^{-(\lambda+r) \times (s-t)}ds\\
\end{align*}
\onslide<4>\begin{align*}
PV_{\text{Protection}}&=N \times (1-Rec)\times\mathbb{E}[e^{-\int_{t}^{\tau}r_s ds}\mathbbm{1}_{\tau<T}]\\
&=\color{red}{N \times (1-Rec) \times \lambda \times\frac{ 1-e^{-(\lambda+r) \times (T-t)}}{\lambda+r}}\\
\end{align*}
\end{overprint}
\onslide<3->On peut terminer le calcul dans le cas où le taux d'intérêt continu et l'intensité de défaut sont constants:
\begin{align*}
B(t)&=e^{-r \times t}\\
Q(t)&=e^{-\lambda \times t}
\end{align*}
\end{frame}

\begin{frame}{Jambe de Prime "seule"}
\begin{overprint}
\onslide<1-2>\begin{align*}
PV_{\text{Premium Only}}&=N \times C \times \sum_{i=1}^{n}\delta_i \times\mathbb{E}[e^{-\int_{t}^{\tau}r_s ds}\mathbbm{1}_{T_i<\tau}]\\
\visible<2>{&=N \times C \times \sum_{i=1}^{n}\delta_i \times \underbrace{B(t,T_i) \times Q(t,T_i)}_{\text{Zéro Coupon risqué}}}
\end{align*}
\onslide<3>\begin{align*}
PV_{\text{Premium Only}}&=N \times C \times \sum_{i=1}^{n}\delta_i \times\mathbb{E}[e^{-\int_{t}^{\tau}r_s ds}\mathbbm{1}_{T_i<\tau}]\\
&=\color{red}{N \times C \times \sum_{i=1}^{n}\delta_i \times e^{-(r+\lambda) \times (T_i-t)}}
\end{align*}

\end{overprint}
\onslide<3->On peut terminer le calcul dans le cas où le taux d'intérêt continu et l'intensité de défaut sont constants:
\begin{align*}
B(t)&=e^{-r \times t}\\
Q(t)&=e^{-\lambda \times t}
\end{align*}
\end{frame}

\begin{frame}{Jame de coupon courru}
\begin{overprint}
\onslide<1-2>\begin{align*}
PV_{\text{Accrued Interest}}&=N \times C \times \sum_{i=1}^{n} \times\mathbb{E}[\text{\small DCC}(T_{i-1},s)e^{-\int_{t}^{\tau}r_s ds}\mathbbm{1}_{T_{i-1}< \tau \leq T_i}]\\
\visible<2>{&=N \times C \times \sum_{i=1}^{n} \times \int_{T_{i-1}}^{T_i} (s-T_{i-1})B(t,s) \frac{dQ(t,s)}{ds}ds}
\end{align*}
\onslide<3>\begin{align*}
PV_{\text{Accrued Interest}}&=N \times C \times \sum_{i=1}^{n} \times\mathbb{E}[\text{\small DCC}(T_{i-1},s)e^{-\int_{t}^{\tau}r_s ds}\mathbbm{1}_{T_{i-1}< \tau \leq T_i}]\\
&=N \times C \times \sum_{i=1}^{n} \times \lambda \times \int_{T_{i-1}}^{T_i} (s-T_{i-1})e^{-(r+\lambda) \times (s-t)}ds
\end{align*}
\onslide<4>\[
PV_{\text{Accrued Interest}}=N \times C \times \sum_{i=1}^{n} \times\mathbb{E}[\text{\small DCC}(T_{i-1},s)e^{-\int_{t}^{\tau}r_s ds}\mathbbm{1}_{T_{i-1}< \tau \leq T_i}]
\]
\begin{align*}
=\color{red}{N \times C \times \sum_{i=1}^{n} \times \lambda \times} &\color{red}{\big[ \frac{e^{-(r+\lambda) \times (T_{i-1}-t)}-e^{-(r+\lambda) \times (T_{i}-t)}}{(r+\lambda)^2}}\\
&\color{red}{- \delta_i \times \frac{e^{-(r+\lambda) \times (T_{i}-t)}}{r+\lambda}  \big]}\\
\end{align*}
\end{overprint}
\onslide<3->On peut terminer le calcul dans le cas où le taux d'intérêt continu et l'intensité de défaut sont constants:
\begin{align*}
B(t)&=e^{-r \times t}\\
Q(t)&=e^{-\lambda \times t}
\end{align*}
\end{frame}

\begin{frame}{Exercice}
Considérons un CDS \textbf{5 ans} qui démarre aujourd'hui et qui paie une prime \textbf{trimestrielle}.\\
\vspace{0.5cm}
On suppose les données de marché suivante:\\
\begin{center}
\begin{tabular}{|l|l|}
\hline
Taux d'intérêt continu&\textbf{1\%}\\
Intensité de défaut&\textbf{3\%}\\
Recovery&\textbf{40\%}\\
\hline
\end{tabular}
\end{center}
Quelle est la prime qui rend la structure au pair?\\
Pour un nominal de \textbf{10 millions d'euros}, quelles sont les valeurs:\\
\begin{itemize}
\item de la Jambe de protection ?\\
\item de la Jambe de prime ? \\
\item de la Jambe de coupon courru ?\\
\end{itemize}
\end{frame}

\begin{frame}{CDS Big bang}
Une réforme du marché des CDS a eu lieu en Avril 2009:
\begin{itemize}
\item \textbf{Fixed Coupon:} 
\item \textbf{IMM Dates:}
\item  
\end{itemize}
\end{frame}

\begin{frame}{Asset Swap}

\tikzset{
block/.style={
  rectangle,
  draw,
  text width=5.5em,
  text centered,
  rounded corners,
  minimum height=4em},
line/.style={draw},
edge/.style={draw}
}

\begin{tikzpicture}
\node (wa) [wa]  {System Combination};
\path (wa.west)+(-3.2,1.5) node (asr1) [sensor] {$ASR_1$};
\path (wa.west)+(-3.2,0.5) node (asr2)[sensor] {$ASR_2$};
    \path (wa.west)+(-3.2,-1.0) node (dots)[ann] {$\vdots$}; 
    \path (wa.west)+(-3.2,-2.0) node (asr3)[sensor] {$ASR_N$};    

\end{tikzpicture}
\end{frame}

\end{document}
