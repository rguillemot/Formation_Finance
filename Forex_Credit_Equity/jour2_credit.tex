\documentclass{beamer}
\usepackage[english,francais]{babel}
\usepackage[utf8]{inputenc}
\usepackage{multicol}
\usepackage{bm}
\usepackage{tikz}
\usetikzlibrary{positioning}

\usepackage{graphicx}
\usepackage{multirow}
\graphicspath{{./Forex_Credit_Equity/}}

\newcommand{\FIG}[3]{\includegraphics<#1>[width=#2]{#3}}
\newcommand{\FIGSCALE}[3]{\includegraphics<#1>[resolution=72dpi]{#3}}
\newcommand{\tikzmark}[1]{\tikz[overlay,remember picture] \node (#1) {};}

\usetheme{Warsaw}
\title[Produits dérivés actions,change et credit]{Produits dérivés de crédit}
\author{Richard Guillemot}
\institute{DIFIQ}
\date{22 Avril 2014}

\begin{document}

\begin{frame}
\titlepage
\end{frame}

\begin{frame}{Le Crédit}

\begin{exampleblock}{}
  {\large ``Un crédit est une mise à disposition d'argent sous forme de prêt, consentie par un \textcolor{red}{\textbf{créancier (créditeur, prêteur)}} à \textcolor{blue}{\textbf{un débiteur (emprunteur)}}.''}
  \vskip5mm
  \hspace*\fill{\small Wikipedia}
\end{exampleblock}
\vspace{0.5cm}
\visible<2->{Le \textbf{Risque de Crédit} fait référence à l'incapacité du \textcolor{red}{débiteur} de remplir ses engagements (le remboursement du capital ou le paiement des intérêts) totalement ou en partie. On dit alors que ce dernier fait défaut.\\
Le \textbf{Risque de Crédit} est intégralement porté par \textcolor{blue}{le créancier}.} 
\end{frame}


\begin{frame}{Vocabulaire}
\begin{center}
\begin{tabular}{|l|l|}
\hline
\textbf{Français} & \textbf{Anglais} \\
\hline
\hline
Prêt, Crédit, Obligation&\visible<2->{Loan, Credit, Bond}\\ 
\hline
Créditeur, Prêteur&\visible<3->{Creditor, Lender}\\ 
\hline
Créancier, Emprunteur&\visible<4->{Debtor, Borrower}\\ 
\hline
(Qualité de) crédit&\visible<5->{Creditworthiness}\\
\hline
Défaut, Insolvabilité, Faillite&\visible<6->{Default, Insolvency, Bankruptcy}\\
\hline
Capital prêté, Intérêts & \visible<7->{Princpal, Interests}\\
\hline
Remboursement & \visible<8->{Repayment}\\
\hline
Prêt hypothécaire ou garanti & \visible<9->{Mortgage or secured load}\\
\hline
Saisie&\visible<10->{Foreclosure}\\
\hline
\hline
\end{tabular}
\end{center}
\end{frame}

\begin{frame}{Struture du capital (ou passif) d'une entreprise.}
\begin{overprint}
\begin{tabular}{|c|lccc}
\cline{1-1}
\textbf{Capital}\\
\cline{1-1}
\cline{1-1}
Secured Debt& \multirow{4}{*}{$\left.\rule{0cm}{1cm}\right\}{\text{Dettes}}$}&\\
Senior Debt&\\
Subordinate Debt&\\
Mezzanine Debt&\\
\cline{1-1}
& \multirow{3}{*}{$\left.\rule{0cm}{0.75cm}\right\}{\text{Fonds propres}}$}&\\
Equity\\
\\
\cline{1-1}
\end{tabular}
\onslide<5>
\vspace{0.5cm}
Ratio à respecter pour une banque $\frac{\text{Fonds Propres}}{\text{Dettes}}\geq 8\%$
%\begin{tikzpicture}
%\draw (1,1) -- (2,2)
%\end{tikzpicture}
\end{overprint}
\begin{tikzpicture}[remember picture,overlay]
\draw<2> [<-,black, ultra thick] (8,1.5) node[yshift=-0.2cm]{\textbf{Première perte}} -- (8,4.4) node[yshift=0.2cm] {\textbf{Dernière perte}};
\draw<3> [->,black, ultra thick] (8,1.5) node[yshift=-0.2cm]{\textbf{Risque Maximum}} -- (8,4.4) node[yshift=0.2cm] {\textbf{Risque Minimum}};
\draw<4-> [->,black, ultra thick] (8,1.5) node[yshift=-0.2cm]{\textbf{Intérêts élevés}} -- (8,4.4) node[yshift=0.2cm] {\textbf{Intérêts faibles}};
\end{tikzpicture}
\end{frame}
\begin{frame}{Risque de crédit}
Comment atténuer le risque de crédit ? \\
\visible<2->{En anglais : How to mitigate the credit risk ?}\\
\begin{enumerate}
\item \visible<3->{\textbf{Valorisation basée sur la qualité de crédit} : Introduction d'une marge (ou spread) de crédit.}
\item \visible<4->{\textbf{Conventions de crédit (Loan covenants)}: clauses restrictives pour le prêteur}.
\item \visible<5->{\textbf{Limtes de Risque}}
\item \visible<6->{\textbf{Diversification}}
\item \visible<7->{\textbf{Dépot de garantié}: Hypothèque, Collatéral, Haircut}
\item \visible<8->{\textbf{Securitization}}
\item \visible<9->{\textbf{Assurance ou \textcolor{red}{dérivés de crédits}}}
\end{enumerate}
\end{frame}

\end{document}
