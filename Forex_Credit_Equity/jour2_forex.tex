\documentclass{beamer}
\usepackage[english,francais]{babel}
\usepackage[utf8]{inputenc}
\usepackage{multicol}
\usepackage{bm}

\usepackage{graphicx}
\graphicspath{{./Forex_Credit_Equity/}}

\newcommand{\FIG}[3]{\includegraphics<#1>[width=#2]{#3}}

\newcommand{\FIGSCALE}[3]{\includegraphics<#1>[resolution=72dpi]{#3}}

\usetheme{Warsaw}
\title[Produits dérivés actions,change et credit]{Produits dérivés de change}
\author{Richard Guillemot}
\institute{DIFIQ}
\date{22 Avril 2014}

\begin{document}

\begin{frame}
\titlepage
\end{frame}

\begin{frame}{Zero delta straddle}
Un \textbf{zéro delta straddle} EUR/USD est pour un même strike ($K^{ATM}$) et un même nominal:
\begin{itemize}
\item l'achat d'un call EUR
\item et l'achat d'un put EUR.
\end{itemize}
La delta du portefeuille doit être nul.
\begin{figure}
\begin{center}
\FIG{1}{11cm}{figures/zero-delta-straddle-0.png}
\FIG{2}{11cm}{figures/zero-delta-straddle-1.png}
\FIG{3}{11cm}{figures/zero-delta-straddle-2.png}
\FIG{4}{11cm}{figures/zero-delta-straddle-3.png}
\FIG{5}{11cm}{figures/zero-delta-straddle-4.png}
\FIG{6}{11cm}{figures/zero-delta-straddle-5.png}
\end{center}
\end{figure}
\end{frame}

\begin{frame}{25\% delta risk reversal}
Un \textbf{25\% delta risk reversal} EUR/USD est pour un même nominal: 
\begin{itemize}
\item l'achat d'un call EUR de delta égal à 25\% de strike $K^{25Call}$
\item et la vente d'un put EUR de delta égal à -25\% de strike $K^{25Put}$.
\end{itemize}
\begin{figure}
\begin{center}
\FIG{1}{10cm}{figures/risk-reversal-0.png}
\FIG{2}{10cm}{figures/risk-reversal-1.png}
\FIG{3}{10cm}{figures/risk-reversal-2.png}
\end{center}
\end{figure}
\end{frame}

\begin{frame}{25\% delta Butterfly}
Un \textbf{25\% delta Butterfly} est pour un même nominal:
\begin{itemize}
\item l'achat d'un call EUR de strike $K^{25Call}$
\item l'achat d'un call EUR de strike $K^{25Put}$
\item et la vente de 2 calls EUR de strike $K^{ATM}$.
\end{itemize}
\FIG{1}{10cm}{figures/butterfly-0.png}
\FIG{2}{10cm}{figures/butterfly-1.png}
\FIG{3}{10cm}{figures/butterfly-2.png}
\end{frame}

\begin{frame}{Cotation du smile de change}
Les différentes options de change ne sont pas cotées en prix mais en volatilité.\\
\vspace{0.5cm}

\begin{tabular}{|l|l|l|}
\hline
&\textbf{Cotation}\\
\hline
\textbf{0\% delta straddle}&$\sigma^{ATM}$\\
\textbf{25\% delta risk reversal}&$RR^{25}=\sigma^{25Call}-\sigma^{25Put}$\\
\textbf{25\% delta Butterfly}&$BF^{25}=\sigma^{25Call}+\sigma^{25Put}-2 \times \sigma^{ATM}$\\
\hline
\end{tabular}
\vspace{0.5cm}
\\
Comment à partir des cotations de marché des différents produits reconstituer le smile de change?
\end{frame}

\begin{frame}{Cotation du smile de change}
\begin{itemize}
\item \textbf{Etape 1}: On calcule les 3 points de volatilité de change.
\begin{align*}
\sigma^{25Call}&=\sigma^{ATM}+BF^{25}+\frac{1}{2} RR^{25}\\
\sigma^{25Put}&=\sigma^{ATM}+BF^{25}-\frac{1}{2} RR^{25}\\
\end{align*}
\item \textbf{Etape 2}: On calcule les strikes à partir de la volatilité et du delta.
\end{itemize}
\end{frame}

\begin{frame}{Cotation du smile de change - Exercice}
Construire le smile de change 1 an à partir des données suivantes:

\begin{center}
\begin{tabular}{|l|l|l|l|}
\hline
\textbf{Maturité}&1 an&$\mathbf{\sigma^{ATM}}$&12\%\\
\textbf{EUR/USD}&1.3889&$\mathbf{RR^{25}}$&-2\%\\
$\mathbf{r^{USD}}$&0.3\%&$\mathbf{BF^{25}}$&1\%\\
$\mathbf{r^{EUR}}$&0.5\%&$\mathbf{RR^{10}}$&-4\%\\
\textbf{Basis EUR}&0.1\%&$\mathbf{BF{10}}$&4\%\\
\hline
\end{tabular}
\end{center}
\begin{figure}
\FIG{1}{7cm}{figures/fxopt-smile.png}
\end{figure}
\end{frame}

\begin{frame}{Cotation du smile de change - Exercice}

\begin{center}
\begin{tabular}{|c|c|c|c|} 
\hline 
$\mathbf{K^{10Put}}$&\visible<3->{1.1201}&$\mathbf{\sigma^{10Put}}$&\visible<2->{18.0\%}\\
$\mathbf{K^{25Put}}$&\visible<3->{1.2755}&$\mathbf{\sigma^{25Put}}$&\visible<2->{14.0\%}\\
$\mathbf{K^{ATM}}$&\visible<3->{1.3975}&$\mathbf{\sigma^{ATM}}$&\visible<2->{12.0\%}\\
$\mathbf{K^{25Call}}$&\visible<3->{1.5148}&$\mathbf{\sigma^{25Call}}$&\visible<2->{12.0\%}\\
$\mathbf{K^{10Call}}$&\visible<3->{1.6760}&$\mathbf{\sigma^{10Call}}$&\visible<2->{14.0\%}\\
\hline 
\end{tabular}
\end{center} 
\end{frame}

\begin{frame}
\frametitle<1>{Interpolation linéaire}
\frametitle<2->{Interpolation spline cubique}
\begin{align*}
y&=q(x)=(1-t)\times y_{1}+t \times y_{2}\visible<2->{+\underbrace{t \times (1-t) \times (\mathbf{a} \times (1-t)+\mathbf{b} \times t)}_{\text{Termes quadratiques et cubiques}}}\\
t&=\frac{x-x_1}{x_2-x_1}
\end{align*}
\begin{figure}
\begin{center}
\FIG{1,2}{9cm}{figures/interpolation-0.png}
\FIG{3}{9cm}{figures/interpolation-1.png}
\FIG{4}{9cm}{figures/interpolation-2.png}
\end{center}
\end{figure}
\end{frame}

\begin{frame}
\frametitle{Interpolation spline cubique}
On peut facilement calculer les dérivés premières et secondes de $q$ aux points $x_1$ et $x_2$:\\
\vspace{0.5cm}
\begin{tabular}{r l r l r l}
$q'(x)$&$=\frac{\partial q}{\partial x}$&$q'(x_1)$&$=\frac{y_2-y_1}{x_2-x_1}+\frac{a}{x_2-x_1}$&$q'(x_2)$&$=\frac{y_2-y_1}{x_2-x_1}-\frac{b}{x_2-x_1}$\\
$q''(x)$&$=\frac{\partial^2 q}{\partial x^2}$&$q''(x_1)$&$=2\frac{b-2a}{(x_2-x_1)^2}$&$q''(x_2)$&$=2\frac{a-2b}{(x_2-x_1)^2}$\\
\end{tabular}
\\
\vspace{0.5cm}
On peut facilement calculer $a$ et $b$ en fonction des dérivées premières:
\begin{align*}
a&=\underbrace{q'(x_1)}_{k_1}(x_2-x_1)-(y_2-y_1)\\
b&=-\underbrace{q'(x_2)}_{k_2}(x_2-x_1)+(y_2-y_1)\\
\end{align*}
\end{frame}

\begin{frame}{Interpolation spline cubique}
\begin{figure}
\begin{center}
\FIG{1-}{9cm}{figures/interpolation-3.png}
\end{center}
\end{figure}
On considère $n$ tronçons de spline qui raccordent les $n+1$ points de $(x_0,y_0)$ à  $(x_n,y_n)$.
\end{frame}
\begin{frame}{Interpolation spline cubique}
\begin{center}
\begin{tabular}{c c c}
$k_{i-1}$&$k_i$&$k_{i+1}$\\
\hline
$\frac{y_{i}-y_{i-1}}{x_{i}-x_{i-1}}+\frac{a_i}{x_i-x_{i-1}}$&$\frac{y_{i+1}-y_{i}}{x_{i+1}+x_{i}}+\frac{a_{i+1}}{x_{i+1}-x_{i}}$&\\
&$\frac{y_{i}-y_{i-1}}{x_{i}-x_{i-1}}-\frac{b_i}{x_i-x_{i-1}}$&$\frac{y_{i+1}-y_{i}}{x_{i+1}+x_{i}}-\frac{b_{i+1}}{x_{i+1}-x_{i}}$\\
\hline
\end{tabular}
\end{center}
\begin{overprint}
\onslide<1>\[
q''(x_i)=2\frac{b_i-2a_i}{(x_{i}-x_{i-1})^2}=2\frac{a_{i+1}-2b_{i+1}}{(x_{i+1}-x_{i})^2}
\]
\onslide<2>\[
\underbrace{\frac{1}{x_i-x_{i-1}}k_{i-1}}_{a_{i,i-1}}+\underbrace{2\big[ \frac{1}{x_i-x_{i-1}} + \frac{1}{x_{i+1}-x_{i}}  \big]}_{a_{i,i}} k_{i} +\underbrace{\frac{1}{x_{i+1}-x_{i}}}_{a_{i,i+1}}k_{i+1}
\]
\[
=\underbrace{3\big[ \frac{y_{i}-y_{i-1}}{(x_i-x_{i-1})^2} + \frac{y_{i+1}-y_{i}}{(x_{i+1}-x_{i})^2} \big]}_{b_i}
\]
\end{overprint}
\end{frame}

\begin{frame}{Interpolation spline cubique}
Pour les points extrêmes on suppose que la dérivée seconde est nulle:\\
\begin{overprint}
\onslide<1>\begin{align*}
q''(x_0)&=\frac{b_1 -2 a_1}{(x_1-x_0)^2}=0\\
q''(x_n)&=\frac{a_n -2 b_n}{(x_n-x_{n-1})^2}=0\\
\end{align*}
\onslide<2>\begin{align*}
\underbrace{2\frac{1}{x_{1}-x_{0}}  }_{a_{0,0}} k_{0}&+\underbrace{\frac{1}{x_{1}-x_{0}}}_{a_{0,1}}k_{1}&&=\underbrace{3\frac{y_{1}-y_{0}}{(x_{1}-x_{0})^2}}_{b_0}\\
&\underbrace{\frac{1}{x_{n}-x_{n-1}}}_{a_{n,n-1}}k_{n-1}&+\underbrace{2\frac{1}{x_{1}-x_{0}}}_{a_{n,n}} k_{n-1} &=\underbrace{3\frac{y_{n}-y_{n-1}}{(x_{n}-x_{n-1})^2}}_{b_n}
\end{align*}
\end{overprint}
\end{frame}
\begin{frame}{Interpolation spline cubique}
Il nous faut maintenant résoudre le système linéaire précédemment défini où $K$ est l'inconnue:\\
\begin{overprint}
\onslide<1>\Huge
\[
A \times K = B
\]
\onslide<2>\footnotesize
\[
\underbrace{\left( 
\begin{array}{ccccccc}
a_{0,0} & a_{0,1} & \hdots & \hdots & \hdots & \hdots & 0 \\
a_{1,0} & \ddots & \ddots & \ddots  &  &  & \vdots  \\
\vdots & & a_{i,i-1} & a_{i,i} & a_{i,i+1} & & \vdots \\
\vdots & & & \ddots & \ddots & \ddots & a_{n-1,n}\\
0 & \hdots & \hdots & \hdots & \hdots & a_{n,n-1} & a_{n,n} 
\end{array}
\right)}_{\text{\huge A}}
\underbrace{\left(
\begin{array}{c}
k_0\\
\vdots\\
k_i\\
\vdots\\
k_n\\
\end{array}
\right)}_{\text{\huge K}}
=
\underbrace{\left(
\begin{array}{c}
b_0\\
\vdots\\
b_i\\
\vdots\\
b_n\\
\end{array}
\right)}_{\text{\huge B}}
\]
\end{overprint}
\end{frame}

\begin{frame}{Interpolation spline cubique - Exercice}
Construire un spline cubique à partir des points de smile calculés précédemment.
\begin{center}
\begin{tabular}{|c|c|c|c|}
\hline
&\textbf{k}&\textbf{a}&\textbf{b}\\
\hline
\visible<2->{0}&\visible<3->{-27.43\%}&\visible<2->{-}&\visible<2->{-}\\
\visible<2->{1}&\visible<3->{-22.38\%}&\visible<3->{-0.26\%}&\visible<3->{-0.52\%}\\
\visible<2->{2}&\visible<3->{-8.37\%}&\visible<3->{-0.73\%}&\visible<3->{-0.98\%}\\
\visible<2->{3}&\visible<3->{7.09\%}&\visible<3->{-0.98\%}&\visible<3->{-0.83\%}\\
\visible<2->{4}&\visible<3->{15.07\%}&\visible<3->{-0.86\%}&\visible<3->{-0.43\%}\\
\hline
\end{tabular}
\end{center}
\end{frame}

\begin{frame}{Sensibilités au change et aux paramètres de smile}
On peut calculer la sensibilité de chacun des 5 produits 
\begin{itemize}
\item \textbf{ZDS}: Zéro Delta Straddle
\item \textbf{RR25, RR10}: Risk Reversal 25 et 10 delta
\item \textbf{BF25, BF10}: Butterfly 25 et 10 delta
\end{itemize}
aux paramètres du smile:\\
\begin{overprint}
\onslide<1>\begin{center}
\begin{tabular}{|c|c|c|c|c|}
\hline
\textbf{Avec Smile}&\textbf{Delta FX}&\textbf{Sensi ATM}&\textbf{Sensi RR25}&\textbf{SensiBF25}\\
\hline
ZDS&5\%&0.56\%&0.00\%&0.00\%\\
RR25&38\%&0.03\%&0.39\%&0.01\%\\
BF25&-2\%&-0.16\%&0.00\%&0.35\%\\
RR10&10\%&-0.00\%&0.32\%&-0.09\%\\
BF10&-4\%&-0.39\%&-0.00\%&0.55\%\\
\hline
\end{tabular}
\end{center}
\onslide<2>\begin{center}
\begin{tabular}{|c|c|c|c|c|}
\hline
\textbf{Sans Smile}&\textbf{Delta FX}&\textbf{Sensi ATM}&\textbf{Sensi RR25}&\textbf{SensiBF25}\\
\hline
ZDS&0\%&0.55\%&0.00\%&0.00\%\\
RR25&50\%&0.00\%&0.44\%&-0.02\%\\
BF25&-0\%&-0.10\%&-0.01\%&0.39\%\\
RR10&20\%&0.00\%&0.48\%&-0.05\%\\
BF10&-0\%&-0.29\%&-0.00\%&0.73\%\\
\hline
\end{tabular}
\end{center}
\end{overprint}
\end{frame}

\begin{frame}{Forward FX Range}
Un \textbf{Forward FX Range} permet de garantir un taux de change $K$ dans le futur. Cette garantie est active à condition que le taux de change soit compris dans un intervalle (range) $[K_{Min},K_{Max}]$.\\
\vspace{0.5 cm}
L'objectif de ce produit est de proposer au client un taux de change forward bonifié en lui faisant courir un risque minimum.
\end{frame}

\begin{frame}{Exemple de Forward FX Range}
Considérons un Forward FX Range de maturité 1 an qui se désactive si le cours de l'EUR/USD s'écarte de plus de \textbf{30\%} de sa valeur spot à terme.\\
\vspace{0.5cm}
La banque propose d'améliorer de \textbf{30 pips} le taux de change forward classique dans le cas \textbf{d'une vente à terme d'euros} contre l'achat de dollars dans un an. 
\begin{center}
\FIG{1}{9cm}{figures/EURUSDPerf.png}
\end{center}
\end{frame}

\begin{frame}{Exercice}
\begin{enumerate}
\item Comment répliquer le Forward FX Range avec des options de change "vanille" ?
\item Déterminer la marge réalisée par la banque à partir des données de marché précédemment utilisées.
\item Quelle est la probabilité (risque neutre et non pas historique) que le produit se désactive en la défaveur du client. 
\item Calculer la couverture nécessaire en utilisant les produits de marché standard, Spot, Money Market, Basis, Risk Reversal Butterfly.
\end{enumerate}
\end{frame}
\end{document}
