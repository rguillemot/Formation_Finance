\documentclass{beamer}
\usepackage[english,francais]{babel}
\usepackage[utf8]{inputenc}
\usepackage{multicol}
\usepackage{color}

\usepackage{graphicx}
\graphicspath{{Derives_de_taux/}}


\newcommand{\FIG}[2]{\includegraphics[width=#1]{#2}}

\usetheme{Warsaw}
\title[Produits dérivés de taux]{Actualisation OIS. }
\author{Richard Guillemot}
\institute{Master IFMA}
\date{27 Février 2014}

\begin{document}

\begin{frame}
\titlepage
\begin{figure}[h]
\centering \FIG{5cm}{figures/UPMC_IFMA.jpg}
\end{figure}
\end{frame}

\begin{frame}{Unicité de la courbe des taux}
Nous supposons qu'il existe une \textbf{unique courbe} des taux dite \textbf{sans risque}, cela a 2 conséquences:
\begin{itemize}
\item On peut déduire les taux forward à partir des taux au comptant.
\[
L(t,T_1, T_2)=\frac{1}{\delta}\left(\frac{B(t,T_1)}{B(t,T_2)}-1\right)
\]
\item La valeur d'une jambe variable de swap est indépendante de sa fréquence.
\[
\sum_{i=1}^{n}\delta_i \times L(t,T_{i-1}, T_i) \times  B(t,T_i)=B(t,T_0)-B(t,T_n)
\]
\end{itemize}
\end{frame}

\begin{frame}{Risque de crédit et liquidité}
Cette méthode de valorisation avec une unique courbe des taux ignore 2 risques:
\begin{itemize}
\item le risque \textbf{de crédit}: L'emprunteur va t'il rembourser sa créance et payer les intérets au prêteur?
\item le risque \textbf{de liquidité}: Le prêteur va t'il prêter à un emprunteur pour la maturité demandée?
\end{itemize}
\end{frame}

\begin{frame}{Collatéralisation}
$V_t$ le prix d'un actif collatéralisé.\\
\vspace{0.5cm}
Les flux d'un actif collatéralisé à la date $t+dt$:\\
\vspace{0.5cm}
\begin{tabular}{l|r}
On rembourse le collatéral livré en $t$. & $-V_t$ \\
On paie les intérêts au taux de collatéral $c$.& $-c V_t dt$\\
On reçoit la nouvelle valeur du collatéral.& $V_{t+dt}$ \\
\end{tabular}
\vspace{0.5cm}\\
Le flux résultant : $dV_t - c \times V_t \times dt$
\end{frame}

\begin{frame}{Collatéralisation et actualisation OIS}
Soient les prix de 2 actifs sous la probabilité historique $P$:
\[
\begin{split}
dV^1_t&=\mu_1 V^1_t dt + \sigma_1 V^1_t dW_t\\
dV^2_t&=\mu_2 V^2_t dt + \sigma_2 V^2_t dW_t
\end{split}
\]
On constitue un portefeuille en vue de neutraliser l'aléa. Il génère le flux suivant en $t+dt$:
\[
\sigma_2 V_2 \times (dV^1_t - c V^1_t dt)-\sigma_1 V_1 \times (dV^2_t - c V^2_t dt)
\]
En l'absence d'opportunité d'arbitrage, un tel portefeuille a une valeur nulle, par conséquent son drift est aussi nul:
\[
\lambda=\frac{\mu_1-c}{\sigma_1}=\frac{\mu_2-c}{\sigma_2}
\]
\end{frame}

\begin{frame}{Collatéralisation et actualisation OIS}
On définit une nouvelle probabilité dite risque-neutre $Q$:\\
\[
d\widetilde{W}_t=dW_t+\lambda dt
\]
Sous cette probabilité les actifs on un drift égal au taux de collatéral:\\
\[
\begin{split}
dV^1_t&=c V^1_t dt + \sigma_1 V^1_t d\widetilde{W}_t\\
dV^2_t&=c V^2_t dt + \sigma_2 V^2_t d\widetilde{W}_t
\end{split}
\]
Il nous faut donc actualiser les flux des actifs collatéralisés au taux de collatéral $c$.

\end{frame}

\begin{frame}{Les produits de liquidité}
Sur le marché on peut traiter 2 catégories de produits de liquidité:
\begin{itemize}
\item \textbf{les swaps à taux fixe}: on distingue les différentes cotations de ce swap $K^f$ suivant la fréquence $f$ de la jambe variable.\\
\item \textbf{les swaps de basis}: ces produits échangent des jambes variables de fréquences différentes. $m^{f_1-f_2}$ est la marge à ajouter à la jambe de la fréquence la plus faible qui rend le swap nul.\\ 
\end{itemize}
\small
\begin{center}
\begin{tabular}{|c|c|c|c|c|c|c|}
\hline
Jambe&Fixe&OIS&1M&3M&6M&12M\\
\hline
Fixe&&$K^{OIS}$&$K^{1M}$&$K^{3M}$&$K^{6M}$&$K^{12M}$\\
\hline
OIS&&&$m^{OIS-1M}$&$m^{OIS-3M}$&$m^{OIS-6M}$&$m^{OIS-12M}$\\
1M&&&&$m^{1M-3M}$&$m^{1M-6M}$&$m^{1M-12M}$\\
3M&&&&&$m^{3M-6M}$&$m^{3M-12M}$\\
6M&&&&&&$m^{6M-12M}$\\
\hline
\end{tabular}
\end{center}
\end{frame}

\begin{frame}{Les produits de liquidité}
\begin{figure}[h]
\vspace{2mm}
\FIG{12cm}{figures/basis.png} 
\vspace{1mm}
\end{figure}
\end{frame}

\begin{frame}{Les produits de liquidité}
Toutes ces cotations sont liées les unes aux autres:\\
\[
K^m-K^n=\frac{LVL^m(T^m_0,T^m_m)}{LVL(T_0,T_n)} m^{n,m}
\]
On peut approcher cette relation:
\[
K^m-K^n \simeq m^{n,m}
\]
On en déduit une "sorte" de relation de Chasles pour les marges de basis:
\[
m^{n,m}+m^{m,p} \simeq m^{n,p}
\]
$m$,$n$ et $p$ correspondent à trois fréquences différentes.
\end{frame}

\begin{frame}{Algorithme de calage \textbf{"des"} courbes.}
Il nous faut construire un algorithme, 
\begin{itemize}
\item où tous les flux sont actualisés au taux de collatéral standard, c'est à dire l'OIS de la devise,
\item consistant avec les cotations de tous les produits de liquidité.
\end{itemize}
\vspace{0.5cm}
Algorithme en 2 étapes:
\begin{itemize}
\item \textbf{Etape 1}: On calibre la courbe OIS avec l'algorithme classique, en effet une jambe variable OIS utilise la même courbe pou calculer les taux forward et actualiser les flux.
\item \textbf{Etape 2}: On calibre une courbe pour chacune des fréquences 1M, 3M, 6M et 12M afin de reproduire les cotations de marchés $K^f$.
\end{itemize}
\end{frame}

\begin{frame}{Exemple de calage}
\begin{center}
\begin{tabular}{|l|c|c|}
\hline
Produit & Taux (\%) & Symbole\\
\hline
EURIBOR 6M & 1\% & $R_1$ \\
EURIBOR 12M & 1.5\% & $R_2$ \\
EONIA vs BOR 6M & 50bps & $m$\\
EONIA vs BOR 12M & 50bps & $m$\\
SWAP 12M vs 6M & 1.5\% & $R_2$ \\
FRA 6M dans 6M & \textbf{??} & F \\
\hline
\end{tabular}
\end{center}
Il nous faut résoudre l'équation suivante:
\small
\[
\delta_1 \times B^{OIS}(t,T_{6M}) \times R_1 + \delta_2 \times B^{OIS}(t,T_{12M}) \times F = (\delta_1 + \delta_2) \times R_2 \times B^{OIS}(t,T_{6M})
\]
\end{frame}

\begin{frame}{Exemple de calage}
\small
Les facteurs d'actualisation OIS:
\begin{center}
\begin{tabular}{r l}
$B^{OIS}(t,T_{6M})=$ & $\frac{1}{1+\delta_1 i(R_1-m)}$ \\
$B^{OIS}(t,T_{12M})=$ & $\frac{1}{1+(\delta_1 + \delta_2) (R_2-m)}$ \\
\end{tabular}
\end{center}
Le taux forward classique:
\[
L(t,T_{6M},T_{12M})=\frac{1}{\delta_2}\big(\frac{1+(\delta_1+\delta_2) R_2}{1+\delta_1 R_1}-1\big)
\]
La solution de l'équation est la suivante:
\[
F=L^{OIS}(t,T_1,T_2)+m \big[ 1 - \delta_1 L^{OIS}(t,T_{6M},T_{12M}) \big] \simeq L(t,T_{6M},T_{12M})
\]
avec:
\[
L^{OIS}(t,T_{6M},T_{12M})=\frac{1}{\delta_2}\big(\frac{1+(\delta_1+\delta_2) (R_2-m)}{1+\delta_1 (R_1-m)}-1\big)
\]
\end{frame}

\begin{frame}{Exemple de calage}
\[
\text{Ajustement}=F-L(t,T_{6M},T_{12M})
\]
\begin{center}
\begin{tabular}{|c|c|c|}  
\hline  
OIS-BOR & $F$ & Ajustement \\ 
\hline  
0 bps & 1.9900\%&0.00bps\\ 
10 bps & 1.9905\%&0.05bps\\ 
20 bps & 1.9910\%&0.10bps\\ 
30 bps & 1.9915\%&0.15bps\\ 
40 bps & 1.9920\%&0.20bps\\ 
\textbf{\color{red}50 bps} & \textbf{\color{red}1.9925\%}&\textbf{\color{red}0.25bps}\\ 
\hline 
\end{tabular}
\end{center}
\end{frame}

\begin{frame}{Gestion dans un environement multi courbe}
\small
Couvrir un swap client par un swap de marché:\\
\begin{center}
\begin{tabular}{|c|c|c|c|c|}
\hline
\textbf{Marché} &20 ans&Receveur&Non Collatéralisé&100 Mios EUR \\ 
\hline
\textbf{Client} &20 ans&Payeur&Collatéralisé&\textbf{??} \\ 
\hline
\end{tabular}
\end{center}
Le marché:\\
\begin{center}
\begin{tabular}{|c|c|}
\hline
\textbf{EURIBOR} & 2\% \\
\hline
\textbf{OIS vs EURIBOR} & 50 bps \\
\hline
\end{tabular}
\end{center}
Sensibilités et convexités:
\begin{center}
\begin{tabular}{|c|c|c|c|}
\hline
&\textbf{PV}&\textbf{Sensi}& \textbf{Convexité}  \\ 
\hline
\textbf{Marché} & 0&-172 kEUR/bp&-333 EUR/bp/bp \\ 
\hline
\textbf{Client} & 2&166 kEUR/bp&315 EUR/bp/bp \\ 
\hline
\end{tabular}
\end{center}

\end{frame}
\begin{frame}{Gestion dans un environement multi courbe}
Il faudra traiter 96.5 millions d'euros de swap de marché:\\
\[
\frac{166}{172} \times 100 = 96.5 \text{ millions d'euros}
\]  
Le portefeuille a une convexité négative de 6 euros bp$^2$:\\
\[
- \frac{166}{172} \times 333 + 315 = -6 \text{ euro bp}^2 
\]
Un mouvement de taux à la hausse ou à la baisse de 100 points de base entraine une perte de 30 kEUR:\\
\[
- \frac{1}{2} \times 6 \times 10000 = -30\text{ kEUR}  
\]
\end{frame}


\end{document}
