\documentclass{beamer}
\usepackage[english,francais]{babel}
\usepackage[utf8]{inputenc}
\usepackage{multicol}
\usepackage{color}

\usepackage{graphicx}
\graphicspath{{Derives_de_taux/}}

\newcommand{\FIG}[2]{\includegraphics[width=#1]{#2}}

\usetheme{Warsaw}
\title[Produits dérivés de taux]{Options de taux}
\author{Antonin Chaix - Richard Guillemot}
\institute{Master IFMA}
\date{17 Février 2017}

\begin{document}

\begin{frame}
\titlepage
\begin{figure}[h]
\centering \FIG{5cm}{figures/UPMC_IFMA.jpg}
\end{figure}

\end{frame}

\begin{frame}{Quizz}
Soit une courbe de taux constante égale à 2\% (taux actuariel à composition annuelle).\\
\vspace{0.5cm}
Quel est le nominal et le sens d'un swap de marché, de maturité 20 ans dont la sensibilité est égale à 328 kEUR/bp?\\
\vspace{0.5cm}
a) 50 Mios d'euros payeur de taux fixe.\\
b) 100 Mios d'euros receveur de taux fixe.\\
c) 150 Mios d'euros receveur de taux fixe.\\
d) 200 Mios d'euros payeur de taux fixe.\\
\end{frame}


\begin{frame}{Quizz}
Soit une courbe de taux constante égale à 2\% (taux actuariel à composition annuelle).\\
\vspace{0.5cm}
Quel est le nominal et le sens d'un swap de marché, de maturité 20 ans dont la sensibilité est égale à 328 kEUR/bp?\\
\vspace{0.5cm}
a) 50 Mios d'euros payeur de taux fixe.\textbf{\color{red}FAUX}\\
b) 100 Mios d'euros receveur de taux fixe.\textbf{\color{red}FAUX}\\
c) 150 Mios d'euros receveur de taux fixe.\textbf{\color{red}FAUX}\\
d) 200 Mios d'euros payeur de taux fixe.\textbf{\color{green}VRAI}\\
\end{frame}


\begin{frame}{Quizz}
Soit une courbe de taux constante égale à 2\% (taux actuariel à composition annuelle).\\
\vspace{0.5cm}
Quel est le tenor d'un swap de marché payeur de taux fixe de nominal 100 Mios d'euros dont la sensibilité est égale à 128 kEUR/bp?\\
\vspace{0.5cm}
a) 10 ans.\\
b) 15 ans.\\
c) 20 ans.\\
d) 30 ans.\\
\end{frame}


\begin{frame}{Quizz}
Soit une courbe de taux constante égale à 2\% (taux actuariel à composition annuelle).\\
\vspace{0.5cm}
Quel est le tenor d'un swap de marché payeur de taux fixe de nominal 100 Mios d'euros dont la sensibilité est égale à 128 kEUR/bp?\\
\vspace{0.5cm}
a) 10 ans.\textbf{\color{red}FAUX}\\
b) 15 ans.\textbf{\color{green}VRAI}\\
c) 20 ans.\textbf{\color{red}FAUX}\\
d) 30 ans.\textbf{\color{red}FAUX}\\
\end{frame}

\begin{frame}{Quizz}
Soit un swap dont la sensibilité est de -90kEUR/bp et la convexité est de 94 EUR bp/bp.\\
\vspace{0.5cm}
Quelle est sa sensibilité si les taux augmentent de 100bp ?\\
a) -81 kEUR/bp.\\
b) -99 kEUR/bp.\\
c) -85 kEUR/bp.\\
d) -95 kEUR/bp.\\
\end{frame}


\begin{frame}{Quizz}
Soit un swap dont la sensibilité est de -90kEUR/bp et la convexité est de 94 EUR bp/bp.\\
\vspace{0.5cm}
Quelle est sa sensibilité si les taux augmentent de 100bp ?\\
a) -81 kEUR/bp.\textbf{\color{green}VRAI}\\
b) -99 kEUR/bp.\textbf{\color{red}FAUX}\\
c) -85 kEUR/bp.\textbf{\color{red}FAUX}\\
d) -95 kEUR/bp.\textbf{\color{red}FAUX}\\
\end{frame}

\begin{frame}{Quizz}
Quel est le produit qui apporte le plus de convexité/concavité de taux (pour une maturité, un sens et un nominal donné) ?\\
\vspace{0.5cm}
a) Une jambe variable de swap.\\
b) Une jambe fixe de swap.\\
c) Un swap.\\
d) Une obligation à taux variable.\\
\end{frame}


\begin{frame}{Quizz}
Quel est le produit qui apporte le plus de convexité/concavité de taux (pour une maturité, un sens et un nominal donné) ?\\
\vspace{0.5cm}
a) Une jambe variable de swap.\textbf{\color{red}FAUX}\\
b) Une jambe fixe de swap.\textbf{\color{red}FAUX}\\
c) Un swap.\textbf{\color{green}VRAI}\\
d) Une obligation à taux variable.\textbf{\color{red}FAUX}\\
\end{frame}

\begin{frame}{Taux Stochastiques}
En l'absence d'opportunité d'arbitrage il existe une probabilité $Q$ où la valeur actualisée d'un actif financier est une martingale, c'est à dire que $M_t$, la valeur actualisée d'un actif financier à la date $t$, est une $Q$-martingale:
\[
	M_t = e^{-\int_0^t r_s ds}X_t
\]
Par conséquent la valeur actuelle de l'actif est:
\[
	X_t=\mathbb{E}^Q_t[\color{red} e^{-\int_0^T r_sds} \color{black} X_T]
\]
Lorsque \color{red}\textbf{les taux sont stochastiques}\color{black}, on ne peut pas sortir le facteur d'actualisation de l'espérance!!!
\end{frame}

\begin{frame}{Changement de numéraire}
Un numéraire est un actif financier $N_t$:
\begin{itemize}
\item ne distribuant pas de dividendes.
\item toujours strictement positif.
\end{itemize}
\vspace{0.5cm}
$N_t$ est associé à une mesure de probabilité $Q^N$.\\
\vspace{0.5cm}
Pour tout actif X de marché de valeur $X_t$,\\
 $\frac{X_t}{N_t}$ est une $Q^N$-martingale donc:
\[
\frac{X_t}{N_t}=\mathbb{E}^{Q^N}_t\left[\frac{X_T}{N_T}\right]
\]
\vspace{0.5cm}
Par conséquent si on considère 2 numéraires $N_t$ et $N'_t$:\\
\[
X_t=N_t\mathbb{E}^{Q^N}_t\left[\frac{X_T}{N_T}\right]=N'_t\mathbb{E}^{Q^{N'}}_t\left[\frac{X_T}{N'_T}\right]
\]
\end{frame}

\begin{frame}{Les probabilités risque-neutre et forward-neutre}
\begin{itemize}
\item La probabilité \textbf{risque-neutre} $Q$ est associée au numéraire:
\[
N_t=e^{\int_0^t r_s ds }
\]
qui correspond à 1 euro capitalisé jusqu'en t de façon continue.
\vspace{0.5cm}\\
\item La probabilité \textbf{forward-neutre} $Q^T$ est associée au numéraire:
\[
N_t=B(t,T)=\mathbb{E}^Q_t[e^{-\int_0^T r_s ds}]
\]
qui correspond à recevoir 1 euro en T.
\end{itemize}

\end{frame}

\begin{frame}{Exemple "classique" de changement de numéraire}
Soit X un call sur un sous-jacent de valeur $S_t$ de maturité $T$. Il paie donc en $T$ le flux suivant $(S_T-K)_+$.\\
Sa valeur est en $t$ est:
\[
X_t=\mathbb{E}[e^{-\int_t^T r_s ds}(S_T-K)_+]
\]
Si l'on passe sous la probabilité forward-neutre $Q^T$:
\[
X_t=B(t,T)\mathbb{E}^{Q^T}\left[\frac{(S_T-K)_+}{B(T,T)}\right]
\]
On ne considère plus l'actif $S_t$ mais le forward $F_t=\frac{S_t}{B(t,T)}$  qui par construction est une martingale sous $Q^T$ (pas de drift dans la diffusion).\\
$F_t$ est la valeur de S exprimée dans le numéraire $B(t,T)$.
\end{frame}
\begin{frame}{Exemple "classique" de changement de numéraire}
On modélise alors $F_t$ de la façon suivante:
\[
\left\{
\begin{split}
&dF_t=\sigma\; F_t \;dW_t^{Q^T}\\
&\text{où } W^{Q^T}\text{est un mouvement brownien standard sous la mesure }Q^T
\end{split}
\right.
\]
On valorise alors l'option au moyen de la formule de Black \& Scholes :
\[
X_t=B(t,T)\;\text{\textbf{BS}}_\text{call}(T-t,K,F_t,\sigma)
\]
avec:
\[
\left\{
\begin{split}
&\text{\textbf{BS}}_\text{call}(\tau,K,F,\sigma)=F\mathcal{N}(d_1)-K \mathcal{N}(d_2)\\
&\text{\textbf{BS}}_\text{put}(\tau,K,F,\sigma)=K\mathcal{N}(-d_2)-F \mathcal{N}(-d_1)\\
&\mathcal{N} : \text{fonction de répartition de la loi normale centrée réduite}\\
&d_1=\frac{\ln\left(\frac{F}{K}\right)+\frac{1}{2}\sigma^2\tau}{\sigma\sqrt{\tau}}\\
&d_2=d_1-\sigma\sqrt{\tau}
\end{split}
\right.
\]
\end{frame}

\begin{frame}{Caps et Floors}
Considérons un cap sur EURIBOR 6M de strike $K$ qui démarre en $T_0$ et mature en $T_n$.
\begin{figure}[h]
\FIG{10cm}{figures/schema_cap.jpg}
\end{figure}
A chaque date $T_i$, le cap verse le flux suivant:
\[
\delta_i\; \max \big(L(T_{i -1}^f,T_{i -1},T_i) - K, 0\big)
\]
avec:
\begin {itemize}
\item $T_{i -1}^f$: la date de fixing de l'EURIBOR 6M qui démarre en $T_{i -1}$ et mature en $T_{i}$.
\item $\delta_i$: la fraction d'année exprimée dans la convention ACT 360.
\end {itemize}
\end{frame}

\begin{frame}{Caps et Floors}
Un cap est un panier de caplets. On peut valoriser un caplet ainsi:
\[
\textbf{PV}_\text{Caplet}^i(t)=\delta_i \mathbb{E}_t^{Q}\Big(e^{-\int_0^{T_i} r_s ds}\big(L(T_ {i-1}^f,T_{i-1},T_i)-K\big)^+\Big)
\]
en passant sous la mesure forward neutre $Q^{T_i}$:
\[
\textbf{PV}_\text{Caplet}^i(t)=\delta_i \;B(t,T_i)\;\mathbb{E}_t^{Q^{T_i}}\Big(\big(L(T_ {i-1}^f,T_{i-1},T_i)-K\big)^+\Big)
\]
Nous allons reproduire le raisonnement de l'exemple classique, en effet $L(T_ {i-1}^f,T_{i-1},T_i)$ est une martingale sous $Q^{T_i}$:
\[
L(t,T_{i-1},T_i)= \frac{B(t,T_{i-1})-B(t,T_i)}{B(t,T_i)}
\]
C'est la valeur d'un prêt forward qui démarre en $T_{i-1}$ et mature en $T_i$ exprimée dans le numéraire forward $T_i$.
\end{frame}

\begin{frame}{Caps et Floors}
On peut aussi réécrire la valeur d'un caplet ainsi:
\[
\textbf{PV}_\text{Caplet}^i(t)= \mathbb{E}_t^{Q}\Big(e^{-\int_0^{T_i} r_s ds}\big(\color{red}1+\delta_i\color{black}L(T_ {i-1}^f,T_{i-1},T_i)-\color{red}1+\delta_i\color{black}K\big)^+\Big)
\]
Nous passons sous la mesure forward neutre $Q^{T_{i-1}}$:\\
\[
\textbf{PV}_\text{Caplet}^i(t)= B(t,T_{i-1})\mathbb{E}_t^{Q^{T_{i-1}}}\Big(\big(1-\underbrace{\frac{1+\delta_iK}{1+\delta_iL(T_ {i-1}^f,T_{i-1},T_i)}}_{\text{Obligation zéro coupon de taux fixe K}}\big)^+\Big)
\]
On peut exprimer le caplet comme un "put" sur une obligation zéro coupon de taux fixe K, martingale sous la probabilité $Q^{T_{i-1}}$.
\end{frame}

\begin{frame}{Caps et Floors}
Si on suppose une dynamique log-normale sur le LIBOR :
\[
\left\{
\begin{split}
&dL(t,T_{i-1},T_i)=\sigma\;L(t,T_{i-1},T_i) \;dW_t^{Q^{T_i}}\\
&\text{où } W^{Q^{T_i}}\text{est un mouvement brownien standard sous la mesure }Q^{T_i}
\end{split}
\right.
\]

On peut exprimer la valeur actuelle du caplet au moyen de la formule de Black \& Scholes  :
\[
\textbf{PV}_\text{Caplet}^i(t)=\delta_i B(t,T_i) \;\text{\textbf{BS}}_\text{call}\big(T_{i-1}^f-t,K,L(t,T_{i-1},T_i),\sigma\big)
\]

\end{frame}

\begin{frame}{Swaption}
Soit l'échéancier d'un swap standard qui démarre en $T_0$ et mature en $T_n$\\
Le détenteur de la swaption associée payeuse de strike K est l'option d'entrer, à la date de maturité $T_f=T_0-2\text{ jours ouvrés}$,  dans ce swap sans frais quelque soit le niveau de marché.
\begin{figure}[h]
\FIG{10cm}{figures/schema_swaption.jpg}
\end{figure}
\end{frame}
\begin{frame}{Swaption}
On valorise la swaption de la façon suivante:
\[
\textbf{PV}_\text{Sw}^P(t)=B(t,T_f)\;\mathbb{E}_t^{Q^{T_f}}\Big(\big(\textbf{PV}_V(T_f)-\textbf{PV}_F(T_f)\big)^+\Big)
\]
que l'on peut réécrire en fonction du taux de swap et de l'annuité:
\[
\textbf{PV}_\text{Sw}^P(t)=B(t,T_f)\;\mathbb{E}_t^{Q^{T_f}}\Big(\text{LVL}(T_f,T_0,T_n)\;\big(S(T_f,T_0,T_n)-K\big)^+\Big)
\]
avec:
\[
\text{LVL}(t,T_0,T_n)=\sum_{i=1}^{n}\delta_i \;B(t,T_i)
\]
\[
S(t,T_0,T_n)= \frac{B(t,T_0)-B(t,T_n)}{\text{LVL}(t,T_0,T_n)}
\]
\end{frame}

\begin{frame}{Swaption}
On passe sous la mesure Level $Q^{\text{LVL}}$ aussi dite \textbf{swap-neutre}, la mesure du numéraire Level sous laquelle le taux de swap est naturellement martingale:
\[
\textbf{PV}_\text{Sw}^P(t)=\text{LVL}(t,T_0,T_n)\;\mathbb{E}_t^{Q^{\text{LVL}}}\Big(\big(S(T_f,T_0,T_n)-K\big)^+\Big)
\]
\\
On suppose une dynamique lorgnormale pour le taux de swap:
\[
\left\{
\begin{split}
&dS(t,T_0,T_n)=\sigma S(t,T_0,T_n) \;dW_t^{Q^{\text{LVL}}}\\
&\text{où } W^{Q^{\text{LVL}}}\text{est un mouvement brownien standard sous la mesure }Q^{\text{LVL}}
\end{split}
\right.
\]
ce qui nous donne:
\[
\textbf{PV}_\text{Sw}^P(t)=\text{LVL}(t,T_0,T_n) \;\text{\textbf{BS}}_\text{call}\big(T_f-t,K,S(t,T_{0},T_n),\sigma\big)
\]
\end{frame}

\begin{frame}{Swaption}
Sans changer de mesure, on peut réécrire la valeur d'une swaption:
\[
\textbf{PV}_\text{Sw}^P(t)=B(t,T_f)\;\mathbb{E}_t^{Q^{T_f}}\Big(\big(B(T_f,T_0)-\underbrace{B(T_f,T_n)-K\text{LVL}(T_f,T_0,T_n)}_{\text{Obligation de taux fixe K}}\big)^+\Big)
\]
La swaption peut être exprimée comme un put de strike $B(T_f,T_0)\simeq1$ sur une obligation de de taux fixe K.\\
\vspace{0.5cm}
Cette obligation est naturellement martingale sous la probabilité $Q^{T_f}$.
\end{frame}

\begin{frame}{Quantile de la loi normale - Sens de la volatilité}
Au bout d'un an un actif financer de volatilité $\sigma$ a plus d'\textbf{une} chance sur \textbf{deux} de s'être écartée de $\pm \sigma$ de sa valeur initiale.
\begin{center}
\begin{figure}[h]
\FIG{11cm}{figures/loinormale_quantile.png}
\end{figure}
\end{center}
\end{frame}

\begin{frame}{Le modèle normal}
La dynamique du taux est la suivante:
\[
dF_t=\sigma\; dW_t
\]
Son intégration est immédiate:
\[
F_t = F_0 + \sigma W_t
\]
On peut facilement calculer la valeur actuelle d'un call ou d'un put:
\[
\begin{split}
&\text{\textbf{N}}_\text{call}(\tau,K,F,\sigma)=\sigma\sqrt{\tau}\Big(d^+\mathcal{N}(d^+)+\mathcal{N}'(d^+)\Big)\\
&\text{\textbf{N}}_\text{put}(\tau,K,F,\sigma)=\sigma\sqrt{\tau}\Big(d^-\mathcal{N}(d^-)+\mathcal{N}'(d^-)\Big)\\
\end{split}
\]
avec :
\[
d^\pm = \pm \frac{F-K}{\sigma\sqrt{\tau}}
\]
\end{frame}

\begin{frame}{Le modèle normal}
Il produit un smile décroissant:
\begin{figure}[h]
\FIG{8cm}{figures/smile_normal.jpg}
\end{figure}
Le taux forward est 5\%. La volatilité normale $\sigma$ est égale à 0,90\%. 
\end{frame}

\begin{frame}{Le modèle normal}
\begin{figure}[h]
\FIG{11cm}{figures/lognormalvsnormal.png}
\end{figure}
\end{frame}


\begin{frame}{Le modèle Lognormal décalé}
La dynamique du taux est la suivante:
\[
dF_t=\sigma (F_t + d)dW_t
\]
On l'intègre facilement:
\[
F_t = (F_0+d)\exp\left(-\frac{1}{2}\sigma^2t+\sigma W_{t} \right)-d
\]
On peut facilement calculer la valeur d'un call ou d'un put en adaptant la formule de Black \& Scholes:
\[
\mathbf{SL}_\text{call/put}(\tau,K,F,\sigma,d)=\mathbf{BS}_\text{call/put}(\tau,K+d,F+d,\sigma)
\]
\end{frame}

\begin{frame}{Le modèle Lognormal décalé}
Le modèle lornormal décalé permet de contrôler la pente du smile gràce au paramètre de décalage:\\
\begin{figure}[h]
\FIG{7cm}{figures/smile_shifted_log.jpg}
\end{figure}
Le forward est égal à 5\%. Le paramètre $\sigma$ est calibré de telle sorte que la volatilité à la monnaie (strike 5\%) demeure égale à 18\%. 
\end{frame}

\begin{frame}{Le modèle SABR}
SABR est l'acronyme de Sigma-Alpha-Beta-Rho, le noms de ses paramètres.\\
Sa diffusion est la suivante:

\[
\left\{
\begin{split}
dF_t &= \sigma_t \;F_t^\beta dW_t^1\\
\frac{d\sigma_t}{\sigma_t} &= \alpha \; dW_t^2
\end{split}
\right.
\]

Le modèle SABR est défini par quatre paramètres:

\begin{itemize}
\item $\sigma_0$ : valeur initiale de la volatilité

\item $\alpha$ : volatilité (log-normale) de la volatilité (\textit{volvol})

\item $\beta$ : exposant CEV, compris entre 0 et 1%; on l'appelle parfois le \textit{backbone}.

\item $\rho$ : corrélation entre les deux browniens ($d<W^1,W^2>_t = \rho dt$, $\rho\in[-1,1]$)

\end{itemize}
\end{frame}

\begin{frame}{Le modèle SABR}
\begin{figure}[h]
\FIG{7cm}{figures/sabr_impact_param.jpg}
\end{figure}
On considère une option de maturité 1 an sur un taux sous-jacent de forward $F_0=5.00\%$. On part du jeu de paramètres SABR suivant: $\sigma_0=0.03$, $\alpha=0.60$, $\rho=-0.10$, $\beta=0.40$.
\end{frame}

\begin{frame}{Pricing d'une swaption}
\begin{figure}[h]
\FIG{6cm}{figures/SwaptionPrice.jpg}
\end{figure}
\end{frame}


\end{document}
