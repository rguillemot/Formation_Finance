\documentclass{beamer}
\usepackage[english,francais]{babel}
\usepackage[utf8]{inputenc}
\usepackage{multicol}
\usepackage{color}

\usepackage{graphicx}
\graphicspath{{Derives_de_taux/}}


\newcommand{\FIG}[2]{\includegraphics[width=#1]{#2}}

\usetheme{Warsaw}
\title[Produits dérivés de taux]{Valorisation post 2008.}
\author{Antonin Chaix - Richard Guillemot}
\institute{Master IFMA}
\date{20 Février 2015}

\begin{document}

\begin{frame}
\titlepage
\begin{figure}[h]
\centering \FIG{5cm}{figures/UPMC_IFMA.jpg}
\end{figure}
\end{frame}

\begin{frame}{Après la crise de 2008}
Avant la crise de 2008, il était d'usage de se focaliser sur \textbf{les risques de marché} dans la valorisation des produits dérivés.\\
\vspace{1cm}
On dit maintenant considérer:
\begin{itemize}
\item le risque de crédit
\item le risque de liquidité
\end{itemize}
\end{frame}

\end{document}
