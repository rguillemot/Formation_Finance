\documentclass{article}
\usepackage[english,francais]{babel}
\usepackage{multirow}
\usepackage[english,francais]{babel}\usepackage[utf8]{inputenc}
\usepackage{amssymb,amsmath}
\usepackage{color}
\usepackage{graphicx}
\usepackage{enumerate}
\usepackage{comment}
\usepackage{float}
\graphicspath{{./Derives_de_taux/}}
\setlength{\parindent}{0cm}
\newcommand{\FIG}[2]{\includegraphics[width=#1]{#2}}
\newcommand{\indentitem}{\setlength\itemindent{25pt}}

\begin{document}
\title{Evaluation - Produits Dérivés de taux}
\date{24 Février 2017}
\author{Richard GUILLEMOT}
\maketitle

\textbf{Question 1:} Supposons un taux linéaire $R^L$=1\% (100 bps). On considère: 
\begin{itemize}
	\indentitem \item le taux équivalent actuariel de fréquence semestrielle $R^A$.
	\indentitem \item le taux équivalent continu $R^C$.
\end{itemize}
\vspace{0.2cm}
Parmi les assertions suivantes laquelle est correcte:
\begin{enumerate}[a)]
\indentitem \item $R^A$ = 99.75 bps $R^C$ = 99.50 bps.
\indentitem \item $R^A$ = 99.50 bps $R^C$ = 99.75 bps.
\indentitem \item $R^A$ = 100.50 bps $R^C$ = 100.75 bps.
\indentitem \item $R^A$ = 100.75 bps $R^C$ = 100.50 bps.
\end{enumerate}
\smallskip
\centerline{******}
\smallskip
\textbf{Question 2:}
Si on emprunte 240 000 euros pour une durée de 20 ans à un taux de 2\%, quelle mensualité doit on payer pour rembourser le capital et payer les intérêts :
\begin{enumerate}[a)]
\indentitem \item 900 euros.
\indentitem \item 1000 euros.
\indentitem \item 1200 euros.
\indentitem \item 1400 euros.
\end{enumerate}
\smallskip
\centerline{******}
\smallskip
\textbf{Question 3:}
Parmi les produits suivants lesquels sont des opérations OTC (Over The Counter) :
\begin{enumerate}[a)]
\indentitem \item un Money Market.
\indentitem \item un FRA (Forward Rate Agreement).
\indentitem \item un Future.
\indentitem \item un Swap.
\end{enumerate}
\newpage
\smallskip
\centerline{******}
\smallskip
\textbf{Question 4:}
Parmi les produits suivants lequel est insensible à un mouvement de taux d’intérêts :
\begin{enumerate}[a)]
\indentitem \item une obligation à taux fixe le jour de son émission et à ses dates de tombée de coupon.
\indentitem \item une obligation à taux variable le jour de son émission et à ses dates de tombée de coupon.
\indentitem \item un FRA.
\indentitem \item un Swap.
\end{enumerate}
\smallskip
\centerline{******}
\smallskip
\textbf{Question 5:}
On veut calculer la fraction (FA) d’année du 24 Février 2017 au 24 Février 2018 suivant les conventions A360, A365. Choisir la bonne réponse :
\begin{enumerate}[a)]
\indentitem \item $FA^{A365}$ = 0.997 et $FA^{A360}$ = 1
\indentitem \item $FA^{A365}$ = 1 et $FA^{A360}$ = 1.013
\indentitem \item $FA^{A365}$ = 1 et $FA^{A360}$ = 1.
\indentitem \item $FAA365$ = 1.013 et $FA^{A360}$ = 1.
\end{enumerate}
Pour les questions Q4 et Q5, nous supposerons une courbe de taux constante égale à 2\% (taux actuariel à composition annuelle).\\

\smallskip
\centerline{******}
\smallskip
\textbf{Question 6:}
On achète aujourd’hui un contrat Futur MARS 2017  (nominal 1 millions de dollars) à un prix de 99.75. Le Libor 3M vaut aujourd’hui 0.20\%. A la date d’échéance du contrat le 23 Mars 2017. Le Libor 3M a augmenté de 10 bps. Quelle est la somme totale des appels de marge :
\begin{enumerate}[a)]
\indentitem \item on a payé 125 euros.
\indentitem \item on a reçu 250 euros.
\indentitem \item on a reçu 500 euros.
\indentitem \item on a payé 1000 euros.
\end{enumerate}
\newpage
\smallskip
\centerline{******}
\smallskip
\textbf{Question 7:} Parmi les caractéristiques suivantes d’un swap, laquelle n’a pas d’influence sur le calcul du taux de swap :
\begin{enumerate}[a)]
\indentitem \item La maturité du swap.
\indentitem \item La fréquence de la jambe fixe.
\indentitem \item La convention de calcul de la fraction d’année de la jambe fixe.
\indentitem \item La fréquence de la jambe variable.
\end{enumerate}
\smallskip
\centerline{******}
\smallskip
Pour les questions 8 et 9 nous supposerons que la courbe des taux a une valeur constante égale à 2\% (taux actuariel de fréquence annuelle).
\smallskip\\
\textbf{Question 8:} Soit un swap payeur de taux fixe de maturité 15 ans de nominal 200 millions d’euros. Quelle est sa sensibilité :
\begin{enumerate}[a)]
\indentitem \item 2 kEUR/bp
\indentitem \item -25 kEUR/bp.
\indentitem \item +256 kEUR/bp.
\indentitem \item -2569 kEUR/bp.
\end{enumerate}
\smallskip
\centerline{******}
\smallskip
\textbf{Question 9:}
Soit un swap de nominal 100 millions d’euros dont la sensibilité est de -47 kEUR/bp. Parmi les assertions suivantes laquelle est correcte:
\begin{enumerate}[a)]
\indentitem \item le swap est receveur de taux fixe et de maturité 5 ans.
\indentitem \item le swap est payeur de taux fixe et de maturité 5 ans.
\indentitem \item le swap est receveur de taux fixe et de maturité 10 ans.
\indentitem \item le swap est payeur de taux fixe et de maturité 10 ans.
\end{enumerate}
\newpage
\smallskip
\centerline{******}
\smallskip
\begin{figure}[H]
\FIG{15cm}{figures/hedgeratio3.png}
\end{figure}
\textbf{Question 10:}
Considérons l'analyse de sensibilités et les ratios de couverture précédents. Ils correspondent à quel produit:
\begin{enumerate}[a)]
\indentitem \item un swap au pair payeur de taux fixe de maturité 5 ans.
\indentitem \item un swap au pair receveur de taux fixe de maturité 5 ans.
\indentitem \item un swap à démarrage forward 5 ans et de maturité 5 ans payeur de taux fixe.
\indentitem \item un swap à démarrage forward 5 ans et de maturité 5 ans receveur de taux fixe.
\end{enumerate}
\smallskip
\centerline{******}
\smallskip
\textbf{Question 11:}
Quel est le nominal du swap de la question précédente :
\begin{enumerate}[a)]
\indentitem \item 10 millions d’euros.
\indentitem \item 20 millions d’euros.
\indentitem \item 100 millions d’euros.
\indentitem \item 200 millions d’euros.
\end{enumerate}
\newpage
\smallskip
\centerline{******}
\smallskip
\textbf{Question 12:}
Soit une courbe de taux construite à partir des swaps de marché de maturité 1Y,2Y,3Y,4Y,5Y,7Y,10Y.\\
On calcule la couverture de produits dérivés de taux à partir d’une analyse de sensibilité au taux de marché.\\
Quels sont les produits sensibles au moins à 2 plots de la courbe :
\begin{enumerate}[a)]
\indentitem \item le swap de marché 5Y.
\indentitem \item le swap de marché 6Y.
\indentitem \item le swap de marché 7Y.
\indentitem \item le swap à démarrage forward 5 ans et de maturité 5 ans.
\end{enumerate}
Plusieurs réponses sont possibles.

\smallskip
\centerline{******}
\smallskip
\textbf{Question 13:}
Quel est le produit qui apporte le maximum de convexité/concavité (pour un nominal, un sens et une maturité donnés) : 
\begin{enumerate}[a)]
\indentitem \item une jambe fixe de swap.
\indentitem \item une jambe variable de swap.
\indentitem \item un swap.
\indentitem \item une obligation à taux variable.
\end{enumerate}
\smallskip
\centerline{******}
\smallskip
\textbf{Question 14:}
Le taux Libor forward $L(t,T_1,T_2)$ est martingale sous quelle probabilité:
\begin{enumerate}[a)]
\indentitem \item la probabilité risque neutre: $Q$.
\indentitem \item la probabilité forward neutre: $T_1$ $Q^{T1}$.
\indentitem \item la probabilité forward neutre :$T_2$ $Q^{T2}$.
\indentitem \item la probabilité Level à une période allant de $T_1$ à $T_2$: $Q^{LVL}$.
\end{enumerate}
Plusieurs réponses sont possibles.

\smallskip
\centerline{******}
\smallskip
\textbf{Question 15:}
Le swap forward $S(t,T_0, T_n)$ est martingale sous quelle probabilité:
\begin{enumerate}[a)]
\indentitem \item la probabilité risque neutre: $Q$.
\indentitem \item la probabilité forward neutre: $T_0$ $Q^{T0}$.
\indentitem \item la probabilité forward neutre: $T_0$ $Q^{Tn}$.
\indentitem \item la probabilité Level à une période allant de: $T_0$ à $T_n$ $Q^{LVL}$.
\end{enumerate}
\smallskip
\centerline{******}
\smallskip
\newpage
\textbf{Question 16:}
L'écart entre valeur actuelle d'une option et son payoff ou sa valeur intrinsèque est maximale lorsque:
\begin{enumerate}[a)]
\indentitem \item le strike de l'option est dans la monnaie.
\indentitem \item le strike de l'option est en dehors de la monnaie.
\indentitem \item le strike de l'option est à la monnaie.
\indentitem \item indépendant du strike de l'option.
\end{enumerate}
\smallskip
\centerline{******}
\smallskip
\textbf{Question 17:}
Soit un taux d’intérêts $R=2\%$ et sa volatilité normale égale à 1\%. La volatilité lognormale est égale à :
\begin{enumerate}[a)]
\indentitem \item 20\%.
\indentitem \item 30\%.
\indentitem \item 40\%.
\indentitem \item 50\%.
\end{enumerate}
\smallskip
\centerline{******}
\smallskip
\textbf{Question 18:}
L’achat d’une swaption receveuse de taux fixe protège son détenteur d':
\begin{enumerate}[a)]
\indentitem \item une hausse de taux d’intérêts
\indentitem \item une baisse de taux d’intérêts
\indentitem \item un mouvement quelconque de taux
\indentitem \item la hausse du prix d’une obligation à taux fixe de même maturité.
\end{enumerate}
Plusieurs réponses sont possibles.

\newpage
\smallskip
\centerline{******}
\smallskip
\begin{figure}[H]
\FIG{15cm}{figures/schema_swaption.jpg}
\end{figure}
\[
\text{LVL}(t,T_0,T_n)=\sum_{i=1}^{n}\delta_i \;B(t,T_i)
\]
\[
S(t,T_0,T_n)= \frac{B(t,T_0)-B(t,T_n)}{\text{LVL}(t,T_0,T_n)}
\]
\[
\left\{
\begin{split}
&\text{\textbf{BS}}_\text{call}(\tau,K,F,\sigma)=F\mathcal{N}(d_1)-K \mathcal{N}(d_2)\\
&\text{\textbf{BS}}_\text{put}(\tau,K,F,\sigma)=K\mathcal{N}(-d_2)-F \mathcal{N}(-d_1)\\
&\mathcal{N} : \text{fonction de répartition de la loi normale centrée réduite}\\
&d_1=\frac{\ln\left(\frac{F}{K}\right)+\frac{1}{2}\sigma^2\tau}{\sigma\sqrt{\tau}}\\
&d_2=d_1-\sigma\sqrt{\tau}
\end{split}
\right.
\]

\textbf{Question 19:}
Parmi les formules suivantes laquelle correspond à la valeur d’une swaption receveuse de taux fixe $K$, associée à l’échéancier,  avec une hypothèse de taux lognormal :
\begin{enumerate}[a)]
\indentitem \item $B(t,T_n) \;\text{\textbf{BS}}_\text{put}\big(T_f-t,K,S(t,T_{0},T_n),\sigma\big)$
\indentitem \item $\text{LVL}(t,T_0,T_n) \;\text{\textbf{BS}}_\text{put}\big(T_f-t,K,S(t,T_{0},T_n),\sigma\big)$
\indentitem \item $\text{LVL}(t,T_0,T_n) \;\text{\textbf{BS}}_\text{call}\big(T_n-T_0,K,S(t,T_{0},T_n),\sigma\big)$
\indentitem \item $\text{LVL}(t,T_0,T_n) \;\text{\textbf{BS}}_\text{call}\big(T_f-t,K,S(t,T_{0},T_n),\sigma\big)$
\end{enumerate}
Les solutions font référence aux formules définies ci-dessus.
\newpage
\smallskip
\centerline{******}
\smallskip
\textbf{Question 20:} 
Supposons la valeur actuelle d'un taux d'intérêts $R=1\%$ et sa volatilité normale de $\sigma=1\%$. Quelle prime doit payer l'acheteur d'une swaption receveuse à la monnaie de taux fixe de maturité 1 an et de tenor 5 ans (le swap soujacent paiera ses derniers flux dans 6 ans), de nominal 100 millions d'euros:
\begin{enumerate}[a)]
\indentitem \item 500 000 euros.
\indentitem \item 1 000 000 euros.
\indentitem \item 1 500 000 euros.
\indentitem \item 2 000 000 euros.
\end{enumerate}
Donner la valeur la plus proche de la valeur exacte.
\end{document}
