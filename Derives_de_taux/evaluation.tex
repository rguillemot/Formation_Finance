\documentclass{article}
\usepackage[english,francais]{babel}
\usepackage{multirow}
\usepackage[english,francais]{babel}\usepackage[utf8]{inputenc}
\usepackage{amssymb,amsmath}
\usepackage{color}
\usepackage{graphicx}
\graphicspath{{./Derives_de_taux/}}

\newcommand{\FIG}[2]{\includegraphics[width=#1]{#2}}


\begin{document}
\title{Evaluation - Produits Dérivés de taux}
\date{}
\maketitle
\section*{Conventions et standards de marché}
Remplir le tableau suivant avec les caractéristiques d'un swap standard euro et dollar.
\begin{center}
\begin{tabular}{|l|l|c|c|}
\hline
&&\textbf{Euro}&\textbf{Dollar}\\
\hline
&Date de départ&??&??\\
\hline
\multirow{2}{*}{\textbf{Jambe Fixe}}&Base de calcul&??&??\\
&Fréquence de paiement&??&??\\
&Date de paiement&??&??\\
&Ajustement au calendrier&??&??\\
\hline
\multirow{2}{*}{\textbf{Jambe Variable}}&Base de calcul&??&??\\
&Fréquence de paiement&??&??\\
&Fréquence de fixing&??&??\\
&Date de paiement&??&??\\
&Date de fixing&??&??\\ 
&Ajustement au calendrier&??&??\\
\hline
\end{tabular}
\end{center}

\section*{Swaps exotiques}
Soient 2 swaps exotiques en euros, traités entre une banque et son client, avec les caractistiques suivantes:
\begin{itemize}
\item Swap \textbf{ARREARS}:\\
Le client reçoit une jambe fixe euro standard. En échange il paie une jambe variable avec toutes les caractéristiques standards en euro à l'exception du taux variable \textbf{qui est fixé 2 jours avant la fin de période}.
\item Swap \textbf{Capped \& Floored}:\\
Le client paie une jambe variable euro standard. En échange il recoit quasiment la même jambe variable à l'exception des taux qui sont bornés (capped \& floored). Le client reçoit un taux qui peut être au maximum \textbf{10\% plus élevé} qu'un certains niveau fixe $K$ et au minimum \textbf{10 \% moins élevé} que K. On appellera $K$ le taux fixe du swap \textbf{Capped \& Floored}. 
\end{itemize}

\vspace{0.5cm}

On considère que les taux zéro coupon sont contants égaux à 2\% au format exponentiel (composition continue):\\

\vspace{0.5cm}

On suppose de façon classique, que chacun des taux est martingal sous sa probabilité de paiement.On considère les 3 modèles suivants:
\begin{itemize}
\item le modèle lognormal
\item le modèle normal
\item le modèle lognormal décalé
\end{itemize}
\vspace{0.5cm}
On suppose que ces 3 modèles sont cohérents avec les données de marché.
\begin{enumerate}
\item \textbf{Valorisation}: Etablir les formules de valorisation des ces 2 produits pour chacun des 3 modèles.
\item \textbf{Application numérique}: Calculer le taux fixe de ces swaps pour les 3 maturités 10Y, 20Y et 30Y, afin que le swap soit au pair (valeur nulle). Dans le cas du modèle lognormal décalé on considérera:
\begin{itemize}
\item 3 niveaux de volatilité monnaie : $\sigma_{ATM}$ (15\%,20\% et 25\%). 
\item 6 niveaux de décalage $m$ ($1\% \times R$, $10\% \times R$, $100\% \times R$, $10 \times R$, $100 \times R$, $1000 \times R$).
\end{itemize}
\item \textbf{Analyse}:
\begin{itemize}
\item Quelle est la particularité de la valeur du swap Capped \& Floored dans le cas du modèle normal.
\item A partir des résultats précedents discuter de l'intérêt de prendre en compte la pente du smile pour valoriser chacun de ces 2 produits.
\end{itemize} 
\end{enumerate}
Pour les questions 1 et 2 donner les résultats sans justifications en reprenant les notations que vous trouverez dans l'exercice et le polycopié du cours.

\end{document}
