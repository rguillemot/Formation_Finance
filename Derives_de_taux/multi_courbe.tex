\documentclass{article}
\usepackage[english,francais]{babel}
\usepackage{multirow}
\usepackage[english,francais]{babel}\usepackage[utf8]{inputenc}
\usepackage{amssymb,amsmath}
\usepackage{color}
\usepackage{graphicx}
\graphicspath{{./Derives_de_taux/}}

\newcommand{\FIG}[2]{\includegraphics[width=#1]{#2}}


\begin{document}

\section*{L'unicité de la courbe des taux}

Jusqu’à présent nous avons supposé l’existence d’une “unique” courbe des taux dite sans risque. Cette hypothèse a 2 conséquences très pratiques sur la méthode de valorisation:\\
\begin{itemize}

\item Première conséquence, il existe une structure des taux. Si on connaît le continuum des taux zéro coupon au comptant (c’est à dire les dire les taux zéro coupons qui fixent aujourd’hui et pour un tenor quelconque), il est possible fixer le taux d’un emprunt forward, que l'on le réplique par 2 emprunts au comptant.\\

\item Seconde conséquence, toutes les jambes variables de swap qui débutent et se terminent à la même date de fréquence quelconque ont la même valeur. Elles sont équivalentes à recevoir le nominal en date de départ et à le rembourser à la fin: c’est la formule du double nominal. Ainsi le prix d’une obligation qui paie un coupon égal au LIBOR ou à l'EURIBOR vaut toujours 100 aux dates de son échéancier. Bien entendu les taux variables doivent être payés en fin de période en utilisant la base de calcul Exact 360. Dans le cas particulier d’une obligation qui paie quotidiennement le taux \textbf{EONIA} ou \textbf{Federal Fund} \footnote{Les taux EONIA (Euro Overnight Index Average) en euros et le taux Federal Funds sont les taux interbancaires qui prévalent pour un emprunt du jour au lendemain ou "overnight".}, l’obligation est tous les jours au pair (de valeur égae à 100).\\ 

\end{itemize}

Le taux sans risque signifie sans risque de crédit. Chacune des 2 contreparties respectera ses engagements financiers. Dans le cas d’un prêt/emprunt, cette responsabilité repose entièrement sur l’emprunteur. Une fois qu’il a reçu le nominal, tous les flux suivants, le paiement des intérêts et le remboursement, sont à sa charge. Dans le cas des produits dérivés, par exemple un swap de taux, la situation est symétrique et dépend de l’évolution des taux d’intérêts relativement au taux fixe du swap.\\

Avant la crise financière de 2008, les taux interbancaires LIBOR et EURIBOR étaient considérés comme sans risque de crédit. Les courbes de taux associées étaient utilisées pour valoriser tous les produits dérivés de taux. La faillite de la banque Lehman Brothers ou le renflouement par les états de plusieurs grandes banques d'investissement a remis en cause cette hypothèse.\\

\section*{Multiples courbes des taux et risque de liquidité}

Dans le cas d’un prêt emprunt le risque de crédit est principalement supporté par le prêteur, en effet le paiement des flux à venir est la responsabilité de l’emprunteur. En échange l’emprunteur fait face à un nouveau problème: le manque de liquidité. En d’autres termes, l'emprunteur peut ne pas trouver de prêteur pour la maturité dont il a besoin. Il devra emprunter pour une maturité plus courte et trouver un nouveau prêteur pour la période restante. Il risque que le niveau des taux et la liquidité se soient dégradés à ce moment là. Le risque de liquidité et le risque de crédit sont les 2 factettes d’un même problème: le prêteur doit avoir confiance dans l’emprunteur ne doit pas trahir cette confiance.\\

Ce risque de liquidité se matérialise sur le marché des taux d’intérêts sous la formes de swap de basis. 2 parties s’échangent 2 jambes variables de fréquences différentes. Par exemple un swap 10Y EURIBOR 3M contre EURIBOR 6M - Xbps. La valeur théorique du taux 6M, obtenue par composition des taux 3M, doit être réduite d’une certaine marge, la \textbf{marge de basis}, afin que la valeur du swap lors de sa mise en place soit nulle. Cette marge nous indique l’existence de plusieurs courbes de taux associées à chaque tenor 1D, 1M, 3M, 6M, 12M.\\

La courbe un jour, "one day" (1D) ou \textbf{OIS} \footnote{\textbf{OIS} (Overnight Indexed Swap) est un terme générique pour qualifier les swaps ou les taux d'intérêt de fréquence quotidienne (EONIA ou bien Federal Funds).} joue un rôle bien particulier. Elle est associée au prêt/emprunt à taux variable qui paie le taux OIS quotidiennement. C’est le prêt/emprunt qui présente le moins de risque de crédit pour le prêteur et qui par conséquent a le plus faible côut de liquidité pour l’emprunteur. Chaque jour le prêteur récupère son nominal et il est libre de choisir une nouvelle contrepartie, moins risquée. Il apparaît donc comme une bonne solution, de considérer cette courbe comme la courbe sans risque.


\section*{L’usage du collatéral et l’actualisation OIS/EONIA}

L’usage du collatéral (ou hypothèque) est la méthode en pratique la plus utilisée pour supprimer ou tout au moins réduire le risque de crédit d'un produit dérivé. Le débiteur met en gage de l’argent ou des actifs financiers, le collatéral, à hauteur de sa dette. Chaque jour on constate l’évolution de la valeur du produit et chacune des parties poste ou reçoit le montant qui couvre la nouvelle exposition, ce sont les \textbf{appels de marge}. Le détenteur du collatéral doit rémunérer sa contrepartie à un taux dit \textbf{taux de collatéral}. Du fait des échanges quotidiens ce taux est le plus souvent le taux OIS. On remarque que le futur est un cas particulier de la collatéralisation dans lequel le taux de collatéral est nul.\\

En pratique l’usage du collatéral est régi par le Credit Support Annex (CSA) du contrat ISDA (International Swaps and Derivatives Associations), le contrat standard entre 2 contrepartie qui traitent des swaps. Le CSA définit tous les aspects techniques des échanges de collatéral et en particulier la fréquence, la devise de paiement et le taux de collatéral.\\

En pratique l’usage du collatéral ne supprime pas complètement le risque de contrepartie pour 2 raisons:
\begin{itemize}

\item Le suivi du collatéral et le paiement des appels de marge n’est pas effectué en continu. Ainsi peu de temps avant le défaut, l’exposition de crédit peut fortement se dégrader, sans que les mouvements de valeurs soient capturés par les appels de marge.\\

\item Les valeurs des actifs financiers mis en gage comme collatéral peuvent parfois être négativement corrélées avec le défaut de la contrepartie. Ainsi la valeur du collatéral ne couvre plus le montant emprunté par créditeur une fois son défaut survenu. On parle alors de \textbf{“Wrong Way Risk”} (risque de mauvais sens). Ce problème n’existe pas si le collatéral est posté en argent dans la devise de l’opération collatéralisée.\\

\end{itemize}

Valdimir Piterbarg propose une adaptation du cadre “classique” de valorisation des produits dérivés. Il démontre que le drift d’un actif collatéralisé doit être le taux de collatéral sous la probabilité risque neutre. Par conséquent les flux des produits dérivés doivent être actualisés au taux de collatéral. Le taux OIS est utlisé de façon standard dans le contrats CSA. Nous devons donc actualiser les flux des dérivés standards (Money market, FRA, Swap) avec le courbe OIS. Ce nouveau choix d'actualisation nécessite une adaptation de l'algorithme de calibration de la courbe présenté précédemment.\\

\section*{Calibration des différentes courbes}

La liquidité entraîne des cotations distinctes pour des swaps de même maturité mais faisant intervenir des taux de tenors différents. On peut distinguer 2 catégories parmi les produits disponibles sur le marché:\\
\begin{itemize}
\item \textbf{les swaps à taux fixe}: on distingue les différentes cotations de ce swap suivant la fréquence de la jambe variable $K^f$. Nous supposerons que la fréquence jambe fixe est toujours la même (annuelle en euro et semi-annuelle en dollar).\\

\item \textbf{les swaps de basis}: ces produits échangent des jambes variables de fréquences différentes. On retire une marge $m^{f_1-f_2}$ à la jambe de la fréquence la plus faible afin que la valeur du swap soit nulle lors de sa mise en place.\\ 

\end{itemize}

Le tableau ci dessous illustre les différentes cotations possible pour un tenor donné.

\begin{center}
\begin{tabular}{|c|c|c|c|c|c|c|}
\hline
Jambe&Fixe&OIS&1M&3M&6M&12M\\
\hline
Fixe&&$K^{OIS}$&$K^{1M}$&$K^{3M}$&$K^{6M}$&$K^{12M}$\\
OIS&&&$m^{OIS-1M}$&$m^{OIS-3M}$&$m^{OIS-6M}$&$m^{OIS-12M}$\\
1M&&&&$m^{1M-3M}$&$m^{1M-6M}$&$m^{1M-12M}$\\
3M&&&&&$m^{3M-6M}$&$m^{3M-12M}$\\
6M&&&&&&$m^{6M-12M}$\\
\hline
\end{tabular}
\end{center}

Ces cotations sont rédondantes et peuvent être réliées les unes aux autres par des arguments de réplication. L'achat et la vente de 2 swaps payeurs de taux fixe, de fréquences variables distinctes, réplique un swap de basis.\\

Soient 2 échéanciers de fréquence différentes et synchronisés. Nous supposons que le premier échéancier contient $n$ flux et le second $m$ flux. Nous disposons des 3 équations suivantes:
\[
\begin{split}
\sum_{i=1}^{n} \delta^n_i L(T^n_{i-1},T^n_{i}) B(t,T^n_{i})&=K^n LVL(T_0,T_n)\\
\sum_{i=1}^{m} \delta^m_i L(T^m_{i-1},T^m_{i}) B(t,T^m_{i})&=K^m LVL(T_0,T_m)\\
\sum_{i=1}^{n} \delta^n_i L(T^n_{i-1},T^n_{i}) B(t,T^n_{i})&=\sum_{i=1}^{m} \delta^m_i [L(T^m_{i-1},T^m_{i})-m^{n,m}] B(t,T^m_{i})
\end{split}
\]
\begin{itemize}
\item les quantité avec un  \textbf{exposant n} sont associées à la jambe variable de la fréquence la plus grande.\\
\item les quantité avec un \textbf{exposant m} sont associées à la jambe variable de la fréquence la plus faible.\\
\item les quantité \textbf{sans exposant} sont associées à la jambe fixe des 2 swaps.\\
\end{itemize}

\begin{center}
\begin{figure}[h]
\vspace{2mm}
\FIG{15cm}{figures/basis.png} 
\vspace{1mm}
\caption{Les différents échéanciers.}
\end{figure}
\end{center}


On en déduit ainsi que:
\[
K^m-K^n=\frac{LVL^m(T^m_0,T^m_m)}{LVL(T_0,T_n)} m^{n,m}
\]
On peut approximer cette relation:
\[
K^m-K^n \simeq m^{n,m}
\]
On en déduit une "sorte" de relation de Chasles pour les marges de basis:
\[
m^{n,m}+m^{m,p} \simeq m^{n,p}
\]
$m$,$n$ et $p$ correspondent à trois fréquences différentes.\\

Pour établir les relations précédentes nous n'avons fait aucune hypothèse sur la courbe d'actualisation, c'est à dire comment sont calculés les facteurs d'actualisation $B(t,T_i)$ ou les différents levels $LVL^n(T_0,T_n)$,$LVL^m(T_0,T_n)$,$LVL(T_0,T_n)$.\\

On peut considérer les cotation des swaps comme des invariants et elles vont être utilisées comme valeurs cibles pour l'algorithme de calibration des courbes. L'objectif de l'algorithme est de construire une courbe pour chaque fréquence OIS,1M,3M,6M et 12M, que l'on utilisera pour estimer les taux forward des jambes variables associées. Le niveau des taux sera déterminé afin que la valeur de ces swaps de marché soit nulle.\\

L'algorithme:\\
\begin{itemize}
\item \textbf{Etape 1 : la courbe OIS}\\
La courbe OIS joue un rôle bien particulier dans l'algorithme. En effet on utilisera la même courbe pour estimer les taux forwards des swaps OIS et actualiser leurs flux. Par conséquent l'algorithme classique sera utilisé.\\
\item{\textbf{Etape 2 : les courbes de forward}}\\
Si l'on utilise l'algorihme classique de construction de la courbe des taux pour les index non OIS, on ne respectera pas les cotations de marché du fait de l'actualisation. Le nouvel algorithme déterminera donc de nouveaux taux forwards pour chaque index (1M,3M,6M,12M) afin que les swaps de marché soient au pair.\\ 
\end{itemize}

Nous allons illustrer l'algorithme avec un exemple très simple.
On considère:\\
\begin{itemize}
\item les 2 Money Market EURIBOR 6M et EURIBOR 12M,\\
\item les 2 swaps de basis 6M vs EONIA et 12M vs EONIA avec la même marge de basis,\\ 
\item et enfin le swap taux fixe de maturité 12M et de tenor 6M (la jambe fixe est payée annuellement et la jambe variable semi-annuellement). Son taux est égal à l'EURIBOR 12M. On remarque que dans cet exemple la marge 12M vs 6M est nulle.\\
\end{itemize}

\begin{center}
\begin{tabular}{|l|c|c|}
\hline
Produit & Taux (\%) & Symbole\\
\hline
EURIBOR 6M & 1\% & $R_1$ \\
EURIBOR 12M & 1.5\% & $R_2$ \\
EONIA vs BOR 6M & 50bps & $m$\\
EONIA vs BOR 12M & 50bps & $m$\\
SWAP 12M vs 6M & 1.5\% & $R_2$ \\
FRA 6M dans 6M & \textbf{??} & F \\
\hline
\end{tabular}
\end{center}

On veut calculer le taux forward 6M dans 6M, que l'on notera F. On actalisera tous les flux avec la courbe OIS. Il nous faut donc résoudre l'équation suivante:\\
\begin{equation}
\label{eq:multicurvefra}
\delta_1 \times B^{OIS}(t,T_1) \times R_1 + \delta_2 \times B^{OIS}(t,T_2) \times F = (\delta_1 + \delta_2) \times R \times B^{OIS}(t,T_2)
\end{equation}

Par souci de simplicité, on considère que les conventions de calcul des fractions d'années sont les mêmes sur les jambes variables et les jambes fixes.\\

\textbf{L'étape 1} consiste à déterminer les facteurs d'actualisation à partir des swaps OIS. Le choix de la courbe OIS pour l'actualisation permet de les déduire directement par les formules classiques: 

\begin{center}
\begin{tabular}{r l}
$B^{OIS}(t,T_1)=$ & $\frac{1}{1+\delta_1 (R_1-m)}$ \\
$B^{OIS}(t,T_2)=$ & $\frac{1}{1+(\delta_1 + \delta_2) (R_2-m)}$ \\
\end{tabular}
\end{center}

\textbf{L'étape 2} consiste simplement à résoudre l'équation (\ref{eq:multicurvefra}):\\

\[
F=\frac{(\delta_1+\delta_2) R_2}{\delta_2}-R_1 \delta_1 \frac{1+(\delta_1+\delta_2) (R_2-m)}{\delta_2 (1+\delta_1 (R_1-m))}
\]
Après quelques simplifications:\\
\[
F=L^{OIS}(t,T_1,T_2)+m \big[ 1 - \delta_1 L^{OIS}(t,T_1,T_2) \big] \simeq L(t,T_1,T_2) \\
\]
avec:\\
\begin{center}
\begin{tabular}{c c}
$L^{OIS}=\frac{1}{\delta_2}\big[ \frac{B^{OIS}(t,T_1)}{B^{OIS}(t,T_2)}-1 \big]$&
$L=\frac{1}{\delta_2}\big[ \frac{B(t,T_1)}{B(t,T_2)}-1 \big]=\frac{1}{\delta_2}\big[\frac{1+(\delta_1+\delta_2) R_2}{1+\delta_1 R_1}-1\big]$\\
\end{tabular}
\end{center}
La formule précédente nous indique que le nouveau taux F, prenant en compte l'actualisation OIS, est très proche de la formule classique du forward $L(T_1,T_2)$. Le tableau suivant illustre que l'ajustement, c'est à dire $F-L(T_1,T_2)$, est une fraction de point de base.
\begin{center}
\begin{tabular}{|c|c|c|}  
\hline  
\textbf{m=OIS-BOR} & $\textbf{F}$ & \textbf{Ajustement} \\ 
\hline  
0 bps & 1.9900\%&0.00 bps\\ 
10 bps & 1.9905\%&0.05 bps\\ 
20 bps & 1.9910\%&0.10 bps\\ 
30 bps & 1.9915\%&0.15 bps\\ 
40 bps & 1.9920\%&0.20 bps\\ 
\textbf{\color{red}50 bps} & \textbf{\color{red}1.9925\%}&\textbf{\color{red}0.25 bps}\\ 
\hline 
\end{tabular}
\end{center}
\section{Gestion dans un environement multi courbes}
Nous avons vu précédemment que les produits dérivés collatéralisés doivent être actualisés au taux de collatéral standard OIS. L'algorithme calage des courbes des taux doit être adapté en conséquence. En pratique la formule de calcul des taux forward est légèrement ajustée. Si l'on gère un portefeuille qui ne contient que des produits dérivés collatéralisés, l'actualisation OIS a peu d'impact. Un banque se collatéralise seulement avec ses contreparties bancaires. En effet le client d'une banque qui traite un swap adossé à une obligation ne souhaite pas que sa trésorerie soit perturbée par des appels de marges, sur lequel il n'a aucun contrôle. La banque devra alors couvir des opérations non collatéralisés avec des opérations collatéralisées.

Une opération non collatéralisée ne sera pas actalisée au taux OIS mais au taux de financement de la banque plus proche de l'EURIBOR.  
\end{document}
