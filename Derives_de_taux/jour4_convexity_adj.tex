\documentclass{beamer}
\usepackage[english,francais]{babel}
\usepackage[utf8]{inputenc}
\usepackage{multicol}
\usepackage{color}

\usepackage{graphicx}
\graphicspath{{Derives_de_taux/}}


\newcommand{\FIG}[2]{\includegraphics[width=#1]{#2}}

\usetheme{Warsaw}
\title[Produits dérivés de taux]{Ajustements de convexité. }
\author{Antonin Chaix - Richard Guillemot}
\institute{Master IFMA}
\date{27 Février 2014}

\begin{document}

\begin{frame}
\titlepage
\begin{figure}[h]
\centering \FIG{5cm}{figures/UPMC_IFMA.jpg}
\end{figure}
\end{frame}

\begin{frame}{Quizz}
Quelles sont les quantités qui sont martingales sous la mesure risque neutre $Q$.
\begin{itemize}
\item a) $B(t,T)$\textbf{\color{green}VRAI} 
\item b) $L(t,T_1,T_2)=\frac{1}{\delta}\big(\frac{B(t,T_1)}{B(t,T_2)}-1 \big)$\textbf{\color{red}FAUX}
\item c) $\displaystyle LVL(t,T_0,T_n)=\sum_{i=1}^{N}\delta_i B(t,T_i)$\textbf{\color{green}VRAI}
\item d) $S(t,T_0,T_n)=\frac{B(t,T_0)-B(t,T_n)}{LVL(t,T_0,T_n)}$\textbf{\color{red}FAUX}
\item e) $B(t,T_1,T_2)=\frac{B(t,T_2)}{B(t,T_1)}$\textbf{\color{red}FAUX} 
\item f) $\displaystyle LVL(t,T_f,T_0,T_n)=\sum_{i=1}^{N}\delta_i B(t,T_f,T_i)$\textbf{\color{red}FAUX}
\end{itemize}
\end{frame}


\begin{frame}{Quizz}
Quelles sont les quantités qui sont martingales sous les mesures forward neutre $Q^{T_0}, Q^{T_1},Q^{T_2}, Q^{T_n}, Q^{T_f}$.
\begin{itemize}
\item a) $B(t,T)$\textbf{\color{red}FAUX}
\item b) $L(t,T_1,T_2)=\frac{1}{\delta}\big(\frac{B(t,T_1)}{B(t,T_2)}-1 \big)$\textbf{\color{green}VRAI}$Q^{T_2}$
\item c) $\displaystyle LVL(t,T_0,T_n)=\sum_{i=1}^{N}\delta_i B(t,T_i)$\textbf{\color{red}FAUX}
\item d) $S(t,T_0,T_n)=\frac{B(t,T_0)-B(t,T_n)}{LVL(t,T_0,T_n)}$\textbf{\color{red}FAUX}
\item e) $B(t,T_1,T_2)=\frac{B(t,T_2)}{B(t,T_1)}$\textbf{\color{green}VRAI}$Q^{T_1}$ 
\item f) $\displaystyle LVL(t,T_f,T_0,T_n)=\sum_{i=1}^{N}\delta_i B(t,T_f,T_i)$\textbf{\color{green}VRAI}$Q^{T_f}$
\end{itemize}
\end{frame}

\begin{frame}{Quizz}
Quelles sont les quantités qui sont martingales sous la mesure swap neutre $Q^{LVL}$.
\begin{itemize}
\item a) $B(t,T)$\textbf{\color{red}FAUX}
\item b) $L(t,T_1,T_2)=\frac{1}{\delta}\big(\frac{B(t,T_1)}{B(t,T_2)}-1 \big)$\textbf{\color{red}FAUX}
\item c) $\displaystyle LVL(t,T_0,T_n)=\sum_{i=1}^{N}\delta_i B(t,T_i)$\textbf{\color{red}FAUX}
\item d) $S(t,T_0,T_n)=\frac{B(t,T_0)-B(t,T_n)}{LVL(t,T_0,T_n)}$\textbf{\color{green}VRAI}
\item e) $B(t,T_1,T_2)=\frac{B(t,T_2)}{B(t,T_1)}$\textbf{\color{red}FAUX} 
\item f) $\displaystyle LVL(t,T_f,T_0,T_n)=\sum_{i=1}^{N}\delta_i B(t,T_f,T_i)$\textbf{\color{red}FAUX}
\end{itemize}
\end{frame}

\begin{frame}{Quizz}
Supposons que la quantité $B(t,T_1,T_2)$ est lognormal de volatilité $\sigma$ sous la mesure $Q^{T_1}$.\\
\vspace{0.5cm}
Quelle loi suit la quantité $L(t,T_1,T_2):$
\begin{itemize}
\item a) la loi lognormale\textbf{\color{red}FAUX}
\item b) la loi normale\textbf{\color{green}VRAI} $\sigma \times L(0,T_1,T_2) $ 
\item c) la loi lognormal décalée\textbf{\color{green}VRAI} $(\sigma,\frac{1}{\delta})$
\item d) la loi SABR\textbf{\color{red}FAUX}
\end{itemize}
\vspace{0.5cm}
Donnez le paramétrage de ces lois.
\end{frame}


\begin{frame}{Couverture d'un swap en dessous du marché.}
Maintenant nous couvrons le même swap de taux fixe 100bp \textbf{en dessous} du taux de marché.\\
\begin{center}
\begin{tabular}{|c|c|}
\hline
Swap&Sensibilité \\ 
\hline
Swap 1 &-148 kEUR/bp \\ 
Swap 2 &-163 kEUR/bp \\ 
\hline
\end{tabular}
\end{center}
Il nous faut donc traiter 90Mios ($\frac{148}{163} \times$ 100 Mios EUR)  d'euros de swap de marché. 
\begin{center}
\begin{tabular}{|c|c|c|}
\hline
R&PNL&$\Delta$PNL \\ 
\hline
2\% &-16.351 Mios EUR& \\ 
3\% &-16.341 Mios EUR&10.275 kEUR\\ 
1\% &-16.340 Mios EUR&10.982 kEUR\\ 
\hline
\end{tabular}
\end{center}
\end{frame}
\begin{frame}{Couverture d'un swap en dessous du marché.}
On gagne à l'inverse de l'exemple précédent 10 kEUR (la convexité est positive: 8.5 EUR/bp/bp).
On explique facilement le PNL par le même développement limité.
\[
\Delta \text{PNL}=\underbrace{\text{Sensi}}_{0} \times \Delta R + \frac{1}{2} \times \underbrace{\text{Convexité}}_{8.5} \times \underbrace{\Delta R^2}_{2500}=10.5 kEUR
\]
\end{frame}

\begin{frame}{Couverture d'un swap 20 ans par un swap 10 ans.}
On souhaite maintenant couvrir un swap receveur de taux fixe de marché pour un nominal de 100 millions d'euros et de maturité 20 ans par un swap de payeur de taux fixe de marché mais de maturité 10 ans.\\
\begin{center}
\begin{tabular}{|c|c|}
\hline
Swap&Sensibilité \\ 
\hline
Swap 1 &-163 kEUR/bp \\ 
Swap 2 &-90 kEUR/bp \\ 
\hline
\end{tabular}\\
\end{center}
Il nous faut donc traiter 182 Mios ($\frac{163}{90} \times$ 100 Mios EUR)  d'euros de swap de marché de maturité 10 ans.
\begin{center}
\begin{tabular}{|c|c|c|}
\hline
R&PNL&$\Delta$PNL \\
\hline
2\% &0.000 Mios EUR& \\ 
3\% &0.175 Mios EUR&174.843 kEUR\\
1\% &0.187 Mios EUR&186.815 kEUR\\
\hline
\end{tabular}
\end{center}
\end{frame}
\begin{frame}{Couverture d'un swap 20 ans par un swap 10 ans.}
Les 144 EUR/bp/bp de convexité positive nous font gagner 180 kEUR que les taux augmente ou baisse de 50bp.
\[
\Delta \text{PNL}=\underbrace{\text{Sensi}}_{0} \times \Delta R + \frac{1}{2} \times \underbrace{\text{Convexité}}_{144} \times \underbrace{\Delta R^2}_{2500}=180 kEUR
\]
\end{frame}

\begin{frame}{Couverture d'un swap 20 ans par un swap 30 ans.}
On souhaite maintenant couvrir un swap receveur de taux fixe de marché pour un nominal de 100 millions d'euros et de maturité 20 ans par un swap de payeur de taux fixe de marché mais de maturité 30 ans.\\
\begin{center}
\begin{tabular}{|c|c|}
\hline
Swap&Sensibilité \\ 
\hline
Swap 1 &-163 kEUR/bp \\ 
Swap 2 &-224 kEUR/bp \\ 
\hline
\end{tabular}\begin{tabular}{|c|c|}
\end{tabular}
\end{center}
Il nous faut donc traiter 73 Mios ($\frac{163}{224} \times$ 100 Mios EUR)  d'euros de swap de marché de maturité  30 ans.
\begin{center}
\begin{tabular}{|c|c|c|}
\hline
R&PNL&$\Delta$PNL \\ 
\hline
2\% &-0.000 Mios EUR& \\ 
3\% &-0.157 Mios EUR&-157.214 kEUR\\ 
1\% &-0.179 Mios EUR&-178.950 kEUR\\ 
\hline
\end{tabular}
\end{center}
\end{frame}
\begin{frame}{Couverture d'un swap 20 ans par un swap 10 ans.}
Les 133 EUR/bp/bp de convexité négative nous font perdre 166 kEUR que les taux augmente ou baisse de 50bp.
\[
\Delta \text{PNL}=\underbrace{\text{Sensi}}_{0} \times \Delta R + \frac{1}{2} \times \underbrace{\text{Convexité}}_{-133} \times \underbrace{\Delta R^2}_{2500}=-166 kEUR
\]
\end{frame}

\begin{frame}{LIBOR \it{in arrears} \textnormal{- Valorisation}}
\begin{figure}[h]
\FIG{7cm}{figures/schema_fra.jpg} 
\end{figure}

Le contrat \textbf{FRA} paie le flux $[L(T_f,T_1,T_2)-R_A]$ en $T_2$.\\
Le contrat \textbf{LIBOR In Arrears} paie  le flux $[L(T_f,T_1,T_2)-R_B]$ en $T_1$.\\
$R_A$ et $R_B$ sont les taux fixes qui rendent respectivement la valeur de chacun de ces deux contrats nulle.\\
\vspace{0.5cm}
$R_A$ et $R_B$ sont ils égaux ?

\end{frame}

\begin{frame}{LIBOR \it{in arrears \textnormal{- Valorisation}}}
Dans le cas du FRA le calcul est direct, car on calcule l'espérance du LIBOR sous sa probabilité naturelle $Q^{T_2}$:\\
\[
\begin{split}
R_A&=\mathbb{E}_t^{Q^{T_2}}[L(T_f,T_1,T_2)]\\
&=L(t,T_1,T_2)\\
\end{split}
\]
Dans le cas du LIBOR in arrears, le calcul nécessite un changement de probabilité:
\[
\begin{split}
R_B&=\mathbb{E}_t^{Q^{T_1}}[L(T_f,T_1,T_2)]\\
&=\mathbb{E}_t^{Q^{T_2}}\Big[L(T_f,T_1,T_2)\frac{1+\delta L(T_f,T_1,T_2)}{1+\delta L(t,T_1,T_2)}\Big]\\
&=\frac{L(t,T_1,T_2)+\delta \mathbb{E}_t^{Q^{T_2}}[L(T_f,T_1,T_2)^2]}{1+\delta L(t,T_1,T_2)}
\end{split}
\]
\end{frame}

\begin{frame}{LIBOR \it{in arrears} \textnormal{- Valorisation}}
Si on suppose une dynamique log-normale sur le LIBOR :
\[
\left\{
\begin{split}
&dL(t,T_1,T_2)=\sigma\;L(t,T_1,T_2) \;dW_t^{Q^{T_2}}\\
&\text{où } W^{Q^{T_2}}\text{est un mouvement brownien standard sous la mesure }Q^{T_i}
\end{split}
\right.
\]
On peut achever le calcul:
\[
\begin{split}
R_B&=L(t,T_1,T_2)\underbrace{\frac{1+\delta L(t,T_1,T_2)e^{\sigma^2 (T_f-t)}}{1+\delta L(t,T_1,T_2)}}_{\text{Ajustement de convexité}}
\end{split}
\]
\end{frame}

\begin{frame}{LIBOR \it{in arrears} \textnormal{- Gestion}}
Considérons une gestion qui ignore les ajustements de convexité.\\
\vspace{0.5cm}
Pour simplifier le formules, on suppose que $t=0$ et on note $\delta = T_2-T_1$\\
\vspace{0.5cm}
\small
\begin{tabular}{|c|c|c|c|}
\hline
& PV & Sensi $R$ & Sensi $F$\\
\hline
MM & $\frac{1}{(1+T_1 R)}$ & $h_1=\frac{-T_1}{(1+T_1 R)^2}$ & 0 \\
\hline
FRA & $\frac{\delta(F-K)}{(1+T_1 R)(1+\delta F)}$ &  0 &  $h_2=\frac{1+\delta K}{(1+ T_1 R)(1+\delta F)^2}$\\
\hline
ARREARS & $\frac{\delta(F-K^*)}{(1+T_1 R)}$ & $H_1=\frac{-T_1 \delta (F-K^*)}{(1+ T_1 R)^2}$ &  $H_2=\frac{\delta}{(1+ T_1 R)}$\\
\hline
& & Ratio MM & Ratio FRA\\
\hline
&& $r_1=\delta(F-K^*)$ & $\color{red}{r_2=\frac{ (1+\delta F)^2}{1+\delta K^*}}$\\
\hline
\end{tabular}\\
\normalsize
\vspace{0.5cm}
Plus la valeur du FRA augmente pour plus il faut en acheter !
\end{frame}



\end{document}
